\chapter{Metodologia de Desenvolvimento}

Para o desenvolvimento do jogo educacional foi adotada a metodologia ENgAGED,
proposta por \citeonline{battistella2016engaged}. O processo integra princípios
de design instrucional com design de jogos, estruturando o desenvolvimento em
fases sistemáticas que asseguram coerência pedagógica, engajamento sustentado e
avaliação contínua. Essa abordagem é particularmente adequada para projetos que
buscam equilibrar objetivos educacionais com experiências lúdicas envolventes,
evitando a falha comum de priorizar apenas um dos aspectos em detrimento do
outro.

\section{Fases do Processo ENgAGED}

O processo ENgAGED estrutura-se em cinco fases principais, cada uma com
objetivos e artefatos específicos. A seguir, cada fase é descrita e
contextualizada para o desenvolvimento do jogo proposto.

\subsection{Fase 1 - Análise da Unidade Instrucional}

A primeira fase concentra-se na análise do contexto educacional, da definição
do público-alvo e na especificação dos objetivos de aprendizagem que orientarão
todo o desenvolvimento.

\subsubsection*{A1.1 - Especificação da Unidade Instrucional}

A unidade instrucional foi definida como o ensino de conceitos fundamentais de
estruturas de dados (pilha, fila e lista) no contexto de cursos de graduação em
Análise e Desenvolvimento de Sistemas ou disciplinas correlatas de Computação.
O foco está em permitir que os aprendizes compreendam o funcionamento dessas
estruturas e suas operações básicas (inserção, remoção e ordenação) de forma
prática e contextualizada.

Os objetivos de aprendizagem específicos incluem:

\begin{itemize}
  \item Compreender o funcionamento de estruturas de dados lineares (pilha, fila e lista);
  \item Identificar as diferenças entre operações LIFO (último a entrar, primeiro a sair) e FIFO (primeiro a entrar, primeiro a sair);
  \item Aplicar conceitos de manipulação de estruturas em cenários de resolução de problemas;
\end{itemize}

\subsubsection*{A1.2 - Caracterização dos Aprendizes}

O público-alvo consiste em:

\begin{itemize}
  \item Estudantes de graduação em Análise e Desenvolvimento de Sistemas ou cursos correlatos de Computação;
  \item Idade aproximada entre 18 e 30 anos;
  \item Acesso a um computador pessoal equipado com sistema operacional compatível com navegadores modernos e que disponha de dispositivos de entrada, como teclado e mouse.
\end{itemize}

Embora o jogo tenha sido projetado com esse público em mente, sua
acessibilidade permite que qualquer pessoa interessada em jogos de forma geral
possa participar, sem exigir conhecimento prévio explícito de estruturas de
dados.

\subsubsection*{A1.3 - Definição dos Objetivos de Desempenho}

O objetivo de desempenho do jogo é que, ao final da experiência, o jogador seja
capaz de:

\begin{itemize}
  \item Executar operações de inserção e remoção respeitando as regras de cada estrutura;
  \item Entender e Aplicar conceitos de ordenação;
  \item Reconhecer padrões de funcionamento das estruturas através da interação implícita com o jogo.
\end{itemize}

\subsection{Fase 2 - Projeto da Unidade Instrucional}

A segunda fase dedica-se ao design instrucional, definindo como os conteúdos
serão apresentados, como será realizada a avaliação e quais estratégias
pedagógicas serão empregadas.

\subsubsection*{A2.1 - Definição da Avaliação do Aprendiz}

A avaliação é incorporada diretamente nas mecânicas do jogo, funcionando como
mecanismo de feedback contínuo:

\begin{itemize}
  \item \textbf{Avaliação Formativa Implícita}: ao realizar combinações corretas de elementos alquímicos, o jogador recebe feedback positivo instantâneo (sucesso na ação);
  \item \textbf{Avaliação por Penalidade}: combinações incorretas resultam em penalidade (perda de pontos de vida), estimulando o jogador a refletir sobre suas ações e ajustar sua estratégia;
  \item \textbf{Métricas de Desempenho}: ao final de cada fase, são registradas métricas como tempo de conclusão, tentativas, erros e acertos de combinações.
\end{itemize}

Essa abordagem de avaliação está alinhada ao construcionismo, pois permite que
o aprendiz construa seu conhecimento através da exploração, tentativa e erro,
em um ambiente seguro onde o fracasso é parte do processo de descoberta.

\subsubsection*{A2.2 - Definição do Conteúdo e da Estratégia Instrucional}

O conteúdo é distribuído progressivamente através de fases de dificuldade
crescente:

\begin{itemize}
  \item \textbf{Fase 1 - Pilha}: introduz o conceito LIFO. O jogador manipula
    uma estrutura de pilha para combinar elementos de fogo, água, ar ou terra.
    Apenas a remoção de elementos do topo é permitida, refletindo o
    comportamento de uma pilha.

  \item \textbf{Fase 2 - Fila}: apresenta o conceito FIFO. O jogador utiliza
    uma estrutura de fila, onde os elementos devem ser removidos na ordem em
    que foram inseridos, integrando novos inimigos com vulnerabilidades
    específicas.

  \item \textbf{Fase 3 - Lista}: introduz manipulações mais dinâmicas. O
    jogador pode inserir e remover elementos em qualquer posição, representando
    a flexibilidade de uma lista encadeada.

  \item \textbf{Fase 4 - Consumíveis e Ordenação}: apresenta o item de
    ordenação, que reorganiza a estrutura, e outros consumíveis que auxiliam na
    progressão. Integra conceitos de algoritmos de ordenação de forma
    implícita.

  \item \textbf{Fase 5 - Integração}: os desafios finais combinam as três
    estruturas simultaneamente, exigindo que o jogador compreenda quando
    aplicar cada uma e como coordenar ataques complexos.
\end{itemize}

A estratégia instrucional priorizará o \textbf{aprendizado implícito}: os
conceitos não são apresentados verbalmente ou textualmente, mas descobertos
através da interação com as mecânicas do jogo. Dessa forma, a aprendizagem
emerge da necessidade prática de resolver desafios, alinhando-se à filosofia
construcionista.

\subsubsection*{A2.3 - Decisão sobre Desenvolvimento ou Reutilização}

Optou-se pelo \textbf{desenvolvimento original do jogo}. Essa decisão baseou-se
em:

\begin{itemize}
  \item Inexistência de jogos disponíveis que integrem a mecânica de estruturas
    de dados com temática de alquimia e estilo visual de pixel art inspirado em
    clássicos como Mario e Mega Man;
  \item Necessidade de controle total sobre a implementação das estruturas de
    dados para garantir precisão pedagógica;
  \item Oportunidade de customizar completamente a narrativa, visual e
    mecânicas para alinhar com os objetivos educacionais específicos;
  \item Potencial de reaproveitamento futuro do código e assets desenvolvidos
    em outros projetos.
\end{itemize}

\subsubsection*{A2.4 - Revisão do Modelo de Avaliação}

O modelo de avaliação do jogo segue os princípios do \textbf{MEEGA+} (Modelo
para Avaliação de Jogos Educacionais), abordando dimensões de:

\begin{itemize}
  \item \textbf{Usabilidade}: facilidade de aprender e usar o jogo, clareza da interface;
  \item \textbf{Engajamento}: capacidade de manter o interesse do jogador, imersão, diversão;
  \item \textbf{Aprendizagem Percebida}: percepção do jogador sobre o aprendizado obtido, aplicabilidade dos conceitos;
  \item \textbf{Experiência do Usuário}: satisfação geral, disposição para recomendação do jogo.
\end{itemize}

Os feedbacks são implementados de forma imediata no jogo, reforçando a
aprendizagem e orientando o jogador através de:

\begin{itemize}
  \item Feedback visual (mudanças de cor, animações, efeitos particulares);
  \item Feedback sonoro (sons de sucesso ou erro);
  \item Mensagens textuais contextualizadas que reforçam conceitos sem ser excessivamente expositivas.
\end{itemize}

\subsection{Fase 3 - Desenvolvimento do Jogo Educacional}

Esta fase abrange as etapas práticas de implementação: levantamento de
requisitos, concepção visual e narrativa, design de mecânicas e implementação
técnica.

\subsubsection*{A3.1 - Análise do Jogo e Levantamento de Requisitos}

Os requisitos foram estruturados em duas categorias: funcionais e
não-funcionais.

\paragraph*{Requisitos Funcionais}

Os requisitos funcionais definem as funcionalidades que o sistema deve
implementar. Esses requisitos estão detalhados no documento separado
\textit{requisitos-funcionais.pdf} e incluem aspectos como manipulação de
inventários, sistema de combate, progresso do jogador, salvamento de partida e
geração de feedback educacional.

\paragraph*{Requisitos Não-Funcionais}

Os requisitos não-funcionais especificam características de qualidade e
restrições técnicas. Esses requisitos estão detalhados no documento separado
\textit{requisitos-nao-funcionais.pdf} e abrangem aspectos como performance,
compatibilidade, segurança, mantibilidade e acessibilidade.

\subsubsection*{A3.2 - Concepção do Jogo}

\paragraph*{Gênero e Perspectiva}

O jogo é um \textbf{platformer 2D} (jogo de plataforma em duas dimensões) com
perspectiva \textit{side-scrolling}. Essa escolha baseia-se na familiaridade do
público geral com clássicos como Super Mario Bros e Mega Man, facilitando a
adoção do jogo e tornando a experiência intuitiva.

\paragraph*{Temática e Narrativa}

A narrativa é ambientada em um universo de \textbf{alquimia} mística e
medieval. O protagonista é um \textit{plague doctor} (doutor da peste) que teve
sua pesquisa sobre a \textbf{pedra filosofal} roubada por um alquimista rival
enquanto viajava para a capital para apresentá-la. Seu objetivo é recuperar a
pesquisa, navegando por diferentes regiões e enfrentando adversários
durante a sua jornada.

Essa narrativa fornece contexto e motivação para as ações do jogador,
aumentando o engajamento e criando imersão. O enredo também justifica a
presença dos elementos alquímicos como mecanismo
central de combate.

\paragraph*{Mecânicas Centrais}

As mecânicas principais do jogo são:

\begin{itemize}
  \item \textbf{Movimentação e Exploração}: o jogador controla o plague doctor, movendo-se horizontalmente pelos níveis, pulando e interagindo com o ambiente;

  \item \textbf{Sistema de Inventários Estruturados}: o jogador possui três inventários independentes (pilha, fila e lista), cada um representando uma estrutura de dados específica. Os elementos alquímicos (fogo, água, ar, terra) são armazenados nesses inventários;

  \item \textbf{Combinação de Elementos}: para atacar, o jogador deve remover elementos dos inventários e combiná-los corretamente. Por exemplo:
  \begin{itemize}
    \item Dois elementos fogo combinados geram um ataque de fogo;
    \item Dois elementos água combinados geram um ataque de água;
    \item A combinação respeitará as regras da estrutura de origem (LIFO para pilha, FIFO para fila, acesso livre para lista).
  \end{itemize}

  \item \textbf{Vulnerabilidades de Inimigos}: cada inimigo possui vulnerabilidades específicas a um ou mais elementos. Ataques com o elemento correto causam dano;

  \item \textbf{Consumíveis}: itens especiais encontrados durante o jogo auxiliam na progressão:
  \begin{itemize}
    \item \textit{\fixme{Ordenação}}: ordena a estrutura de dados, permitindo reorganizar elementos;
    \item \textit{\fixme{Insert}}: Insere dois elementos iguais em um inventário específico;
    \item \textit{\fixme{Remove}}: remove um elemento de um inventário específico.
    \item \textit{\fixme{Mana}}: Concede mana infinita por um tempo determinado.
    \item \textit{\fixme{Cura}}: Cura 1 ponto de vida;
    \item \textit{\fixme{Vida Extra}}: Aumenta a quantidade de tentativas em 1;
  \end{itemize}
\end{itemize}

\subsubsection*{A3.3 - Design do Jogo}

\paragraph*{Elementos Visuais e Artísticos}

O jogo utiliza \textit{pixel art} como estilo visual, em homenagem aos
clássicos de 8 e 16 bits, facilitando a conexão com o público-alvo e permitindo
desenvolvimento mais ágil. Os sprites do protagonista, inimigos, objetos e
cenários foram criados em pixel art original, mantendo coerência visual e
legibilidade.

A paleta de cores foi escolhida para refletir a temática do personagem
principal, seguindo majoritariamente tons de azul e roxo, tanto mais escuros
quanto pastéis.

\paragraph*{Interface e Comunicação Visual}

A interface foi projetada com princípios de \textbf{minimalismo} e
\textbf{clareza}, evitando sobrecarga visual:

\begin{itemize}
  \item \textbf{Inventário Visual}: os inventários são exibidos na parte
    inferior à direita, mostrando os elementos armazenados em cada
    estrutura através de representações visuais claras;

  \item \textbf{Feedback Visual de Erros}: quando uma combinação incorreta é
    realizada, o jogo exibe feedback visual (mudança de cor do jogador e efeito
    de penalidade) sem interromper a fluidez do \textit{gameplay};

  \item \textbf{Indicadores de Vulnerabilidade}: inimigos exibem visualmente
    seus elementos de vulnerabilidade através de ícones, permitindo que o
    jogador identifique rapidamente a estratégia apropriada.
\end{itemize}

\paragraph*{Sistema de Pontuação e Progressão}

A pontuação é baseada em:

\begin{itemize}
  \item \textbf{Eficiência}: quantos ataques bem-sucedidos foram realizados em relação ao total;
  \item \textbf{Tempo}: bônus por completar a fase rapidamente;
  \item \textbf{Sem Dano}: bônus adicional se o jogador completar a fase sem perder vidas.
\end{itemize}

O progresso é salvo automaticamente em pontos de controle (checkpoints)
estrategicamente posicionados em cada fase, evitando frustração causada por
perda de progresso.

\subsubsection*{A3.4 - Implementação Técnica}

\paragraph*{Ferramentas e Tecnologias}

\begin{itemize}
  \item \textbf{Motor de Jogo}: Unity 2022 LTS, selecionada por sua robustez,
    suporte multiplataforma e capacidade de criar jogos de performance
    eficiente;

  \item \textbf{Linguagem de Programação}: C\#, linguagem nativa da Unity,
    utilizada para implementar toda a lógica do jogo, incluindo estruturas de
    dados e sistemas de \textit{gameplay};

  \item \textbf{Criação de Assets Visuais}: GIMP (GNU Image Manipulation
    Program), software livre utilizado para criar e editar sprites em pixel
    art;

  \item \textbf{Controle de Versão}: Git com repositório remoto, garantindo
    rastreamento de alterações e possibilidade de recuperação de versões
    anteriores.
\end{itemize}

\paragraph*{Arquitetura de Código}

As estruturas de dados (pilha, fila e lista) foram implementadas como classes
independentes em C\#, permitindo:

\begin{itemize}
  \item Reaproveitamento de código entre diferentes contextos do jogo;
  \item Clareza didática: o código das estruturas reflete diretamente os conceitos ensinados;
  \item Testes unitários simplificados para validar o comportamento das estruturas.
\end{itemize}

Cada classe implementa operações fundamentais:

\begin{itemize}
  \item \textbf{Stack (Pilha)}:
    \texttt{Push()},
    \texttt{Pop()},
    \texttt{Peek()}. Respeitando a ordem LIFO.

  \item \textbf{Queue (Fila)}:
    \texttt{Enqueue()},
    \texttt{Dequeue()},
    \texttt{Peek()}. Respeitando a ordem FIFO.

  \item \textbf{List (Lista)}:
    \texttt{Add()},
    \texttt{Insert()},
    \texttt{RemoveAt()},
    \texttt{Get()}. Permitindo acesso livre em qualquer posição.
\end{itemize}

\paragraph*{Sistemas de Jogo Implementados}

\begin{itemize}
  \item \textbf{Sistema de Movimento}: controle do personagem (esquerda,
    direita, pulo), colisão com plataformas e inimigos;

  \item \textbf{Sistema de Combate}: detecção de combinações de elementos,
    validação contra regras de estruturas, cálculo de dano;

  \item \textbf{Sistema de Inventário}: gerenciamento de elementos nos
    inventários, suporte a operações específicas de cada estrutura;

  \item \textbf{Sistema de Vida}: rastreamento da vida do jogador, aplicação de
    penalidades, condição de derrota;

  \item \textbf{Sistema de Progresso}: salvamento e carregamento de estado,
    gestão de fases desbloqueadas, registro de pontuação;

  \item \textbf{Sistema de Feedback}: implementação de mensagens visuais,
    sonoras e textuais que reforçam o aprendizado.
\end{itemize}

\subsubsection*{A3.5 - Testes do Jogo}

Os testes foram conduzidos em múltiplas etapas:

\paragraph*{Testes Funcionais}

Testes para validar se todas as funcionalidades implementadas operam conforme
especificado:

\begin{itemize}
  \item Operações de estruturas de dados (inserção, remoção, ordenação);
  \item Lógica de combate e detecção de vulnerabilidades;
  \item Sistema de salvamento e carregamento;
  \item Interface e feedback visual.
\end{itemize}

\paragraph*{Testes Educacionais}

Testes com usuários reais para validar a eficácia pedagógica:

\begin{itemize}
  \item \textbf{Participantes}: estudantes do curso de Análise e
    Desenvolvimento de Sistemas (conhecimento prévio de programação),
    entusiastas de jogos (sem conhecimento específico de estruturas de dados) e
    pessoas leigas (sem conhecimento técnico);

  \item \textbf{Instrumentos}: observação direta, questionários pós-jogo,
    análise de métricas de desempenho (tempo, tentativas, acertos);

  \item \textbf{Focos de Avaliação}: compreensão implícita dos conceitos,
    engajamento, diversão, clareza das mecânicas, dificuldade progressiva.
\end{itemize}

\paragraph*{Resultados Preliminares}

Os testes iniciais indicaram:

\begin{itemize}
  \item Boa compreensão dos mecanismos de estruturas de dados mesmo entre
    participantes sem conhecimento prévio;
  \item Engajamento sustentado com a temática alquímica e narrativa do \textit{plague doctor};
  \item Feedback visual e mecânicas intuitivas facilitando a adoção;
  \item Necessidade de ajustes na dificuldade de certas fases e clareza de
    alguns consumíveis.
\end{itemize}

\subsection{Fase 4 - Execução da Unidade Instrucional}

Na quarta fase, o jogo será integrado em contextos educacionais reais, sendo
utilizado em sala de aula como ferramenta complementar de ensino.

\subsubsection*{Implementação em Contexto Educacional}

O jogo será disponibilizado como atividade complementar em disciplinas de
Estruturas de Dados de cursos de Computação. Os docentes poderão:

\begin{itemize}
  \item Integrar o jogo como revisão ou pré-aprendizado de conceitos;
  \item Observar o desempenho dos alunos através das métricas de progresso registradas pelo jogo;
  \item Utilizar os resultados como indicadores de compreensão e dificuldades específicas;
  \item Orientar discussões em sala sobre os conceitos descobertos implicitamente durante o jogo.
\end{itemize}

\subsubsection*{Coleta de Dados}

Durante a execução, serão coletados:

\begin{itemize}
  \item Métricas de jogo: tempo de conclusão, tentativas, acertos de
    combinação;
  \item Feedback dos participantes: satisfação, percepção de aprendizado,
    dificuldades encontradas;
  \item Observações qualitativas: comportamento durante o jogo, discussões com
    pares, reações emocionais.
\end{itemize}

\subsection{Fase 5 - Avaliação da Unidade Instrucional}

A avaliação final mede a efetividade global do jogo como ferramenta
educacional.

\subsubsection*{Metodologia de Avaliação}

Os dados coletados serão analisados de acordo com critérios baseados no MEEGA+:

\begin{itemize}
  \item \textbf{Usabilidade}: clareza da interface, facilidade de aprender,
    ausência de bugs que interfiram na experiência;

  \item \textbf{Engajamento}: engajamento emocional, diversão, imersão,
    motivação intrínseca;

  \item \textbf{Aprendizagem Percebida}: percepção do jogador sobre aprendizado
    obtido, aplicabilidade dos conceitos, confiança em usar estruturas de
    dados;

  \item \textbf{Experiência Geral}: satisfação, disposição para recomendação,
    intenção de uso futuro.
\end{itemize}

\subsubsection*{Análise de Resultados}

Os resultados serão analisados através de:

\begin{itemize}
  \item \textbf{Análise Quantitativa}: estatísticas descritivas das métricas de
    desempenho, correlações entre tempo de jogo e compreensão de conceitos;

  \item \textbf{Análise Qualitativa}: análise temática de respostas abertas,
    observações de comportamento, entrevistas;

  \item \textbf{Triangulação}: cruzamento de dados quantitativos e qualitativos
    para validação de resultados.
\end{itemize}

Os resultados serão cotejados com os objetivos de aprendizagem definidos na
Fase 1, verificando se o jogo contribuiu significativamente para o processo de
ensino e aprendizagem de estruturas de dados.

\subsubsection*{Iterações e Melhorias}

Com base nos resultados da avaliação, serão identificadas oportunidades de
melhoria:

\begin{itemize}
  \item Ajustes nas mecânicas e balanceamento de dificuldade;
  \item Refinamentos na narrativa e imersão;
  \item Otimizações de performance e acessibilidade;
  \item Expansões futuras (novos tipos de estruturas, fases adicionais, modos de jogo).
\end{itemize}

\section{Síntese da Metodologia}

A adoção da metodologia ENgAGED garante que o desenvolvimento do jogo mantenha
coerência sistemática, integrando continuamente princípios de design
educacional com design de jogos. Ao estruturar o processo em fases bem
definidas, com artefatos claros e objetivos específicos em cada etapa, a
metodologia assegura que:

\begin{itemize}
  \item O conteúdo educacional está solidamente fundamentado em objetivos de
    aprendizagem bem definidos;

  \item As mecânicas do jogo refletem os conceitos de estruturas de dados de
    forma implícita, promovendo aprendizado significativo;

  \item A experiência lúdica é priorizada, mantendo o jogo divertido e
    engajador;

  \item A avaliação é contínua e sistemática, fornecendo retroalimentação para
    iterações futuras;

  \item O resultado final é uma ferramenta educacional robusta, testada e
    validada para contribuir efetivamente ao processo de ensino e aprendizagem
    de estruturas de dados.
\end{itemize}

Dessa forma, o presente trabalho posiciona-se não apenas como um exercício
acadêmico de desenvolvimento de software, mas como uma contribuição
metodologicamente estruturada ao campo de educação em computação,
especificamente na inovação de estratégias para tornar o ensino de estruturas
de dados mais motivador, prático e efetivo.
