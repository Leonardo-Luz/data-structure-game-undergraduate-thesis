O ensino e a aprendizagem de conceitos fundamentais da área de computação, como estruturas de dados, constituem um desafio recorrente para educadores e estudantes. \fixme{Segundo Almeida et al. (2021), estruturas de dados representam um dos tópicos mais difíceis de serem assimilados por alunos de cursos de computação, principalmente por sua natureza abstrata e distante da realidade cotidiana dos estudantes}. Tais conceitos, por sua natureza abstrata e formal, são tradicionalmente abordados por meio de aulas expositivas e exercícios de codificação. Contudo, essa abordagem frequentemente resulta em baixa motivação e dificuldade de retenção do conteúdo. \fixme{De acordo com Moraes et al. (2018), métodos tradicionais de ensino — baseados apenas na exposição teórica — apresentam baixos índices de eficácia quando aplicados a conteúdos que demandam visualização e manipulação ativa, como é o caso das estruturas de dados}. Soma-se a isso o fato de que as novas gerações de estudantes estão cada vez mais expostas a um fluxo intenso e constante de informações, o que tem moldado seu comportamento para valorizar respostas imediatas e experiências interativas, tornando o ensino convencional ainda menos atrativo.

Nesse cenário, os jogos sérios surgem como uma estratégia educacional promissora ao promover o aprendizado ativo e engajado. Diferente de abordagens instrucionais diretas, jogos sérios podem ser projetados de modo que a aprendizagem ocorra como consequência da interação do jogador com o ambiente, desafios e regras do jogo. Essa perspectiva está alinhada à teoria do construcionismo, proposta por Seymour Papert, segundo a qual o conhecimento é construído ativamente pelos alunos quando estes se envolvem com a criação, exploração e manipulação de artefatos significativos. \fixme{Para Papert (1980), aprender é mais eficaz quando o estudante está engajado na construção de algo tangível, especialmente se essa construção tiver significado pessoal ou cultural}.

Os jogos sérios são definidos como jogos digitais cujo objetivo principal é educar, treinar ou sensibilizar, sem abrir mão do entretenimento. \fixme{Segundo Silva (2019), "um jogo sério deve ser cativante para que o jogador deseje jogá-lo várias vezes, aprendendo enquanto joga"}. Contudo, uma crítica recorrente a essa abordagem é que muitos jogos sérios acabam falhando como jogos: priorizam o conteúdo educativo de forma explícita, relegando a experiência lúdica a segundo plano. De acordo com \cite{de2025codebo}, "os jogos atualmente estão limitados a utilizar os conceitos de programação como seu tema." Essa limitação evidencia um modelo que tende a transformar o jogo em um pretexto para ensinar diretamente, por meio de quizzes ou simulações superficiais.

O presente trabalho propõe uma abordagem alternativa: utilizar mecânicas de jogo que representem, de forma implícita e interativa, conceitos fundamentais de estruturas de dados. Ou seja, ao invés de apresentar diretamente listas, pilhas ou árvores, o jogo deve incorporar esses elementos em sua lógica e estrutura interna, permitindo que o jogador interaja com esses conceitos de forma intuitiva e contextualizada. Dessa maneira, o aprendizado ocorre como consequência da resolução de problemas e da exploração do sistema, e não como resultado de instruções explícitas ou desafios de programação. Diferente de jogos educativos que simulam exercícios de codificação, o objetivo deste trabalho é projetar um jogo no qual os conceitos estruturais — como inserção em filas, remoção em pilhas, ou navegação em árvores — estejam presentes nas ações e decisões tomadas pelo jogador durante o jogo, mesmo que ele não os reconheça explicitamente como tais.

No contexto do curso de Análise e Desenvolvimento de Sistemas, essa proposta representa uma oportunidade de integrar teoria, prática e design de sistemas interativos. O objetivo central deste TCC é investigar como jogos digitais podem ser utilizados como ferramenta de mediação para o ensino de estruturas de dados, explorando tanto aspectos técnicos do desenvolvimento do jogo quanto sua fundamentação pedagógica. A pesquisa adota uma abordagem metodológica híbrida, envolvendo o desenvolvimento de um jogo protótipo e a avaliação de sua eficácia em contextos educacionais.
