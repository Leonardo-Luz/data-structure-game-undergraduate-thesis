\section{Metodologia de Desenvolvimento}
\label{sec:metodologia-desenvolvimento}

A metodologia de desenvolvimento adotada neste projeto baseia-se no processo
ENgAGED (\textit{EducatioNAl GamEs Development}), proposto por
\citeonline{battistella2016engaged}. Esse processo combina princípios de design
instrucional e design de jogos, estruturando o desenvolvimento em etapas
claras, com foco tanto na experiência do jogador quanto na coerência conceitual
do conteúdo representado.  No presente trabalho, o modelo foi adaptado para um
contexto em que o ensino não é explícito, mas ocorre de forma implícita por
meio da mecânica e das interações do jogador com o sistema.

O produto desenvolvido é um jogo eletrônico 2D no estilo \textit{platformer},
com estética em \textit{pixel art} e ambientação inspirada na alquimia
clássica. O jogador assume o papel de um \textit{plague doctor} em uma jornada
para recuperar sua pesquisa sobre a pedra filosofal, roubada por um alquimista
rival. Para enfrentar os inimigos e superar os desafios, o jogador deve
combinar corretamente elementos alquímicos armazenados em diferentes estruturas
de dados (pilhas, filas e listas). Cada estrutura define restrições próprias de
manipulação, e o jogador precisa raciocinar sobre a ordem de acesso e
combinação dos elementos para realizar ataques eficazes. Combinações incorretas
resultam em falhas ou dano ao personagem.

Adicionalmente, os inimigos apresentam fraquezas elementais específicas. Um
inimigo vulnerável ao elemento fogo, por exemplo, só sofre dano quando atacado
com esse elemento, podendo haver múltiplas fraquezas combinadas. Isso reforça a
necessidade de planejar as manipulações das estruturas antes da execução dos
ataques. Também há consumíveis que auxiliam o jogador, com destaque para o
\textit{item de ordenação}, capaz de reorganizar os elementos dentro das
estruturas, criando oportunidades estratégicas para ataques mais eficientes.

Com base no modelo ENgAGED, o processo foi estruturado em cinco fases
principais, inspiradas no ciclo ADDIE (Análise, Projeto, Desenvolvimento,
Implementação e Avaliação), adaptadas para o contexto de um jogo sério
voltado à integração conceitual e ao engajamento lúdico.

\begin{enumerate}
    \item Análise do contexto e definição dos objetivos;
    \item Projeto do jogo e integração conceitual;
    \item Desenvolvimento do protótipo e iterações de refinamento;
    \item Implementação e testes com jogadores;
    \item Avaliação da experiência e da aplicabilidade conceitual.
\end{enumerate}

\subsection{Fase 1 - Análise do Contexto e Definição dos Objetivos}

Nesta fase, foi realizada a caracterização do público e dos objetivos gerais do
projeto:

\begin{itemize}
    \item \textbf{A1.1 Contextualização:} o jogo foi projetado para um público
      amplo, incluindo estudantes e jogadores casuais, sem exigir conhecimento
      prévio sobre estruturas de dados;
    \item \textbf{A1.2 Delimitação conceitual:} os conceitos de pilha, fila e
      lista são representados simbolicamente por inventários alquímicos, cujas
      regras de acesso e combinação simulam suas propriedades computacionais;
    \item \textbf{A1.3 Objetivos gerais:} promover uma experiência envolvente
      que incentive o raciocínio lógico e a compreensão implícita da
      manipulação de dados.
\end{itemize}

\subsection{Fase 2 - Projeto do Jogo e Integração Conceitual}

Com base na análise inicial, foi desenvolvido o projeto conceitual e estético
do jogo, buscando o equilíbrio entre narrativa, jogabilidade e integração
conceitual:

\begin{itemize}
    \item \textbf{P2.1 Design narrativo e temático:} o enredo acompanha a busca
      do protagonista por sua pesquisa roubada, dentro de um universo regido
      pelos quatro elementos da alquimia;
    \item \textbf{P2.2 Mecânica principal:} ataques são realizados pela
      combinação de elementos retirados de estruturas de dados. Combinações
      corretas geram ataques alquímicos eficazes, enquanto erros causam dano ao
      jogador;
    \item \textbf{P2.3 Sistema de fraquezas:} inimigos possuem vulnerabilidades
      elementais, exigindo que o jogador escolha e combine elementos adequados
      ao tipo de inimigo enfrentado. Alguns possuem múltiplas fraquezas,
      incentivando experimentação e planejamento;
    \item \textbf{P2.4 Consumíveis e estratégia:} há itens que influenciam as
      estruturas de dados, como o consumível de ordenação, que reordena os
      elementos internos, permitindo ao jogador ajustar a sequência de acesso e
      criar combinações mais vantajosas;
    \item \textbf{P2.5 Integração pedagógica implícita:} a manipulação dos
      elementos no inventário reproduz operações típicas de estruturas de
      dados, incentivando o jogador a compreender suas restrições e
      possibilidades por meio da prática lúdica;
    \item \textbf{P2.6 Documentação:} elaboração do
      \textit{Game Design Document} (GDD), storyboards e mapa de progressão,
      detalhando a relação entre os conceitos computacionais e as mecânicas de
      jogo.
\end{itemize}

\subsection{Fase 3 - Desenvolvimento e Iterações de Refinamento}

Esta etapa compreendeu a construção do protótipo e o processo iterativo de
testes e aprimoramentos:

\begin{itemize}
    \item \textbf{D3.1 Protótipo funcional:} desenvolvimento de uma versão
      jogável mínima (\textit{MVP}) com movimentação, ataques e inventários
      baseados em estruturas de dados;
    \item \textbf{D3.2 Testes de jogabilidade:} experimentação de mecânicas,
      feedbacks e níveis de dificuldade, com foco na clareza das respostas do
      sistema e equilíbrio entre desafio e diversão;
    \item \textbf{D3.3 Iterações de refinamento:} ajustes visuais, sonoros e de
      balanceamento, otimizando a fluidez das ações e a coerência entre
      jogabilidade e conceito;
    \item \textbf{D3.4 Controle de versão e documentação técnica:} uso de
      versionamento para registro de melhorias e rastreabilidade das decisões
      de design.
\end{itemize}

\subsection{Fase 4 - Implementação e Testes com Jogadores}

Com o protótipo consolidado, foram conduzidos testes exploratórios com
jogadores de diferentes perfis:

\begin{itemize}
    \item \textbf{I4.1 Seleção do público-teste:} inclusão de jogadores
      familiarizados e não familiarizados com estruturas de dados;
    \item \textbf{I4.2 Avaliação de usabilidade e engajamento:} observação da
      curva de aprendizado, da compreensão das regras implícitas e da resposta
      emocional às mecânicas de combinação e fraquezas;
    \item \textbf{I4.3 Ajustes finais:} aprimoramentos em interface, tempo de
      resposta e balanceamento de inimigos e consumíveis.
\end{itemize}

\subsection{Fase 5 - Avaliação da Experiência e da Aplicabilidade Conceitual}

A fase final consistiu na análise da experiência do jogador e da assimilação
dos conceitos implícitos:

\begin{itemize}
    \item \textbf{E5.1 Avaliação de experiência:} aplicação de questionários de
      satisfação e motivação (como o MEEGA) para mensurar imersão, diversão e
      desafio;
    \item \textbf{E5.2 Observação da aplicação conceitual:} análise qualitativa
      sobre a capacidade dos jogadores em reconhecer padrões operacionais
      semelhantes às estruturas de dados representadas;
    \item \textbf{E5.3 Planejamento de melhorias futuras:} proposição de
      aprimoramentos para reforçar a relação entre mecânicas e conceitos
      computacionais.
\end{itemize}

\subsection{Síntese da Metodologia}

Em síntese, o processo ENgAGED foi adaptado para um contexto de aprendizagem
implícita, em que conceitos de estruturas de dados são explorados dentro de uma
experiência lúdica e narrativa. O modelo permitiu estruturar o desenvolvimento
em etapas claras e iterativas, garantindo o equilíbrio entre jogabilidade,
imersão e aplicabilidade conceitual. O resultado buscado é um jogo que, sem
ensinar diretamente, estimule o pensamento lógico e a intuição sobre operações
de manipulação de dados, dentro de um universo envolvente e esteticamente
coeso.
