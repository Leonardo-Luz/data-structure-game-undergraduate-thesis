\chapter{Considerações Finais} \label{cap:consideracoes_finais}

O desenvolvimento deste trabalho teve como objetivo central a criação de um
jogo sério destinado ao ensino de conceitos fundamentais de estruturas de dados
de forma implícita, utilizando mecânicas lúdicas e interativas para
potencializar o aprendizado. Essa proposta surgiu em resposta às dificuldades
identificadas na literatura, como a natureza abstrata dos conceitos e a baixa
motivação dos estudantes na disciplina de Estruturas de Dados
\cite{mtaho2024difficulties}.

A implementação do jogo seguiu a metodologia de desenvolvimento \emph{ENgAGED}
\cite{battistella2016engaged}, garantindo um processo estruturado que envolveu
a definição da estratégia instrucional baseada na Aprendizagem Baseada em Jogos
\cite{coffey2009digital}, a concepção dos elementos do jogo e o desenvolvimento das
mecânicas centrais e genéricas.

\section{Resultados}

Quanto ao desenvolvimento técnico, no que se refere aos requisitos
estabelecidos, dos 16 requisitos funcionais propostos, 94\% foram implementados
integralmente. Entre os 13 requisitos não funcionais, 93\% foram atendidos de
forma satisfatória. O conteúdo da fase inicial foi concluído, introduzindo com
sucesso os conceitos de Pilha, Fila e Lista Encadeada através da mecânica de
gerenciamento de inventário.

A validação preliminar, realizada com 11 participantes utilizando o modelo
MEEGA+ \cite{meega2020} de forma adaptada, indicou uma recepção positiva da
abordagem proposta:
\begin{itemize}
    \item \textbf{Experiência do Jogador:} Os índices de engajamento e diversão
      foram elevados, com médias superiores a 4.5 na escala Likert, validando o
      potencial lúdico da ferramenta.
    \item \textbf{Conteúdo:} A representação dos conceitos obteve média de
      4.45, sugerindo que a metáfora visual dos inventários foi eficaz para o
      ensino implícito.
    \item \textbf{Usabilidade:} A estética foi o ponto mais elogiado (4.82),
      enquanto a operabilidade apresentou a menor média (3.27), evidenciando a
      necessidade de ajustes nos controles e na curva de aprendizado.
\end{itemize}

A principal inovação deste projeto reside na incorporação implícita de
múltiplas estruturas de dados dentro de um contexto narrativo de fantasia.
Diferente de abordagens tradicionais, o jogo não pausa a ação para explicar o
conceito; o jogador deve internalizar a regra da estrutura para sobreviver aos
desafios de combate.

\section{Comparativo com Trabalhos Relacionados}

A \autoref{tab:cmp_jogos_relatos_v_jogo} apresenta uma comparação direta entre
o jogo desenvolvido e os jogos identificados na revisão bibliográfica.
Observa-se que, enquanto a maioria dos jogos analisados adota formatos focados
estritamente em lógica (\emph{Puzzle}), a presente proposta diferencia-se por
integrar estruturas de dados em um gênero de Plataforma 2D com fantasia e
aventura. Essa escolha de design visa reduzir a carga cognitiva inicial e
aumentar a imersão.

\begin{table}[H]
	\caption{Comparação entre os jogos relacionados e o jogo proposto}
	\label{tab:cmp_jogos_relatos_v_jogo}
	\centering
	\footnotesize
	\begin{tabular}{|r|lllll|}
		\hline
    \rowcolor{headergray}
		\textbf{Trabalho} & \textbf{CED} & \textbf{Ensino}   & \textbf{Estilo}      & \textbf{Gênero}   & \textbf{Gratuito} \\
		\hline
		Human R.M.        & -            & Explícito         & \emph{Top Down}       & \emph{Puzzle}     & Não \\
    AlgoBot           & -            & Implícito         & \emph{Top Down}       & \emph{Puzzle}     & Não \\
		MOP'N SPARK       & -            & Implícito         & Plataformer           & \emph{Puzzle}     & Indefinido \\
    Iron Ears         & P,F,L        & Implícito         & \emph{Drag \& Drop}   & \emph{Puzzle}     & Sim \\
    \rowcolor{accent}
    Trabalho Proposto & P,F,L        & Implícito         & Plataformer           & \emph{Aventura}   & Sim \\
		\hline
	\end{tabular}
	\caption*{Fonte: Autor}
\end{table}


De forma complementar, a \autoref{tab:cmp_trabalhos_relatos_v_jogo} compara o
projeto com outros trabalhos acadêmicos. Nota-se que poucos trabalhos integram
múltiplas estruturas simultaneamente em uma narrativa coesa com ensino
implícito.

\begin{table}[H]
	\caption{Comparação entre os trabalhos relacionados e o trabalho proposto}
	\label{tab:cmp_trabalhos_relatos_v_jogo}
	\centering
	\footnotesize
	\begin{tabular}{|r|cllll|}
		\hline
    \rowcolor{headergray}
		\textbf{Trabalho}      & \textbf{JD}  & \textbf{CED} & \textbf{Ensino}    & \textbf{Estilo}      & \textbf{Gênero}   \\
		\hline
		CodingJob              & Sim          & -            & Explícito          & Simulador            & \emph{Puzzle}     \\
		CodeBô                 & Sim          & F,L,P,B      & Implícito          & Isométrico           & \emph{Puzzle}     \\
		CodeBo Unplugged       & Não          & P            & Implícito          & Tabuleiro            & \emph{Puzzle}     \\
		AuxED                  & Sim          & O            & Explícito          & P\&C                 & \emph{Puzzle}     \\
		Prog-poly              & Não          & -            & Explícito          & Tabuleiro            & \emph{Quiz}       \\
    \rowcolor{accent}
    Trabalho Proposto      & Sim          & P,F,L,O      & Implícito         & Plataformer 2D        & Aventura   \\
		\hline
	\end{tabular}
	\caption*{Fonte: Autor}
\end{table}


\section{Limitações do Estudo}

Apesar dos resultados promissores, o trabalho apresenta limitações decorrentes
de restrições e de tempo e escopo, bem como desafios técnicos identificados
durante a validação.

\begin{itemize}
    \item \textbf{Escopo de Conteúdo:} Devido ao tempo hábil para o
      desenvolvimento dos \emph{assets} e programação, apenas a primeira fase
      do jogo foi implementada. Isso restringiu a cobertura pedagógica à
      introdução das estruturas, deixando de fora a conclusão do arco
      narrativo.
    \item \textbf{Validação Metodológica:} A avaliação realizou-se em caráter
      de piloto com um grupo amostral reduzido. As fases 4 e 5 da metodologia
      ENgAGED, que preveem a aplicação em sala de aula, não puderam ser
      realizadas. Isso limita a generalização dos dados sobre a eficácia
      pedagógica.
    \item \textbf{Aspectos Técnicos:} A geração procedural de elementos (RNG)
      apresentou problemas de balanceamento, ocasionalmente criando situações
      de travamento que frustraram os jogadores.
\end{itemize}

\section{Trabalhos Futuros} \label{sec:trabalhos_futuros}

O desenvolvimento deste jogo sério cumpriu os objetivos fundamentais
estabelecidos para a fase de concepção e implementação do protótipo. No
entanto, devido a restrições de cronograma, algumas etapas previstas na
metodologia ENgAGED não puderam ser concluídas no escopo atual. A
\autoref{tab:tf_incompletos} detalha estas etapas, que são essenciais para a
validação pedagógica formal da ferramenta.

\begin{table}[H]
    \caption{Trabalhos Futuros: Objetivos Incompletos}
    \label{tab:tf_incompletos}
    \centering
    \footnotesize
    \begin{tabular}{|p{4cm}|p{10cm}|}
        \hline
        \rowcolor{headergray}
        \textbf{Trabalho Futuro} & \textbf{Descrição} \\
        \hline
        Execução da Unidade Instrucional & Executar a fase 4 da metodologia ENgAGED, aplicando o jogo em contexto real de ensino e realizando coleta sistemática de dados. \\
        \hline
        Avaliação da Unidade Instrucional & Executar a fase 5 da metodologia ENgAGED, avaliando a eficácia pedagógica do jogo em sala de aula. \\
        \hline
        Fase 2 e 3 & Concluir a narrativa do jogo, introduzindo o alquimista rival como chefão final e adicionando desafios pedagógicos complementares. \\
        \hline
    \end{tabular}
    \caption*{Fonte: Autor}
\end{table}


Além das etapas metodológicas pendentes, a avaliação realizada com os usuários
(descrita no \autoref{cap:avaliacao}) forneceu \emph{insights} valiosos sobre
aspectos técnicos e de design que necessitam de refinamento. A
\autoref{tab:tf_feedback} sintetiza as melhorias prioritárias identificadas a
partir da análise qualitativa dos testes.

\begin{table}[H]
    \caption{Trabalhos Futuros: Melhorias baseadas no \emph{Feedback}}
    \label{tab:tf_feedback}
    \centering
    \footnotesize
    \begin{tabular}{|p{4cm}|p{10cm}|}
        \hline
        \rowcolor{headergray}
        \textbf{Melhoria} & \textbf{Justificativa} \\
        \hline
        Balanceamento da geração aleatória de elementos & Ajustar o algoritmo
        de geração de elementos para reduzir a aleatoriedade excessiva,
        evitando situações em que o jogador fica \enquote{travado} sem as
        combinações necessárias. \\
        \hline
        Reformulação do Tutorial & Criar uma introdução mais guiada e interativa, explicando com clareza as mecânicas de dano elemental e combinação, suavizando a curva de aprendizado inicial. \\
        \hline
    \end{tabular}
    \caption*{Fonte: Autor}
\end{table}


Por fim, visando expandir o ciclo de vida do jogo e seu potencial de
engajamento, foram mapeadas oportunidades de evolução que extrapolam o escopo
original. Estas ideias, apresentadas na \autoref{tab:tf_ideias}, buscam
modernizar a acessibilidade e ampliar a profundidade pedagógica da ferramenta.

\begin{table}[H]
    \caption{Trabalhos Futuros: Ideias de Expansão}
    \label{tab:tf_ideias}
    \centering
    \footnotesize
    \begin{tabular}{|p{4cm}|p{10cm}|}
        \hline
        \rowcolor{headergray}
        \textbf{Expansão} & \textbf{Motivação} \\
        \hline
        Versão Mobile & Aumentar o alcance e acessibilidade do jogo, atingindo novos públicos como estudantes que utilizam dispositivos móveis como plataforma principal. \\
        \hline
        Modo Arena & Introduzir elementos \emph{arcade} focados em repetição e treino, reforçando os conceitos de estruturas de dados por meio de desafios rápidos e pontuação competitiva. \\
        \hline
        Modo PvP & Promover competitividade e interação social, permitindo que jogadores utilizem estratégias de gerenciamento de inventário para superar oponentes em tempo real. \\
        \hline
        Novas Estruturas (Árvores/Grafos) & Tornar o jogo mais robusto pedagogicamente, abordando tópicos avançados da disciplina e ampliando o repertório de mecânicas disponíveis. \\
        \hline
    \end{tabular}
    \caption*{Fonte: Autor}
\end{table}


A implementação destes trabalhos futuros visa permitir que o jogo evolua de um
protótipo funcional para uma ferramenta educacional robusta, capaz de atender a
diferentes perfis de aprendizes. Com essas perspectivas de continuidade
estabelecidas, encerra-se este estudo, apresentando as conclusões finais na
\autoref{sec:conclusao} a seguir.

\section{Conclusão} \label{sec:conclusao}

O desenvolvimento deste trabalho atingiu seu objetivo principal ao projetar e
implementar um jogo sério do gênero plataforma 2D que ensina conceitos de
estruturas de dados de forma implícita. A proposta inovou ao utilizar uma
temática de fantasia e alquimia para contextualizar o aprendizado,
diferenciando-se das abordagens tradicionais baseadas em \emph{puzzles}
isolados ou questionários gamificados.

A validação do protótipo demonstrou que é possível mapear operações abstratas
de memória (inserção e remoção em Pilhas, Filas e Listas) para mecânicas
concretas de gerenciamento de inventário e combate. Os resultados obtidos
indicam que essa abordagem favorece o engajamento e a motivação, mitigando a
barreira inicial causada pela natureza abstrata da disciplina de Estruturas de
Dados.

Embora limitações técnicas relacionadas à aleatoriedade e aos controles tenham
sido identificadas, elas não invalidam a eficácia da metáfora pedagógica
construída. Pelo contrário, apontam caminhos claros para o refinamento da
ferramenta. Assim, conclui-se que o projeto entrega uma contribuição relevante
ao fugir dos modelos convencionais de jogos educativos, provando que o gênero
plataforma 2D pode ser eficaz para o ensino de computação. O jogo estabelece,
portanto, um precedente para o desenvolvimento de ferramentas motivadoras que
priorizam a experiência do jogador sem sacrificar a fidelidade do conteúdo
técnico.
