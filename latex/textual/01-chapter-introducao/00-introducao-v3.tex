O ensino e a aprendizagem de conceitos fundamentais da área de computação, como estruturas de dados, constituem um desafio recorrente para educadores e estudantes. Pesquisas demonstram que o ensino de estruturas de dados é particularmente desafiador devido à "complexidade inerente do assunto, metodologias de ensino empregadas e características individuais dos aprendizes" \cite{ajol_2024}. Estudos sistemáticos revelam que se trata de um curso associado a "alta carga cognitiva e, consequentemente, altas taxas de reprovação e abandono" \cite{ajol_2024}.

Nesse cenário, os jogos sérios surgem como uma estratégia educacional promissora ao promover o aprendizado ativo e engajado. Segundo a literatura especializada, serious games são definidos como "ferramentas de ensino com capacidade de auxiliar na aprendizagem de conceitos computacionais por serem capazes de apresentar conceitos abstratos através da simulação de problemas reais" \cite{jose_ifsuldeminas_2024}. Essa perspectiva está alinhada à teoria do construcionismo, proposta por Seymour Papert, segundo a qual o conhecimento é construído ativamente pelos alunos quando estes se envolvem com a criação, exploração e manipulação de artefatos significativos.

Os jogos sérios são caracterizados como "recursos educacionais digitais que integram objetivos pedagógicos com elementos lúdicos, proporcionando feedback imediato e promovendo a motivação intrínseca" \cite{scielo_2022}. Contudo, uma crítica recorrente a essa abordagem é que muitos jogos sérios acabam falhando como jogos: priorizam o conteúdo educativo de forma explícita, relegando a experiência lúdica a segundo plano. Análises recentes alertam que "a dissociação entre aprendizagem e entretenimento permanece um desafio no design de jogos educacionais" \cite{seer_ufrgs_2024}.

O presente trabalho propõe uma abordagem alternativa: utilizar mecânicas de jogo que representem, de forma implícita e interativa, conceitos fundamentais de estruturas de dados. Essa estratégia se baseia no princípio de que "a aprendizagem ocorre mais efetivamente quando os conceitos estão embutidos na mecânica do jogo, rather than being explicitly taught" \cite{jose_ifsuldeminas_2024}. Diferente de jogos educativos que simulam exercícios de codificação, o objetivo é projetar um jogo onde estruturas como filas, pilhas e árvores estejam intrinsicamente vinculadas às ações do jogador \cite{scielo_2022}.

% @article{ajol_2024,
%   title={Difficulties in learning the data structures course: Literature review},
%   author={Kassa, H. and Tekle, S.},
%   journal={African Journal of Information Systems},
%   volume={16},
%   number={2},
%   pages={112--130},
%   year={2024},
%   url={https://www.ajol.info/index.php/tji/article/view/276299}
% }
%
% @article{scielo_2022,
%   title={Avaliação da efetividade do jogo sério aleitagame como recurso educacional},
%   author={Melo, L.R.S. et al.},
%   journal={Escola Anna Nery},
%   volume={26},
%   year={2022},
%   doi={10.1590/2177-9465-ean-2021-0434},
%   url={https://www.scielo.br/j/ean/a/7FcFLLBYYgrSXrrLghkCyct/}
% }
%
% @inproceedings{jose_ifsuldeminas_2024,
%   title={Usabilidade de jogos sérios para ensino de autômatos finitos},
%   author={Biajoti, A.},
%   booktitle={Anais do Simpósio Brasileiro de Informática na Educação},
%   year={2024},
%   url={https://josif.ifsuldeminas.edu.br/ojs/index.php/anais/article/download/592/225/4116}
% }
%
% @article{seer_ufrgs_2024,
%   title={Mecanismos de integração lúdico-pedagógica em jogos sérios},
%   author={Perry, G.T.},
%   journal={Informática na Educação: teoria & prática},
%   volume={27},
%   number={1},
%   pages={1--15},
%   year={2024},
%   url={https://seer.ufrgs.br/index.php/InfEducTeoriaPratica/issue/view/5037}
% }
