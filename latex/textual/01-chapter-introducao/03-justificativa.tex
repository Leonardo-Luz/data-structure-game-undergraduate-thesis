\section{Justificativa}

A proposta deste trabalho encontra respaldo na demanda por tornar o
ensino de estruturas de dados mais motivador e alinhado às expectativas das
novas gerações de aprendizes. O uso de jogos sérios como recurso educacional
permite contextualizar os conceitos dentro de uma narrativa envolvente,
aumentando o engajamento e favorecendo a construção do conhecimento de forma
mais prática e intuitiva \cite{mouaheb2012serious}.

Além disso, a adoção da metodologia GAMED \cite{aslan2015gamed} no processo de desenvolvimento
garante uma abordagem sistemática e centrada no aprendizado, permitindo que os
objetivos educacionais sejam alcançados sem comprometer a experiência lúdica.

Dessa forma, este trabalho justifica-se por buscar uma alternativa para o
ensino de estruturas de dados, conforme recomendado por
\cite{mtaho2024difficulties}, pretendendo contribuir para a formação de
profissionais mais preparados, criativos e capazes de aplicar o conhecimento de
maneira prática e estratégica no mercado de trabalho.
