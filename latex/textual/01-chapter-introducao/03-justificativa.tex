\section{Justificativa}

A proposta deste trabalho encontra respaldo na necessidade de tornar o ensino de estruturas de dados mais dinâmico, significativo e alinhado às expectativas das novas gerações de aprendizes. O uso de jogos sérios como recurso educacional permite contextualizar os conceitos dentro de uma narrativa ou desafio envolvente, aumentando o engajamento e favorecendo a construção do conhecimento de forma mais prática e intuitiva.

Além disso, a adoção da metodologia GAMED no processo de desenvolvimento garante uma abordagem sistemática e centrada no aprendizado, permitindo que os objetivos educacionais sejam alcançados sem comprometer a experiência lúdica. Dessa forma, este trabalho justifica-se por buscar uma alternativa inovadora e eficaz para o ensino de estruturas de dados, contribuindo para a formação de profissionais mais preparados, criativos e capazes de aplicar o conhecimento de maneira prática e estratégica no mercado de trabalho.
