\section{Justificativa}

A proposta deste trabalho encontra respaldo na demanda por tornar o ensino de
estruturas de dados mais motivador \cite{mtaho2024difficulties} e alinhado às
expectativas das novas gerações de aprendizes. O uso de jogos sérios como
recurso educacional permite contextualizar os conceitos dentro de uma narrativa
envolvente, aumentando o engajamento e favorecendo a construção do conhecimento
de forma mais prática e intuitiva \cite{mouaheb2012serious}, contribuindo para
mitigar as duas principais dificuldades enfrentadas pelos alunos na disciplina
de estrutura de dados, conforme apresentado na
\autoref{tab:dif_data_structure}.

\begin{table}[H]
	\caption{Fatores que contribuem para as dificuldades no aprendizado da disciplina de estruturas de dados}
	\label{tab:dif_data_structure}
	\centering
	\footnotesize
	\begin{tabular}{|lccc|}
\hline
\rowcolor{headergray}
		\textbf{Fator} & \textbf{Frequência} & \textbf{Percentual} & \textbf{Rank} \\
\hline
		\rowcolor{accent}
    Natureza abstrata dos conceitos de estruturas de dados & 13 & 31.0 & 1 \\
		\rowcolor{accent}
    Baixa motivação dos alunos & 10 & 23.8 & 2 \\
    Natureza multidimensional dos conceitos de estruturas de dados & 9 & 21.4 & 3 \\
    Natureza dinâmica dos conceitos de estruturas de dados & 8 & 19.0 & 4 \\
    Metodologia de ensino inadequada & 7 & 16.7 & 5 \\
    Conhecimento prévio deficiente dos alunos & 7 & 16.7 & 6 \\
    Modelo mental defeituoso dos alunos & 7 & 16.7 & 7 \\
    Organização ineficaz dos materiais de aprendizagem & 6 & 14.3 & 8 \\
    Dificuldades no planejamento da solução do programa & 5 & 11.9 & 9 \\
    Organização e implementação ineficaz do currículo & 4 & 9.5 & 10 \\
\hline
	\end{tabular}
  \caption*{Fonte: \cite{mtaho2024difficulties}, editado pelo Autor}
\end{table}


Além disso, a adoção da metodologia ENgAGED \cite{battistella2016engaged} no
processo de desenvolvimento garante uma abordagem sistemática, permitindo que
os objetivos educacionais sejam alcançados sem comprometer a experiência
lúdica.

Dessa forma, este trabalho justifica-se por buscar uma alternativa para o
ensino de estruturas de dados, conforme recomendado por
\citeonline{mtaho2024difficulties}, pretendendo contribuir para a formação de
profissionais mais criativos e capazes de aplicar o conhecimento de maneira
prática e estratégica.

No capítulo seguinte, apresenta-se o referencial teórico
que embasa esta pesquisa, fornecendo os fundamentos conceituais necessários
para compreender o desenvolvimento do trabalho.
