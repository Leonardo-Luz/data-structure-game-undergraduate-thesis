O ensino e a aprendizagem de conceitos fundamentais da área de computação, como estruturas de dados, constituem um desafio recorrente para educadores e estudantes. O autor \cite{mtaho2024difficulties} aponta que a matéria de estrutura de dados é uma das mais difíceis na área da computação e ranqueia as maiores causas desta dificuldade, sendo a primeira no seu ranking a natureza abstrata dos conceitos de estrutura de dados e, em segundo, a baixa motivação dos estudantes. Isso ocorre porque tais conceitos, por sua natureza abstrata e formal, são tradicionalmente abordados por meio de aulas expositivas e exercícios de codificação. Contudo, essa abordagem frequentemente resulta em baixa motivação e dificuldade de retenção do conteúdo. Soma-se a isso o fato de que as novas gerações de estudantes estão cada vez mais expostas a um fluxo intenso e constante de informações, o que tem moldado seu comportamento para valorizar respostas imediatas e experiências interativas, tornando o ensino convencional ainda menos atrativo.

Nesse cenário, os jogos sérios surgem como uma estratégia educacional promissora ao promover o aprendizado ativo e engajado. Diferente de abordagens instrucionais diretas, jogos sérios podem ser projetados de modo que a aprendizagem ocorra como consequência da interação do jogador com o ambiente, desafios e regras do jogo. \fixme{Essa perspectiva está alinhada à teoria do construcionismo, proposta por Seymour Papert, segundo a qual o conhecimento é construído ativamente pelos alunos quando estes se envolvem com a criação, exploração e manipulação de artefatos significativos.}

Os jogos sérios são definidos como uma aplicação de videogames cujo objetivo principal é educar, treinar ou sensibilizar, sem abrir mão do entretenimento. \cite{mouaheb2012serious}. Contudo, uma crítica recorrente a essa abordagem é que muitos jogos sérios acabam falhando como jogos: priorizam o conteúdo educativo de forma explícita, relegando a experiência lúdica a segundo plano \fixme{quando, na verdade, o ensino e o entreterimento deveriam estar em paralelo cite mouaheb2012serious}. De acordo com \cite{de2025codebo}, os jogos atualmente estão limitados a utilizar os conceitos de programação como seu tema e não os incorporam em suas mecânicas. Essa limitação evidencia um modelo que tende a transformar o jogo em um pretexto para ensinar diretamente, por meio de questionários ou simulações superficiais.

O presente trabalho propõe uma abordagem alternativa: utilizar mecânicas de jogo que representem, de forma implícita e interativa, conceitos fundamentais de estruturas de dados. Ou seja, ao invés de apresentar diretamente listas, pilhas ou árvores, o jogo deve incorporar esses elementos em sua lógica e estrutura interna, permitindo que o jogador interaja com esses conceitos de forma intuitiva e contextualizada. Dessa maneira, o aprendizado ocorre como consequência da resolução de problemas e da exploração do sistema, e não como resultado de instruções explícitas ou desafios de programação.

Diferentemente de jogos educativos que simulam exercícios de codificação, o objetivo deste trabalho é projetar um jogo no qual os conceitos ensinados estejam presentes nas ações tomadas pelo jogador, mesmo que ele não os reconheça explicitamente como tais.

A partir de sua pesquisa, o autor \cite{mtaho2024difficulties} recomenda a adoção de novas estratégias de ensino para corrigir as dificuldades e melhorar a experiência de aprendizagem dos estudantes que aprendem estrutura de dados.
