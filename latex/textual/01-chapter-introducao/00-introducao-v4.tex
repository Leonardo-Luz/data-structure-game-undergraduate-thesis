O ensino e a aprendizagem de conceitos fundamentais da área de computação, como
estruturas de dados, constituem um desafio recorrente para educadores e
estudantes. De acordo com o autor \citeonline{mtaho2024difficulties}, a
disciplina de estruturas de dados é altamente exigente para estudantes de
ciência da computação, sendo frequentemente associada a uma elevada carga
cognitiva e, consequentemente, a altas taxas de reprovação e evasão do curso.
Entre os principais fatores que contribuem para essas dificuldades estão a
natureza abstrata dos conceitos envolvidos e a baixa motivação dos alunos. Esse
cenário se agrava pelo fato de que, tradicionalmente, o ensino desses conteúdos
ocorre por meio de aulas expositivas e exercícios de codificação, o que tende a
gerar baixa retenção do conteúdo e desinteresse por parte dos estudantes.

Além disso, as novas gerações de estudantes estão cada vez mais habituadas a um
fluxo constante de informações e experiências interativas, desenvolvendo um
comportamento que valoriza respostas rápidas e estímulos visuais. Isso torna o
ensino convencional ainda menos atrativo. \fixme{A partir de sua pesquisa, o
autor \citeonline{mtaho2024difficulties} recomenda a adoção de novas estratégias de
ensino para mitigar as dificuldades e melhorar a experiência de aprendizagem de
estrutura de dados.}

Nesse cenário, os jogos sérios surgem como uma estratégia educacional
promissora, ao promover o aprendizado ativo e engajado. Diferentemente de
abordagens instrucionais diretas, os jogos sérios podem ser projetados para que
a aprendizagem ocorra como consequência da interação do jogador com o ambiente,
desafios e regras do jogo. Essa perspectiva está alinhada à teoria do
construcionismo, proposta por \citeonline{papert1993children}, segundo a qual o
conhecimento é construído ativamente pelos alunos quando estes se envolvem com
a criação, exploração e manipulação de artefatos significativos.

Jogos sérios são definidos como uma aplicação de videogames cujo objetivo
principal é educar, treinar ou sensibilizar, sem abrir mão do entretenimento
\cite{mouaheb2012serious}. Contudo, uma crítica recorrente a essa abordagem é
que muitos desses jogos sérios falham como jogos, priorizam o conteúdo educativo de
forma explícita, relegando a experiência lúdica a segundo plano, \fixme{quando, na
verdade, ensino e entretenimento deveriam caminhar lado a lado
\cite{mouaheb2012serious}.}

De acordo com \citeonline{de2025codebo}, atualmente, grande parte dos jogos sérios se
utilizam os conceitos de programação apenas como tema, sem integrá-los
verdadeiramente às suas mecânicas. Essa limitação evidencia um modelo que tende
a transformar o jogo em um pretexto para ensinar diretamente, por meio de
\fixme{mecânicas expositivas como} questionários ou simulações superficiais.

O presente trabalho propõe uma abordagem alternativa: utilizar mecânicas de
jogo que representem, de forma implícita e interativa, conceitos fundamentais
de estruturas de dados. Em vez de apresentar diretamente listas, pilhas ou
filas, o jogo incorporará esses elementos em sua lógica e estrutura interna,
permitindo que o jogador interaja com tais conceitos de forma intuitiva e
contextualizada. Dessa maneira, o aprendizado ocorre como consequência da
resolução de problemas e da exploração do sistema, e não como resultado de
instruções explícitas ou desafios de programação.

Diferentemente de jogos educativos que simulam exercícios de codificação, o
objetivo deste trabalho é projetar um jogo no qual os conceitos ensinados
estejam presentes nas ações tomadas pelo jogador, mesmo que ele não os
reconheça explicitamente como tais. Na próxima seção, o objetivo geral deste
trabalho será aprofundado.
