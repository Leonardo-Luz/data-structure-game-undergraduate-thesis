\chapter{Considerações Finais} \label{cap:consideracoes_finais}

O desenvolvimento deste trabalho teve como propósito central a criação de um
jogo sério destinado ao ensino de conceitos fundamentais de estruturas de dados
de forma implícita, utilizando mecânicas lúdicas e interativas para
potencializar o aprendizado. Essa proposta surge em resposta às dificuldades
identificadas na literatura, como a natureza abstrata dos conceitos e a baixa
motivação dos estudantes na disciplina de Estruturas de Dados
\cite{mtaho2024difficulties}.

A implementação do jogo seguiu a metodologia de desenvolvimento ENgAGED
\cite{battistella2016engaged}, garantindo um processo estruturado que envolveu
a definição dos requisitos funcionais e não funcionais, a concepção dos
elementos do jogo, o desenvolvimento das mecânicas centrais e genéricas e, por
fim, a avaliação preliminar utilizando uma adaptação do modelo MEEGA+
\cite{meega2020}.

\fixme{No que se refere aos requisitos estabelecidos, dos x requisitos funcionais
propostos, y\% foram implementados integralmente. Entre os z requisitos não
funcionais, h\% foram atendidos de forma satisfatória}. Quanto ao conteúdo do
jogo, apenas a fase inicial, considerada fundamental por introduzir os
conceitos básicos de estruturas de dados, foi desenvolvida. As fases
subsequentes, com foco maior no enredo e em estruturas de dados mais avançadas,
são indicadas como continuidade natural deste trabalho e poderão expandir a
complexidade pedagógica da experiência.

A inovação deste projeto está na incorporação implícita de múltiplas estruturas
de dados, como pilha, fila e lista encadeada, dentro de um contexto lúdico e
narrativo que incentiva a aprendizagem por meio da exploração e da resolução de
desafios. O estilo plataforma 2D e o gênero aventura, aliados ao tema de
fantasia alquímica, ampliam o potencial de engajamento e imersão do jogador,
diferenciando esta proposta de abordagens mais tradicionais em jogos
educacionais.

A \autoref{tab:cmp_jogos_relatos_v_jogo} apresenta uma comparação direta entre o
jogo desenvolvido e os jogos identificados nos trabalhos relacionados. Observa-se
que, enquanto a maioria dos jogos analisados adota formatos predominantemente
focados em lógica e resolução de problemas no estilo \emph{Puzzle}, a presente
proposta diferencia-se por integrar estruturas de dados em um contexto de ação
e exploração, combinando uma abordagem implícita de ensino com um gênero pouco
explorado nesse domínio: a aventura em plataforma 2D. Essa combinação reforça o
potencial de engajamento e amplia a imersão e motivação.

\begin{table}[H]
	\caption{Comparação entre os jogos relacionados e o jogo proposto}
	\label{tab:cmp_jogos_relatos_v_jogo}
	\centering
	\footnotesize
	\begin{tabular}{|r|lllll|}
		\hline
    \rowcolor{headergray}
		\textbf{Trabalho} & \textbf{CED} & \textbf{Ensino}   & \textbf{Estilo}      & \textbf{Gênero}   & \textbf{Gratuito} \\
		\hline
		Human R.M.        & -            & Explícito         & \emph{Top Down}       & \emph{Puzzle}     & Não \\
    AlgoBot           & -            & Implícito         & \emph{Top Down}       & \emph{Puzzle}     & Não \\
		MOP'N SPARK       & -            & Implícito         & Plataformer           & \emph{Puzzle}     & Indefinido \\
    Iron Ears         & P,F,L        & Implícito         & \emph{Drag \& Drop}   & \emph{Puzzle}     & Sim \\
    \rowcolor{accent}
    Trabalho Proposto & P,F,L        & Implícito         & Plataformer           & \emph{Aventura}   & Sim \\
		\hline
	\end{tabular}
	\caption*{Fonte: Autor}
\end{table}


De forma complementar, a \autoref{tab:cmp_trabalhos_relatos_v_jogo} compara o
projeto proposto com outros trabalhos acadêmicos que também abordam jogos
sérios. Nota-se que, embora diversos trabalhos incluam algum tipo de
estrutura de dados ou foco em lógica computacional, poucos integram múltiplas
estruturas simultaneamente dentro de uma narrativa coesa. Além disso, enquanto
muitos se concentram no ensino explícito, o presente trabalho aposta na
aprendizagem implícita, característica alinhada à literatura que associa esse
modelo ao aumento da motivação e à redução da carga cognitiva inicial.
Esses elementos reforçam a originalidade e a contribuição prática do jogo como
ferramenta complementar de ensino.

\begin{table}[H]
	\caption{Comparação entre os trabalhos relacionados e o trabalho proposto}
	\label{tab:cmp_trabalhos_relatos_v_jogo}
	\centering
	\footnotesize
	\begin{tabular}{|r|cllll|}
		\hline
    \rowcolor{headergray}
		\textbf{Trabalho}      & \textbf{JD}  & \textbf{CED} & \textbf{Ensino}    & \textbf{Estilo}      & \textbf{Gênero}   \\
		\hline
		CodingJob              & Sim          & -            & Explícito          & Simulador            & \emph{Puzzle}     \\
		CodeBô                 & Sim          & F,L,P,B      & Implícito          & Isométrico           & \emph{Puzzle}     \\
		CodeBo Unplugged       & Não          & P            & Implícito          & Tabuleiro            & \emph{Puzzle}     \\
		AuxED                  & Sim          & O            & Explícito          & P\&C                 & \emph{Puzzle}     \\
		Prog-poly              & Não          & -            & Explícito          & Tabuleiro            & \emph{Quiz}       \\
    \rowcolor{accent}
    Trabalho Proposto      & Sim          & P,F,L,O      & Implícito         & Plataformer 2D        & Aventura   \\
		\hline
	\end{tabular}
	\caption*{Fonte: Autor}
\end{table}


\section{Trabalhos Futuros} \label{sec:trabalhos_futuros}

O desenvolvimento deste jogo sério cumpriu os objetivos fundamentais
estabelecidos para a fase de concepção e implementação do protótipo. No
entanto, devido a restrições de cronograma, algumas etapas previstas na
metodologia ENgAGED não puderam ser concluídas no escopo atual. A
\autoref{tab:tf_incompletos} detalha estas etapas, que são essenciais para a
validação pedagógica formal da ferramenta.

\begin{table}[H]
    \caption{Trabalhos Futuros: Objetivos Incompletos}
    \label{tab:tf_incompletos}
    \centering
    \footnotesize
    \begin{tabular}{|p{4cm}|p{10cm}|}
        \hline
        \rowcolor{headergray}
        \textbf{Trabalho Futuro} & \textbf{Descrição} \\
        \hline
        Execução da Unidade Instrucional & Executar a fase 4 da metodologia ENgAGED, aplicando o jogo em contexto real de ensino e realizando coleta sistemática de dados. \\
        \hline
        Avaliação da Unidade Instrucional & Executar a fase 5 da metodologia ENgAGED, avaliando a eficácia pedagógica do jogo em sala de aula. \\
        \hline
        Fase 2 e 3 & Concluir a narrativa do jogo, introduzindo o alquimista rival como chefão final e adicionando desafios pedagógicos complementares. \\
        \hline
    \end{tabular}
    \caption*{Fonte: Autor}
\end{table}


Além das etapas metodológicas pendentes, a avaliação realizada com os usuários
(descrita na seção anterior) forneceu \emph{insights} valiosos sobre aspectos
técnicos e de design que necessitam de refinamento. A \autoref{tab:tf_feedback}
sintetiza as melhorias prioritárias identificadas a partir da análise
qualitativa dos testes.

\begin{table}[H]
    \caption{Trabalhos Futuros: Melhorias baseadas no \emph{Feedback}}
    \label{tab:tf_feedback}
    \centering
    \footnotesize
    \begin{tabular}{|p{4cm}|p{10cm}|}
        \hline
        \rowcolor{headergray}
        \textbf{Melhoria} & \textbf{Justificativa} \\
        \hline
        Balanceamento da geração aleatória de elementos & Ajustar o algoritmo
        de geração de elementos para reduzir a aleatoriedade excessiva,
        evitando situações em que o jogador fica \enquote{travado} sem as
        combinações necessárias. \\
        \hline
        Reformulação do Tutorial & Criar uma introdução mais guiada e interativa, explicando com clareza as mecânicas de dano elemental e combinação, suavizando a curva de aprendizado inicial. \\
        \hline
    \end{tabular}
    \caption*{Fonte: Autor}
\end{table}


Por fim, visando expandir o ciclo de vida do jogo e seu potencial de
engajamento, foram mapeadas oportunidades de evolução que extrapolam o escopo
original. Estas ideias, apresentadas na \autoref{tab:tf_ideias}, buscam
modernizar a acessibilidade e ampliar a profundidade pedagógica da ferramenta.

\begin{table}[H]
    \caption{Trabalhos Futuros: Ideias de Expansão}
    \label{tab:tf_ideias}
    \centering
    \footnotesize
    \begin{tabular}{|p{4cm}|p{10cm}|}
        \hline
        \rowcolor{headergray}
        \textbf{Expansão} & \textbf{Motivação} \\
        \hline
        Versão Mobile & Aumentar o alcance e acessibilidade do jogo, atingindo novos públicos como estudantes que utilizam dispositivos móveis como plataforma principal. \\
        \hline
        Modo Arena & Introduzir elementos \emph{arcade} focados em repetição e treino, reforçando os conceitos de estruturas de dados por meio de desafios rápidos e pontuação competitiva. \\
        \hline
        Modo PvP & Promover competitividade e interação social, permitindo que jogadores utilizem estratégias de gerenciamento de inventário para superar oponentes em tempo real. \\
        \hline
        Novas Estruturas (Árvores/Grafos) & Tornar o jogo mais robusto pedagogicamente, abordando tópicos avançados da disciplina e ampliando o repertório de mecânicas disponíveis. \\
        \hline
    \end{tabular}
    \caption*{Fonte: Autor}
\end{table}


A implementação destes trabalhos futuros visa permitir que o jogo evolua de um
protótipo funcional para uma ferramenta educacional robusta, capaz de atender a
diferentes perfis de aprendizes e contextos de ensino. Com essas perspectivas
de continuidade estabelecidas, encerra-se este estudo, apresentando as
conclusões finais na \autoref{sec:conclusao} a seguir.


\section{Conclusão} \label{sec:conclusao}

O desenvolvimento deste trabalho permitiu explorar o potencial dos jogos sérios
como ferramenta de apoio ao ensino de Computação, abordando especificamente a
dificuldade histórica no aprendizado de Estruturas de Dados. Ao integrar
conceitos de pilhas, filas e listas encadeadas em uma narrativa fantasiosa de
alquimia e mecânicas de plataforma, o projeto buscou mitigar a abstração desses
temas, oferecendo uma alternativa lúdica ao ensino tradicional.

A aplicação da metodologia ENgAGED garantiu que o desenvolvimento não se
distanciasse dos objetivos pedagógicos, resultando em um artefato que equilibra
entretenimento e educação. A avaliação realizada com o público-alvo demonstrou que a
abordagem implícita foi bem-sucedida: os jogadores foram capazes de identificar e
operar as lógicas das estruturas de dados para resolver problemas de combate e
sobrevivência, validando a hipótese de que é possível aprender conceitos complexos
através da experimentação prática em um ambiente de jogo.

Os resultados quantitativos e qualitativos indicaram índices elevados de engajamento,
diversão e estética. A analogia criada entre a manipulação dos inventários e as
operações de memória foi percebida como um diferencial inovador. No entanto, o
estudo também evidenciou desafios técnicos, especificamente no balanceamento da
aleatoriedade (RNG) e na precisão dos controles (operabilidade), fatores que, embora
não tenham invalidado a proposta educacional, impactaram a fluidez da experiência
inicial.

\fixme{
  Conclui-se, portanto, que o jogo atingiu seu objetivo geral de apresentar conceitos
  de estruturas de dados de forma implícita e motivadora. O projeto contribui para a
  área de Informática na Educação ao fornecer um exemplo prático de como mecânicas de
  jogos de ação podem ser utilizadas para o ensino de lógica, fugindo dos tradicionais
  \emph{quizzes} e \emph{puzzles} explícitos. As bases estabelecidas neste trabalho
  oferecem um caminho sólido para futuras iterações que poderão refinar a jogabilidade
  e expandir o conteúdo pedagógico para novos níveis de complexidade.
}

