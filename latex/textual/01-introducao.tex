\chapter{Introdução}

O ensino e a aprendizagem de conceitos fundamentais da área de computação, como
estruturas de dados, constituem um desafio recorrente para educadores e
estudantes. De acordo com o autor \citeonline{mtaho2024difficulties}, a
disciplina de estruturas de dados é altamente exigente para estudantes de
ciência da computação, sendo frequentemente associada a uma elevada carga
cognitiva e, consequentemente, a altas taxas de reprovação e evasão do curso.
Entre os principais fatores que contribuem para essas dificuldades estão a
natureza abstrata dos conceitos envolvidos e a baixa motivação dos alunos. Esse
cenário se agrava pelo fato de que, tradicionalmente, o ensino desses conteúdos
ocorre por meio de aulas expositivas e exercícios de codificação, o que tende a
gerar baixa retenção do conteúdo e desinteresse por parte dos estudantes.

Além disso, as novas gerações de estudantes estão cada vez mais habituadas a um
fluxo constante de informações e experiências interativas, desenvolvendo um
comportamento que valoriza respostas rápidas e estímulos visuais. Isso torna o
ensino convencional ainda menos atrativo. \fixme{A partir de sua pesquisa, o
autor \citeonline{mtaho2024difficulties} recomenda a adoção de novas estratégias de
ensino para mitigar as dificuldades e melhorar a experiência de aprendizagem de
estrutura de dados.}

Nesse cenário, os jogos sérios surgem como uma estratégia educacional
promissora, ao promover o aprendizado ativo e engajado. Diferentemente de
abordagens instrucionais diretas, os jogos sérios podem ser projetados para que
a aprendizagem ocorra como consequência da interação do jogador com o ambiente,
desafios e regras do jogo. Essa perspectiva está alinhada à teoria do
construcionismo, proposta por \citeonline{papert1993children}, segundo a qual o
conhecimento é construído ativamente pelos alunos quando estes se envolvem com
a criação, exploração e manipulação de artefatos significativos.

Jogos sérios são definidos como uma aplicação de videogames cujo objetivo
principal é educar, treinar ou sensibilizar, sem abrir mão do entretenimento
\cite{mouaheb2012serious}. Contudo, uma crítica recorrente a essa abordagem é
que muitos desses jogos sérios falham como jogos, priorizam o conteúdo educativo de
forma explícita, relegando a experiência lúdica a segundo plano, \fixme{quando, na
verdade, ensino e entretenimento deveriam caminhar lado a lado
\cite{mouaheb2012serious}.}

De acordo com \citeonline{de2025codebo}, atualmente, grande parte dos jogos sérios se
utilizam os conceitos de programação apenas como tema, sem integrá-los
verdadeiramente às suas mecânicas. Essa limitação evidencia um modelo que tende
a transformar o jogo em um pretexto para ensinar diretamente, por meio de
\fixme{mecânicas expositivas como} questionários ou simulações superficiais.

O presente trabalho propõe uma abordagem alternativa: utilizar mecânicas de
jogo que representem, de forma implícita e interativa, conceitos fundamentais
de estruturas de dados. Em vez de apresentar diretamente listas, pilhas ou
filas, o jogo incorporará esses elementos em sua lógica e estrutura interna,
permitindo que o jogador interaja com tais conceitos de forma intuitiva e
contextualizada. Dessa maneira, o aprendizado ocorre como consequência da
resolução de problemas e da exploração do sistema, e não como resultado de
instruções explícitas ou desafios de programação.

Diferentemente de jogos educativos que simulam exercícios de codificação, o
objetivo deste trabalho é projetar um jogo no qual os conceitos ensinados
estejam presentes nas ações tomadas pelo jogador, mesmo que ele não os
reconheça explicitamente como tais. Na próxima seção, o objetivo geral deste
trabalho será aprofundado.


\section{Objetivo Geral}

Este trabalho tem como objetivo geral desenvolver um jogo sério \cite{mouaheb2012serious} que explore conceitos fundamentais de estruturas de dados de forma implícita por meio de mecânicas lúdicas e interativas, promovendo um processo de aprendizagem mais significativo, intuitivo e motivador.

\section{Objetivos Específicos}

Com base no objetivo geral, este trabalho também visa alcançar os seguintes objetivos específicos:

\fixme{REFAZER: Grande parte destes é a metodologia cientifica}

\begin{itemize}
  \item Investigar modelos de jogos sérios e sua aplicação no ensino de conteúdos relacionados à computação;
  \item Projetar e implementar um jogo educacional fundamentado na metodologia ENgAGED;
  \item Incorporar, nas mecânicas do jogo, representações implícitas de estruturas de dados, como listas, filas e pilhas, além de algoritmos de busca e ordenação;
  \item Avaliar a usabilidade e a eficácia do jogo no processo de ensino e aprendizagem;
  \item Coletar e analisar o \emph{feedback} dos usuários com o intuito de orientar futuras melhorias da ferramenta desenvolvida.
\end{itemize}

\section{Justificativa}

A proposta deste trabalho encontra respaldo na \fixme{necessidade} de tornar o ensino de estruturas de dados mais dinâmico e alinhado às expectativas das novas gerações de aprendizes. O uso de jogos sérios como recurso educacional permite contextualizar os conceitos dentro de uma narrativa ou desafio envolvente, aumentando o engajamento e favorecendo a construção do conhecimento de forma mais prática e intuitiva. \cite{mouaheb2012serious}

Além disso, a adoção da metodologia \emph{GAMED} no processo de desenvolvimento garante uma abordagem sistemática e centrada no aprendizado, permitindo que os objetivos educacionais sejam alcançados sem comprometer a experiência lúdica.

Dessa forma, este trabalho justifica-se por buscar uma alternativa para o ensino de estruturas de dados, pretendendo contribuir para a formação de profissionais mais preparados, criativos e capazes de aplicar o conhecimento de maneira prática e estratégica no mercado de trabalho.


%  TODO:
% - Adicionar uma gráfico identificando trabalhos a respeito de jogos serios | jogos serios relacionados a programação | jogos serios relacionados a estrutura de dados
% - diferença entre jogo simulado, jogo educacional e jogo sério
% - Adicionar uma gráfico identificando o crescimento de trabalhos relacionados a jogos / jogos serios nos ultimos anos
% - Adicionar uma gráfico identificando de quantas pessoas que estudam na área jogam jogos digitais, gráfico de quantas pessoas jogam jogos educacionais por conta própria
% - Justificar o use de ensino implicito para desacoplar o conceito ensinado de uma linguagem de programação
% - jogos digitais são uma forma eficiente para ensinar adolescentes
% - Citar evasão escolar elevada nos cursos de tecnologias, motivada pela:
%   * Dificuldade de alunos em abstrair conceitos abstratos.
%   * Maioria dos alunos possuem trabalho em tempo integral.
% - os jogos atualmente estão limitados a utilizar os conceitos da programação como seu tema. \cite{de2025codebo}
% - adicionar gráfico - tempo/evasão
% - adicionar gráfico - tempo/porcentagem de aprovação em estrutura de dados
% - Adiocionar justficativa e objetivos geral(um paragrafo / frase) e expeficio (tópicos)
% - justificar nas considerações finais os objetivos (se foram alcançados ou não)
