\subsection{Comparação dos artigos}

Com base na análise dos cinco trabalhos selecionados, observa-se uma variedade
de abordagens no uso de jogos sérios para o ensino de conceitos de programação.
Cada proposta apresenta escolhas distintas quanto aos conceitos abordados,
estilo visual, mecânicas de interação e a forma como os conteúdos educacionais
são apresentados ao jogador.

A \autoref{tab:cmp_trabalhos_relatos} a seguir apresenta uma síntese
comparativa dos trabalhos analisados, destacando os aspectos centrais de cada
proposta e os elementos recorrentes identificados entre eles. Os critérios
utilizados para a comparação incluem:

\begin{itemize}
  \item \textbf{Jogo Digital (JD)}: Indica se o jogo é digital ou físico.
  \item \textbf{Conceitos de Estrutura de Dados (CED)}: Quais conceitos foram abordados: Pilha (P), Fila (F), Lista (L), Lista Encadeada (LE), Lista Duplamente Encadeada (LDE), Árvore Binária (AB), Algoritmo de Busca (B) e Algoritmo de Ordenação (O).
  \item \textbf{Forma de Abordagem Educacional (Ensino)}: Define se o ensino é tratado de forma explícita ou implícita.
  \item \textbf{Estilo}: Representação visual do jogo.
  \item \textbf{Gênero}: Categoria do jogo.
\end{itemize}

\begin{table}[H]
	\caption{Comparação entre os trabalhos relacionados}
	\label{tab:cmp_trabalhos_relatos}
	\centering
	\footnotesize
	\begin{tabular}{|r|cllll|}
		\hline
    \rowcolor{headergray}
		\textbf{Trabalho}      & \textbf{JD}  & \textbf{CED} & \textbf{Ensino}    & \textbf{Estilo}      & \textbf{Gênero}   \\
		\hline
		CodingJob              & Sim          & -            & Explícito          & Simulador            & \emph{Puzzle}     \\
		CodeBô                 & Sim          & F,L,P,B      & Implícito          & Isométrico           & \emph{Puzzle}     \\
		CodeBo Unplugged       & Não          & P            & Implícito          & Tabuleiro            & \emph{Puzzle}     \\
		AuxED                  & Sim          & O            & Explícito          & P\&C                 & \emph{Puzzle}     \\
		Prog-poly              & Não          & -            & Explícito          & Tabuleiro            & \emph{Quiz}       \\
		\hline
	\end{tabular}
	\caption*{Fonte: Autor}
\end{table}

