\section{Conclusão}

Com base na análise dos trabalhos e jogos relacionados, observa-se que, embora existam diversas iniciativas que utilizam jogos digitais para o ensino de estruturas de dados, a maioria adota abordagens mais explícitas e centradas em gêneros como puzzle.
Além disso, os conceitos de estruturas de dados costumam ser apresentados de forma segmentada, focando em apenas um ou dois tipos, como pilhas ou filas.
O trabalho proposto se diferencia ao incorporar múltiplas estruturas de dados — incluindo pilha, fila e lista encadeada — dentro de um contexto lúdico e narrativo, com uma abordagem de ensino implícita, que estimula o aprendizado por meio da exploração e resolução de desafios integrados à mecânica do jogo.
O estilo plataformer e o gênero aventura também representam uma inovação em relação aos estilos predominantes encontrados, ampliando o potencial de engajamento e imersão dos jogadores.

