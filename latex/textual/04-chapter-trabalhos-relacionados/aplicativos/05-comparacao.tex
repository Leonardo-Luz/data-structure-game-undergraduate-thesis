\subsection{Comparação}

A análise dos jogos selecionados revela uma variedade de abordagens no uso de jogos sérios voltados ao ensino de lógica de programação. No entanto, observa-se uma presença ainda limitada de propostas que exploram conceitos de estruturas de dados de maneira mais aprofundada. A \autoref{tab:cmp_jogos_relatos} apresenta uma síntese comparativa dessas iniciativas, destacando os principais elementos de cada proposta.

Os critérios utilizados para a comparação foram:

\begin{itemize}
  \item \textbf{Conceitos de Estrutura de Dados (CED)} - Quais conceitos foram abordados:
  \begin{itemize}
    \item Pilha (P)
    \item Fila (F)
    \item Lista (L)
  \end{itemize}
  \item \textbf{Forma de Abordagem Educacional (Ensino)} - Define se o ensino é tratado de forma explícita ou implícita.
  \item \textbf{Estilo} - Representação visual do jogo.
  \item \textbf{Gênero} - Categoria do jogo
  \item \textbf{Gratuito} - Indica se o jogo está disponível gratuitamente.
  \item \textbf{Avaliação} - Nota atribuída com base no sistema de avaliação da plataforma onde o jogo é disponibilizado.
\end{itemize}

Nos casos em que o jogo está disponível em plataformas que não possuem sistema de avaliação integrado (como o itch.io), ou em que não há avaliações registradas, a avaliação é considerada \textit{indefinida}. Para os jogos disponíveis na Steam, foi utilizada uma conversão aproximada baseada na porcentagem de avaliações positivas, conforme mostrado na \autoref{tab:avalicao_steam}.

\begin{table}[H]
	\caption{Comparação entre jogos relacionados e o trabalho proposto}
	\label{tab:cmp_jogos_relatos}
	\centering
	\footnotesize
	\begin{tabular}{r|clllll}
		\toprule
		\textbf{Trabalho}      & \textbf{ED}         & \textbf{Ensino}    & \textbf{Estilo}      & \textbf{Gênero}   & \textbf{Gratuito} & \textbf{Avaliação}  \\
		\midrule
		Human R.M.             & -                   & Explícito          & Top Down             & Puzzle            & Não               & 4.5                 \\
		AlgoBot                & -                   & Implícito          & Top Down             & Puzzle            & Não               & 4.2                 \\
		MOP'N SPARK            & -                   & Implícito          & Plataformer          & Puzzle            & Indefinido        & Indefinido          \\
		Iron Ears              & P,F,L               & Implícito          & Drag \& Drop         & Puzzle            & Sim               & Indefinido          \\
		\rowcolor{gray!20}
		\textbf{Este trabalho} & \textbf{P,F,LE,B,O} & \textbf{Implícito} & \textbf{Plataformer} & \textbf{Aventura} & \textbf{Sim}      & \textbf{Indefinido} \\
		\bottomrule
	\end{tabular}

	\vspace{1.25em}
	\begin{minipage}{0.8\linewidth}
		\footnotesize
		\textbf{CED:} Conceitos de Estrutura de Dados utilizados -
		\textbf{P:} Pilha, \textbf{F:} Fila, \textbf{L:} Lista, \textbf{LE:} Lista Encadeada, \textbf{B:} Algoritmo de Busca, \textbf{O:} Algoritmo de Ordenação. \\
		\textbf{Ensino:} Forma de abordagem educacional (Explícito ou Implícito). \\
		\textbf{Estilo:} Estilo de interação do jogo. - \textbf{P\&C:} Point and Click \\
		\textbf{Gênero:} Categoria do jogo. \\
		\textbf{Gratuito:} Se o jogo é gratuito ou não. \\
		\textbf{Avaliação:} Avaliação do jogo, \textbf{indefinido} ocorre quando a plataform que disponibiliza o jogo não possui um sistema de avaliação, como o itch.io, ou nenhuma avaliação foi feita ao jogo. Jogos da steam possuem um sistema de avaliação próprio e por conta disto foi feito uma conversão aproximada apresentada na \autoref{tab:avalicao_steam}
	\end{minipage}
\end{table}


\begin{table}[H]
	\caption{Conversão do Sistema de Avaliação da Steam para um sistema numeral de 1 a 5}
	\label{tab:avalicao_steam}
	\centering
	\footnotesize
	\begin{tabular}{clc}
		\toprule
		\textbf{Porcentagem de Avaliações Positivas} & \textbf{Avaliação Steam} & \textbf{Conversão Aproximada} \\
		\midrule
		90\% - 100\%                                 & Extremamente positivas   & 4.5 - 5.0                     \\
		70\% - 89\%                                  & Muito positivas          & 3.5 - 4.4                     \\
		40\% - 69\%                                  & Neutras                  & 2.0 - 3.4                     \\
		10\% - 39\%                                  & Muito Negativas          & 0.5 - 1.9                     \\
		0\% -  9\%                                   & Extremamente negativas   & 0.0 - 0.4                     \\
		\bottomrule
	\end{tabular}
	\caption*{Fonte: Autor}
\end{table}


O cálculo utilizado para a conversão é dado pela fórmula:

\[
	\text{Porcentagem de Avaliações Positivas} \times \frac{5}{100}
\]
