\subsection{Comparação}

A análise dos jogos selecionados revela uma variedade de abordagens no uso de jogos sérios voltados ao ensino de lógica de programação. No entanto, observa-se uma presença ainda limitada de propostas que exploram conceitos de estruturas de dados de maneira mais aprofundada. A \autoref{tab:cmp_jogos_relatos} apresenta uma síntese comparativa dessas iniciativas, destacando os principais elementos de cada proposta.

Os critérios utilizados para a comparação foram:

\begin{itemize}
  \item \textbf{Conceitos de Estrutura de Dados (CED)} - Quais conceitos foram abordados:
  \begin{itemize}
    \item Pilha (P)
    \item Fila (F)
    \item Lista (L)
  \end{itemize}
  \item \textbf{Forma de Abordagem Educacional (Ensino)} - Define se o ensino é tratado de forma explícita ou implícita.
  \item \textbf{Estilo} - Representação visual do jogo.
  \item \textbf{Gênero} - Categoria do jogo
  \item \textbf{Gratuito} - Indica se o jogo está disponível gratuitamente.
  \item \textbf{Avaliação} - Nota atribuída com base no sistema de avaliação da plataforma onde o jogo é disponibilizado.
\end{itemize}

Nos casos em que o jogo está disponível em plataformas que não possuem sistema de avaliação integrado (como o itch.io), ou em que não há avaliações registradas, a avaliação é considerada \textit{indefinida}. Para os jogos disponíveis na Steam, foi utilizada uma conversão aproximada baseada na porcentagem de avaliações positivas, conforme mostrado na \autoref{tab:avalicao_steam}.

\begin{table}[H]
	\caption{Comparação entre os jogos relacionados}
	\label{tab:cmp_jogos_relatos}
	\centering
	\footnotesize
	\begin{tabular}{|r|lllllc|}
		\hline
    \rowcolor{headergray}
		\textbf{Trabalho} & \textbf{CED} & \textbf{Ensino}   & \textbf{Estilo}      & \textbf{Gênero}   & \textbf{Gratuito} & \textbf{Avaliação}  \\
		\hline
		Human R.M.        & -            & Explícito         & \emph{Top Down}       & \emph{Puzzle}     & Não               & 4.5 \\
    AlgoBot           & -            & Implícito         & \emph{Top Down}       & \emph{Puzzle}     & Não               & 4.2 \\
		MOP'N SPARK       & -            & Implícito         & Plataformer           & \emph{Puzzle}     & Indefinido        & Indefinido \\
    Iron Ears         & P,F,L        & Implícito         & \emph{Drag \& Drop}   & \emph{Puzzle}     & Sim               & Indefinido \\
		\hline
	\end{tabular}
	\caption*{Fonte: Autor}
\end{table}


\begin{table}[H]
	\caption{Conversão do Sistema de Avaliação da Steam para um sistema numeral de 1 a 5}
	\label{tab:avalicao_steam}
	\centering
	\footnotesize
	\begin{tabular}{clc}
		\toprule
		\textbf{Porcentagem de Avaliações Positivas} & \textbf{Avaliação Steam} & \textbf{Conversão Aproximada} \\
		\midrule
		90\% - 100\%                                 & Extremamente positivas   & 4.5 - 5.0                     \\
		70\% - 89\%                                  & Muito positivas          & 3.5 - 4.4                     \\
		40\% - 69\%                                  & Neutras                  & 2.0 - 3.4                     \\
		10\% - 39\%                                  & Muito Negativas          & 0.5 - 1.9                     \\
		0\% -  9\%                                   & Extremamente negativas   & 0.0 - 0.4                     \\
		\bottomrule
	\end{tabular}
\end{table}


O cálculo utilizado para a conversão é dado pela fórmula:

\[
	\text{Porcentagem de Avaliações Positivas} \times \frac{5}{100}
\]
