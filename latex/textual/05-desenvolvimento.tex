\chapter{Desenvolvimento}

Este capítulo descreve o processo de desenvolvimento do jogo educacional seguindo a metodologia ENgAGED. O desenvolvimento foi estruturado em fases práticas, com foco no que foi efetivamente implementado durante este trabalho.

\section{Arquitetura e Estrutura do Projeto}

O projeto foi desenvolvido utilizando o motor de jogo \textit{Unity 2022 LTS} com linguagem de programação \textit{C\#}, garantindo compatibilidade e suporte a longo prazo. A estrutura do projeto segue padrões de organização profissionais, separando assets visuais, scripts, cenas, prefabs e recursos.

\subsection{Estruturas de Dados Implementadas}

As três estruturas de dados fundamentais foram implementadas como classes independentes em \textit{C\#}, refletindo com precisão os conceitos teóricos:

\subsubsection*{Pilha (Stack)}

A implementação da pilha segue o padrão LIFO (Last In, First Out). As operações principais implementadas são:

\begin{itemize}
  \item \texttt{Push(element)}: insere um elemento no topo da pilha;
  \item \texttt{Pop()}: remove e retorna o elemento do topo;
  \item \texttt{Peek()}: retorna o elemento do topo sem remover;
  \item \texttt{IsEmpty()}: verifica se a pilha está vazia.
\end{itemize}

Esta estrutura representa o inventário da pilha no jogo, onde elementos só podem ser removidos do topo, refletindo a ordem LIFO durante o combate.

\subsubsection*{Fila (Queue)}

A implementação da fila segue o padrão FIFO (First In, First Out). As operações principais são:

\begin{itemize}
  \item \texttt{Enqueue(element)}: insere um elemento no final da fila;
  \item \texttt{Dequeue()}: remove e retorna o primeiro elemento;
  \item \texttt{Peek()}: retorna o primeiro elemento sem remover;
  \item \texttt{IsEmpty()}: verifica se a fila está vazia.
\end{itemize}

Este inventário representa a fila no jogo, onde elementos devem ser removidos na ordem em que foram adicionados, essencial para desafios que exigem sequência ordenada.

\subsubsection*{Lista (Linked List)}

A implementação da lista encadeada permite acesso livre em qualquer posição. As operações principais são:

\begin{itemize}
  \item \texttt{Add(element)}: adiciona um elemento ao final da lista;
  \item \texttt{Insert(index, element)}: insere um elemento em posição específica;
  \item \texttt{RemoveAt(index)}: remove o elemento na posição especificada;
  \item \texttt{Get(index)}: retorna o elemento na posição especificada;
  \item \texttt{Size()}: retorna o tamanho da lista.
\end{itemize}

Este inventário oferece máxima flexibilidade, permitindo ao jogador acessar qualquer elemento independentemente de sua posição.

\section{Sistemas Principais do Jogo}

\subsection{Sistema de Movimento}

O sistema de movimento implementa controles responsivos para o protagonista \textit{plague doctor}. Através de entrada de teclado, o jogador pode:

\begin{itemize}
  \item Movimentar-se horizontalmente (esquerda e direita);
  \item Pular com física realística de plataforma;
  \item Interagir com elementos do ambiente.
\end{itemize}

O sistema inclui detecção de colisão com plataformas, paredes e inimigos, garantindo movimento suave e previsível.

\subsection{Sistema de Combate}

O sistema de combate é central à mecânica educacional do jogo. Funciona através de:

\begin{enumerate}
  \item \textbf{Seleção de Elementos}: o jogador remove elementos de seus inventários respeitando as regras de cada estrutura;
  
  \item \textbf{Combinação}: elementos iguais são combinados para formar um ataque (ex.: dois elementos fogo geram ataque de fogo);
  
  \item \textbf{Validação}: o sistema verifica se o elemento do ataque corresponde à fraqueza do inimigo;
  
  \item \textbf{Aplicação de Dano}: ataques corretos causam dano ao inimigo; ataques incorretos causam dano ao jogador.
\end{enumerate}

Este sistema reforça implicitamente o aprendizado das estruturas de dados, pois o sucesso depende diretamente de operações corretas.

\subsection{Sistema de Inventário}

Cada um dos três inventários é gerenciado independentemente, refletindo as características de sua estrutura:

\begin{itemize}
  \item \textbf{Inventário de Pilha}: display visual mostra elementos em ordem LIFO, com remoção apenas do topo;
  
  \item \textbf{Inventário de Fila}: display visual mostra elementos em ordem FIFO, com remoção apenas do primeiro;
  
  \item \textbf{Inventário de Lista}: display visual permite visualizar todos os elementos com acesso livre a qualquer posição.
\end{itemize}

O sistema inclui geração contínua de elementos alquímicos (fogo, água, ar, terra) durante o jogo, preenchendo os inventários.

\subsection{Sistema de Vida e Penalidades}

O jogador inicia cada fase com quantidade determinada de vidas. Penalidades são aplicadas em casos de:

\begin{itemize}
  \item Combinação incorreta de elementos (culpa didática);
  \item Contato com inimigos ou obstáculos;
  \item Queda fora dos limites do mapa.
\end{itemize}

Quando a vida atinge zero, o jogador retorna ao último checkpoint (ponto de controle), sem necessidade de reiniciar toda a fase.

\subsection{Sistema de Progresso e Salvamento}

O sistema gerencia:

\begin{itemize}
  \item \textbf{Checkpoints}: posições de salvamento automático distribuídas estrategicamente;
  
  \item \textbf{Salvamento de Estado}: persistência de inventários, vida, fases desbloqueadas;
  
  \item \textbf{Métricas de Desempenho}: registro de tempo de conclusão, tentativas, acertos e erros de combinação.
\end{itemize}

Dados são salvos em arquivo local, permitindo continuação de partidas.

\subsection{Sistema de Feedback}

O jogo fornece feedback contínuo através de múltiplos canais:

\subsubsection*{Feedback Visual}

\begin{itemize}
  \item Animações do personagem indicando sucesso ou erro;
  \item Mudanças de cor (avermelhamento ao tomar dano);
  \item Efeitos particulares em ataques bem-sucedidos;
  \item Display visual dos elementos nos inventários.
\end{itemize}

\subsubsection*{Feedback Sonoro}

\begin{itemize}
  \item Sons de sucesso em combinações corretas;
  \item Sons de erro em combinações incorretas;
  \item Trilha sonora adaptativa ao contexto.
\end{itemize}

\subsubsection*{Feedback Textual}

\begin{itemize}
  \item Mensagens contextualizadas explicando erros;
  \item Indicadores de status (falta de mana, vida baixa);
  \item Descrições de inimigos e suas fraquezas.
\end{itemize}

\section{Interface de Usuário}

\subsection{Layout Principal do Jogo}

A interface foi projetada com princípios de minimalismo, evitando sobrecarga visual. O layout principal inclui:

\begin{itemize}
  \item \textbf{Área de Jogo}: centro da tela, ocupando a maior parte do espaço;
  
  \item \textbf{Inventários}: localizado na parte inferior direita, mostrando os três inventários lado a lado;
  
  \item \textbf{Barra de Vida}: indica a vida atual do jogador de forma clara;
  
  \item \textbf{Indicador de Inimigos}: mostra quantos inimigos restam na arena.
\end{itemize}

\subsection{Menus Implementados}

\subsubsection*{Menu Principal}

Permite ao jogador:
\begin{itemize}
  \item Iniciar novo jogo;
  \item Carregar jogo salvo;
  \item Acessar configurações;
  \item Sair do jogo.
\end{itemize}

\subsubsection*{Menu de Pausa}

Disponível durante o jogo, permitindo:
\begin{itemize}
  \item Retomar jogo;
  \item Acessar configurações;
  \item Voltar ao menu principal.
\end{itemize}

\subsubsection*{Telas de Livros}

Livros informativos integrados, incluindo:
\begin{itemize}
  \item \textbf{Livro de Tutorial}: explica mecânicas do jogo;
  \item \textbf{Livro de Inventários}: detalha cada estrutura de dados;
  \item \textbf{Livro de Combate}: documenta tipos de ataques e estratégias;
  \item \textbf{Livro de Consumíveis}: lista itens especiais disponíveis;
  \item \textbf{Bestiário}: informações sobre inimigos e fraquezas.
\end{itemize}

\section{Implementação de Fases}

Cada fase foi implementada para enfatizar uma estrutura de dados específica:

\subsection{Fase 1 - Pilha}

Inimigos simples com uma única vulnerabilidade. Mecânica focada em:
\begin{itemize}
  \item Compreensão do conceito LIFO;
  \item Remoção de elementos apenas do topo;
  \item Construção de confiança básica.
\end{itemize}

\subsection{Fase 2 - Fila}

Inimigos com padrões de movimento previsíveis. Mecânica focada em:
\begin{itemize}
  \item Compreensão do conceito FIFO;
  \item Sequência ordenada obrigatória;
  \item Planejamento de ataques.
\end{itemize}

\subsection{Fase 3 - Lista}

Inimigos mais desafiadores com múltiplas fraquezas. Mecânica focada em:
\begin{itemize}
  \item Flexibilidade de acesso à estrutura;
  \item Seleção estratégica de elementos;
  \item Adaptabilidade em combate.
\end{itemize}

\subsection{Fase 4 - Consumíveis e Ordenação}

Introduz itens especiais que modificam estruturas. Inclui:
\begin{itemize}
  \item Item de Ordenação: reorganiza a estrutura de dados;
  \item Item de Inserção: adiciona elementos específicos;
  \item Item de Remoção: remove elementos específicos;
  \item Conceitos de algoritmos de ordenação aplicados implicitamente.
\end{itemize}

\subsection{Fase 5 - Integração}

Desafios finais combinando todas as estruturas. Requer:
\begin{itemize}
  \item Domínio de todas as estruturas;
  \item Decisão de qual estrutura usar para cada ataque;
  \item Coordenação complexa de estratégias.
\end{itemize}

\section{Implementação de Inimigos}

\subsection{Tipos de Inimigos}

Diversos tipos foram implementados, cada um ensinando conceitos específicos:

\begin{itemize}
  \item \textbf{Inimigo Melee}: aproxima-se do jogador para atacar;
  
  \item \textbf{Inimigo Ranged}: atira projéteis à distância;
  
  \item \textbf{Inimigo com Múltiplas Fraquezas}: requer combinações mais complexas;
  
  \item \textbf{Inimigo Blindado}: requer ataques mais poderosos (combinações longas).
\end{itemize}

\subsection{Sistema de Vulnerabilidades}

Cada inimigo possui vulnerabilidades definidas (fogo, água, ar, ou terra). Ataques com elemento correto causam dano multiplicado; ataques com elemento incorreto causam dano ao jogador, reforçando a necessidade de planejamento.

\subsection{Estatísticas Randomizadas}

Para aumentar rejogabilidade, inimigos podem ter variações aleatórias em:
\begin{itemize}
  \item Vida total;
  \item Velocidade de movimento;
  \item Cadência de ataque;
  \item Dano causado.
\end{itemize}

\section{Implementação de Consumíveis}

Seis tipos de consumíveis foram implementados:

\subsection{Ordenação}

Reorganiza a estrutura de dados ativa, servindo como auxílio estratégico quando elementos desejáveis estão em posições inacessíveis.

\subsection{Insert}

Adiciona dois elementos iguais ao inventário especificado, útil quando o jogador precisa de elementos específicos rapidamente.

\subsection{Remove}

Remove um elemento específico de um inventário, liberando espaço ou removendo obstáculos.

\subsection{Mana}

Concede mana infinita por tempo determinado, permitindo uso ilimitado de ataques durante o período.

\subsection{Cura}

Restaura 1 ponto de vida, fornecendo segunda chance quando vida está crítica.

\subsection{Vida Extra}

Aumenta a quantidade de tentativas em 1, estendendo a capacidade do jogador em uma fase.

\section{Desenvolvimentos Técnicos Realizados}

\subsection{Persistência de Dados}

Sistema de salvamento implementado utilizando serialização JSON, permitindo:
\begin{itemize}
  \item Salvamento automático em checkpoints;
  \item Carregamento de partidas anterior;
  \item Registro de métricas de desempenho.
\end{itemize}

\subsection{Gerenciamento de Câmera}

Camera com pixel-perfect para manter qualidade artística do pixel art:
\begin{itemize}
  \item Seguimento suave do personagem;
  \item Parallax visual com backgrounds;
  \item Limites de câmera para não revelar áreas não prontas.
\end{itemize}

\subsection{Sistema de Diálogos}

Sistema robusto de diálogos implementado, permitindo:
\begin{itemize}
  \item Narrativa contextualizada;
  \item Interações com NPCs;
  \item Feedback narrativo sobre progresso.
\end{itemize}

\subsection{Animações e Efeitos Visuais}

Implementação completa de:
\begin{itemize}
  \item Animações de movimento do personagem;
  \item Animações de ataque e dano;
  \item Efeitos particulares de elementos alquímicos;
  \item Transições e fade-outs.
\end{itemize}

\section{Desafios Enfrentados e Soluções}

\subsection{Balanceamento de Dificuldade}

\textbf{Desafio}: Manter dificuldade progressiva sem frustrar o jogador.

\textbf{Solução}: Testes iterativos com diferentes públicos, ajustando vida de inimigos, velocidade e frequência de ataques baseado em feedback.

\subsection{Clareza das Mecânicas}

\textbf{Desafio}: Comunicar regras de estruturas de dados implicitamente.

\textbf{Solução}: Feedback visual claro e progressão cuidadosa, introduzindo um conceito por fase.

\subsection{Performance}

\textbf{Desafio}: Manter 60 FPS em máquinas variadas.

\textbf{Solução}: Otimização de sprites, redução de efeitos particulares em plataformas menos potentes, uso de object pooling para inimigos e projéteis.

\subsection{Tratamento de Estados}

\textbf{Desafio}: Gerenciar múltiplos estados do jogador (movimento, combate, diálogo, pausa).

\textbf{Solução}: Implementação de state machine para transições limpas entre estados.

\section{Testes Realizados}

\subsection{Testes Funcionais}

Validaram a corretude de todas as funcionalidades:
\begin{itemize}
  \item Operações de estruturas funcionam conforme especificado;
  \item Lógica de combate resolve corretamente;
  \item Salvamento e carregamento preservam estado;
  \item Interface responde adequadamente.
\end{itemize}

\subsection{Testes de Usabilidade}

Realizados com público diverso:
\begin{itemize}
  \item Estudantes do curso de ADS;
  \item Entusiastas de jogos;
  \item Pessoas sem experiência em programação.
\end{itemize}

Feedback indicou:
\begin{itemize}
  \item Interface é intuitiva;
  \item Mecânicas são compreensíveis;
  \item Narrativa engaja os jogadores;
  \item Dificuldade é bem equilibrada.
\end{itemize}

\subsection{Testes Educacionais}

Avaliaram eficácia pedagógica:
\begin{itemize}
  \item Compreensão implícita dos conceitos foi validada;
  \item Participantes conseguiram explicar comportamentos das estruturas;
  \item Interesse em estruturas de dados aumentou;
  \item Reconhecem aplicações práticas dos conceitos.
\end{itemize}

\section{Conformidade com Requisitos}

O desenvolvimento seguiu rigorosamente os requisitos funcionais e não-funcionais definidos na metodologia, detalhados nos \autoref{apd:rf} e \autoref{apd:nrf} (Apêndices A e B).

Todos os requisitos de alta prioridade foram implementados, grande maioria dos requisitos de média prioridade foi atendida, e alguns requisitos de baixa prioridade ficaram para futuras versões.

\section{Resultados do Desenvolvimento}

O desenvolvimento resultou em um jogo funcional, educacionalmente efetivo e visualmente coerente que:

\begin{itemize}
  \item Ensina implicitamente conceitos de estruturas de dados através de mecânicas engajantes;
  
  \item Mantém coerência narrativa com temática de alquimia e plague doctor;
  
  \item Oferece experiência visual coerente em pixel art;
  
  \item Fornece feedback imediato e contextualizado para aprendizado;
  
  \item Permite progressão clara em dificuldade através das fases;
  
  \item Suporta reusabilidade de código em futuras extensões.
\end{itemize}

O jogo está pronto para integração em ambientes educacionais e validação de eficácia pedagógica junto a maior população de alunos.
