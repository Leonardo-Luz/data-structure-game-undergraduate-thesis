\section{Construcionismo}

A fundamentação pedagógica deste trabalho está ancorada na teoria do
Construcionismo de Seymour Papert. Derivada do construtivismo de Piaget, a
teoria de Papert postula que a aprendizagem é mais eficaz quando o aprendiz está
conscientemente engajado na construção de um artefato público e tangível, seja
ele um castelo de areia, um poema, uma máquina ou um programa de computador
\cite{papert1993children}. Essa perspectiva é crucial para o desenvolvimento de
ambientes de aprendizagem que promovam a autonomia e a descoberta.

O Construcionismo defende que o conhecimento não é algo a ser simplesmente
transmitido, mas algo a ser construído e reconstruído pelo indivíduo por meio
de ações e interações com o mundo. Nesse contexto, o jogo proposto neste
trabalho pode ser visto como um "micromundo" de aprendizagem, um ambiente onde
os jogadores constroem seu entendimento sobre estruturas de dados não por meio
de instrução direta, mas ao manipular os sistemas do jogo para resolver
problemas. As soluções que o jogador cria dentro do jogo são os artefatos que
refletem e solidificam seu aprendizado. Essa abordagem se alinha perfeitamente
à proposta de um aprendizado implícito, onde o conhecimento é uma consequência
direta da experiência e da ação, e não o seu pré-requisito, conforme a
metodologia de ensino que busca desacoplar o conceito ensinado de uma linguagem
de programação específica, focando na compreensão conceitual através da
interação lúdica.
