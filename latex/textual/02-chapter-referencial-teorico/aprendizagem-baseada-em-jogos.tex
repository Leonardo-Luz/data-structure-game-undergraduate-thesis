\section{Aprendizagem Baseada em Jogos}

A Aprendizagem Baseada em Jogos, do inglês \emph{Game-Based Learning} (GBL), é
uma estratégia pedagógica que emprega jogos completos como ferramentas para
alcançar objetivos de aprendizagem específicos. Diferencia-se da gamificação,
que apenas aplica elementos de jogos (como pontos, medalhas e \emph{rankings}) a
contextos não lúdicos. Na GBL, o aprendizado emerge da própria interação do
jogador com as mecânicas, sistemas e narrativas do jogo
\cite{malone2021making}.

O potencial da GBL reside na sua capacidade de criar um ciclo de aprendizado
motivado intrinsecamente. Um jogo bem projetado desafia o jogador, oferece
\emph{feedback} constante e permite a experimentação em um ambiente seguro,
onde o erro é parte do processo de descoberta. Em vez de receber informações
de forma passiva, o jogador aprende ativamente ao testar hipóteses, resolver
problemas e superar obstáculos. Essa abordagem é particularmente relevante para
conceitos abstratos como os de estruturas de dados, pois permite que os alunos
visualizem e manipulem representações concretas desses conceitos dentro do
universo do jogo, promovendo um engajamento que o ensino tradicional muitas
vezes não consegue \cite{mtaho2024difficulties}.
