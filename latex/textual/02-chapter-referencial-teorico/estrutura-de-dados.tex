\section{Estruturas de Dados}

Estruturas de dados como listas, pilhas, filas, árvores e grafos fornecem
mecanismos distintos para organizar e acessar informações, cada uma com
vantagens e limitações específicas em termos de complexidade de inserção,
remoção e busca \cite{cormen2022introduction}. A compreensão dessas diferenças
é essencial para o desenvolvimento de algoritmos eficientes e para a resolução
de problemas computacionais complexos. Entretanto, a abstração desses conceitos
representa uma barreira significativa para estudantes, tornando necessário
explorar métodos pedagógicos mais interativos e motivadores.

Conforme destacado por \citeonline{mtaho2024difficulties}, a disciplina de
estruturas de dados é frequentemente associada a uma elevada carga cognitiva e
a altas taxas de reprovação e evasão em cursos de computação. As dificuldades
advêm, em grande parte, da natureza abstrata dos conceitos envolvidos e da
baixa motivação dos alunos conforme indicado na
\autoref{tab:dif_data_structure}. O modelo de ensino tradicional, que se baseia
em aulas expositivas e exercícios de codificação, muitas vezes não consegue
proporcionar a retenção efetiva do conteúdo nem despertar o interesse
necessário para a compreensão aprofundada \cite{chilwant2012comparison}. Essa
lacuna pedagógica ressalta a urgência de explorar e implementar estratégias de
ensino inovadoras que possam mitigar essas barreiras e enriquecer a experiência
de aprendizagem.
