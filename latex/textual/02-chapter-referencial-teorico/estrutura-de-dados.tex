\section{Estruturas de Dados}

As estruturas de dados representam um pilar fundamental na ciência da
computação, sendo essenciais para a organização, gestão e armazenamento
eficiente de informações, o que, por sua vez, viabiliza o desenvolvimento de
algoritmos otimizados e a construção de \emph{software} robusto e escalável
\cite{cormen2022introduction}. Contudo, a aprendizagem desses conceitos
constitui um desafio significativo para estudantes da área.

Conforme destacado por \citeonline{mtaho2024difficulties}, a disciplina de
estruturas de dados é frequentemente associada a uma elevada carga cognitiva e
a altas taxas de reprovação e evasão em cursos de computação. As dificuldades
advêm, em grande parte, da natureza abstrata dos conceitos envolvidos e da
baixa motivação dos alunos. O modelo de ensino tradicional, que se baseia em
aulas expositivas e exercícios de codificação, muitas vezes não consegue
proporcionar a retenção efetiva do conteúdo nem despertar o interesse necessário
para a compreensão aprofundada. Essa lacuna pedagógica ressalta a urgência de
explorar e implementar estratégias de ensino inovadoras que possam mitigar
essas barreiras e enriquecer a experiência de aprendizagem.
