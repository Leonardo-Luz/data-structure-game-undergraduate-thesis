\section{Unity}

Para a concretização da proposta pedagógica deste trabalho, que visa o ensino
implícito de estruturas de dados por meio de um jogo sério, a \emph{game engine}
Unity foi selecionada como a principal ferramenta de desenvolvimento. A Unity é
um ambiente de desenvolvimento multiplataforma amplamente reconhecido na
indústria de jogos, simulações e aplicações interativas, destacando-se por sua
flexibilidade e robustez.

A escolha da Unity é fundamentada em suas características que se alinham
diretamente aos objetivos do projeto. Primeiramente, sua arquitetura baseada em
componentes e a utilização da linguagem C\# para \emph{scripting} permitem a
criação de mecânicas de jogo sofisticadas e a representação abstrata de
estruturas de dados de forma eficiente. Isso é crucial para a implementação de
um aprendizado implícito, onde os conceitos são integrados à lógica do jogo sem
serem explicitamente ensinados. Em segundo lugar, a capacidade multiplataforma
da Unity garante que o jogo possa ser acessado por um público mais amplo de
estudantes, independentemente do sistema operacional. Por fim, a vasta
comunidade de desenvolvedores e a rica documentação disponível para a Unity
aceleram o processo de desenvolvimento, permitindo que a equipe se concentre
mais no design da experiência de aprendizagem e menos em desafios técnicos
básicos.
