\section{Jogos Sérios}

Jogos sérios são definidos como aplicações interativas que utilizam
o design e a tecnologia dos videogames para fins que vão além do puro
entretenimento, como educação, treinamento ou conscientização
\cite{mouaheb2012serious}. Eles surgem como uma alternativa promissora às
metodologias de ensino tradicionais, buscando engajar os alunos em ambientes de
aprendizagem imersivos e motivadores.

Contudo, a eficácia de um jogo sério não reside apenas em seu conteúdo
educacional, mas na sua capacidade de integrá-lo de forma coesa à experiência
lúdica. Uma crítica recorrente a muitos jogos com propósito educacional é que
eles falham em ser bons jogos, priorizando a instrução direta em detrimento da
jogabilidade \cite{mouaheb2012serious}. De acordo com
\citeonline{de2025codebo}, muitos jogos sérios para o ensino de programação, por
exemplo, utilizam os conceitos apenas como tema, sem incorporá-los
profundamente em suas mecânicas centrais. Isso resulta em experiências que se
assemelham mais a questionários interativos ou a exercícios de codificação
disfarçados, perdendo o potencial transformador que a mídia dos jogos pode
oferecer.
