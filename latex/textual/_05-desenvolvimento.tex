\chapter{Desenvolvimento}

Este capítulo descreve o processo de desenvolvimento do jogo educacional,
estruturado com base na metodologia ENgAGED \cite{battistella2016engaged}. O
processo orienta a construção de jogos educativos de forma sistemática,
equilibrando aspectos instrucionais e de design de jogos.

\section{Fase 1 – Análise da Unidade Instrucional (UI)}

\subsection{A1.1 Especificar a UI do jogo}

Nesta etapa, define-se o contexto instrucional em que o jogo será aplicado.  O
jogo é voltado ao ensino de \textbf{estruturas de dados} (pilha, fila e lista)
no contexto de cursos de graduação em Análise e Desenvolvimento de Sistemas.
Os objetivos de aprendizagem incluem:

\begin{itemize}
  \item Compreender o funcionamento das estruturas de dados lineares;
  \item Aplicar conceitos de inserção, remoção e ordenação em situações lúdicas;
\end{itemize}

\subsection{A1.2 Caracterizar os aprendizes}

% o jogo foi projetado para um público
% amplo, incluindo estudantes e jogadores casuais, sem exigir conhecimento
% prévio sobre estruturas de dados;

O público-alvo são estudantes de graduação em Computação, geralmente entre 18 e
30 anos, familiarizados com jogos digitais e programação básica.

O jogo foi projetado para execução em computadores pessoais com sistema Linux,
utilizando teclado como principal meio de interação.

Os aprendizes demonstram afinidade com jogos de plataforma clássicos (como
\textit{Mega Man} e \textit{Super Mario Bros.}), justificando a escolha do
gênero.

\subsection{A1.3 Definir objetivos de desempenho}

O objetivo de desempenho do jogo é que, ao final da experiência, o jogador seja
capaz de identificar e aplicar corretamente os conceitos de estruturas de dados
para resolver desafios práticos representados no ambiente do jogo.

\section{Fase 2 – Projeto da Unidade Instrucional}

\subsection{A2.1 Definir a avaliação do aluno}

A avaliação é integrada ao próprio jogo por meio das regras de combinação de
elementos alquímicos.

Ao realizar combinações corretas (elementos iguais na estrutura adequada), o
jogador obtém sucesso e causa dano ao inimigo.

Combinações incorretas resultam em penalidade (perda de vida).

Esse mecanismo atua como avaliação formativa, permitindo ao aluno aprender com
o erro.

\subsection{A2.2 Definir o conteúdo da estratégia instrucional}

O conteúdo é distribuído em níveis que representam o domínio gradual das estruturas:

\begin{itemize}
  \item \textbf{Nível 1 – Pilha:} o jogador aprende o funcionamento LIFO (último a entrar, primeiro a sair);
  \item \textbf{Nível 2 – Fila:} apresenta o conceito FIFO (primeiro a entrar, primeiro a sair);
  \item \textbf{Nível 3 – Lista:} introduz manipulações dinâmicas (inserção, remoção e ordenação);
  \item \textbf{Nível 4 – Integração:} combinação das estruturas para formar ataques complexos.
\end{itemize}

\subsection{A2.3 Decidir pelo desenvolvimento ou reutilização}

Optou-se pelo \textbf{desenvolvimento do jogo} do zero, utilizando o motor
\textit{Unity} e \textit{pixel art} autoral.

A decisão baseou-se na inexistência de jogos previamente disponíveis que
integrem a mecânica de estruturas de dados com temática de alquimia.

\subsection{A2.4 Revisar o modelo de avaliação do jogo}

O modelo de avaliação segue os princípios do \textbf{MEEGA+}
\cite{savi2011meega}, abordando usabilidade, engajamento e aprendizagem
percebida.

Os feedbacks são implementados de forma imediata no jogo, reforçando o
aprendizado.

\section{Fase 3 – Desenvolvimento do Jogo Educacional}

\subsection{A3.1 Análise do Jogo – Levantamento de requisitos}

Foram definidos os requisitos funcionais e não funcionais:

% \newcounter{frcount}
\newcommand{\frvalue}{\ifnum\value{frcount}<10 0\fi\arabic{frcount}}
\newcommand{\frid}{FR-\frvalue}

\newcommand{\fr}[3]{
  \stepcounter{frcount}

  \begin{table}[H]
    \caption{Requisito Funcional \frvalue}
    \begin{tabular}{|p{3cm}|p{8cm}|p{3cm}|}
      \hline
      \rowcolor{headergray}
      \textbf{Identificador} & \textbf{Nome} & \textbf{Prioridade} \\

      \hline
      \frid & {#1} & {#2} \\

      \hline
      \multicolumn{3}{|p{14cm}|}{\textbf{Descrição:}{#3}} \\
      \hline
    \end{tabular}
    \caption*{Fonte: Autor}
  \end{table}

  \vspace{1em}
}

\section{Requisitos Funcionais}

\fixme{escrever uma introdução}

\fr{Salvar progresso}{Alta}
{
O sistema deve permitir que os usuários salvem o progresso de suas partidas em multiplas instâncias.
}


\fr{Ponto de controle}{Médio}
{
O sistema deve possuir um ponto de controle no meio de cada fase, onde o usuário poderá retornar caso falhe.
}

\fr{Manipular inventários}{Alta}
{
O sistema deve permitir que os usuários manipulem seus inventários apartir ações.
}


% \newcounter{frcount}
\newcommand{\frvalue}{\ifnum\value{frcount}<10 0\fi\arabic{frcount}}
\newcommand{\frid}{FR-\frvalue}

\newcommand{\fr}[3]{
  \stepcounter{frcount}

  \begin{table}[H]
    \caption{Requisito Funcional \frvalue}
    \begin{tabular}{|p{3cm}|p{8cm}|p{3cm}|}
      \hline
      \rowcolor{headergray}
      \textbf{Identificador} & \textbf{Nome} & \textbf{Prioridade} \\

      \hline
      \frid & {#1} & {#2} \\

      \hline
      \multicolumn{3}{|p{14cm}|}{\textbf{Descrição:}{#3}} \\
      \hline
    \end{tabular}
    \caption*{Fonte: Autor}
  \end{table}

  \vspace{1em}
}

\section{Requisitos Funcionais}

\fixme{escrever uma introdução}

\fr{Salvar progresso}{Alta}
{
O sistema deve permitir que os usuários salvem o progresso de suas partidas em multiplas instâncias.
}


\fr{Ponto de controle}{Médio}
{
O sistema deve possuir um ponto de controle no meio de cada fase, onde o usuário poderá retornar caso falhe.
}

\fr{Manipular inventários}{Alta}
{
O sistema deve permitir que os usuários manipulem seus inventários apartir ações.
}


\subsection{A3.2 Concepção do jogo}

O jogo é um \textit{platformer} 2D fantasioso.

O protagonista é um \textit{plague doctor} que busca recuperar sua pesquisa
sobre a pedra filosofal, roubada por um alquimista rival.

O jogador avança por fases derrotando inimigos ao combinar elementos alquímicos
de acordo com regras de estruturas de dados.

Erros de combinação causam dano ao personagem.

Consumíveis especiais, como o \textbf{item de ordenação}, reorganizam as
estruturas e ajudam na progressão.

\subsection{A3.3 Design do jogo}
\begin{itemize}
  \item \textbf{Narrativa:} jornada do alquimista em busca da pedra filosofal;
  \item \textbf{Regras:} combinação correta = ataque bem-sucedido; combinação incorreta = penalidade;
  \item \textbf{Mecânicas:} movimentação lateral, salto, uso de inventários estruturados (pilha, fila, lista) e consumíveis;
  \item \textbf{Elementos visuais:} cenário em pixel art, interface minimalista e leitura clara dos elementos;
  \item \textbf{Pontuação:} baseada em eficiência de combinações e tempo de conclusão;
  \item \textbf{Feedback educacional:} explicação visual dos erros, reforçando o conceito de estrutura utilizada.
\end{itemize}

\subsection{A3.4 Implementação}
O desenvolvimento foi realizado no motor \textit{Unity}, utilizando linguagem
\texttt{C\#}.

Os sprites e animações foram criados em pixel art utilizando a ferramenta GIMP.

Os inventários (pilha, fila e lista) foram implementados como classes
independentes, permitindo reaproveitamento e clareza didática.

\subsection{A3.5 Testes do jogo}

Foram conduzidos testes funcionais e educacionais com alunos do curso de
Análise e Desenvolvimento de Sistemas, pessoas entusiastas e pessoas que não
possuem conhecimento da área.

Os resultados preliminares indicaram boa compreensão dos conceitos e
engajamento com a temática.

\section{Fase 4 – Execução da Unidade Instrucional}

Nesta fase o jogo será aplicado em sala de aula, integrando-se a atividades
práticas de revisão sobre estruturas de dados.

O professor poderá observar o desempenho dos alunos a partir das métricas de
progresso e pontuação.

\section{Fase 5 – Avaliação da Unidade Instrucional}
Serão coletados dados de engajamento e aprendizagem com base nos critérios do
MEEGA+.

Os resultados serão analisados para verificar a efetividade do jogo como
ferramenta de apoio ao ensino de estruturas de dados.
