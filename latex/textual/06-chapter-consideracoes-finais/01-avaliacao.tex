\section{Avaliação}

\fixme{Tirar a avaliação das considerações finais ?}

A avaliação do jogo foi conduzida conforme descrito na
\autoref{sec:metodologia_avaliacao}, com o objetivo de verificar se o jogo
atingia seu propósito de ser implícito, acessível e divertido. A análise foi
organizada em três dimensões principais: \textbf{Experiência do Jogador},
\textbf{Usabilidade} e \textbf{Conteúdo}. Cada dimensão permite compreender,
sob diferentes perspectivas, como os participantes perceberam a jogabilidade, a
clareza das mecânicas e a efetividade pedagógica do jogo.

A amostra da avaliação foi composta por 11 participantes. Deste grupo, todos
identificaram-se como do sexo masculino e declararam jogar jogos digitais
regularmente. Além disso, 9 participantes afirmaram possuir conhecimento prévio
na área de desenvolvimento de software.

A seguir, cada subdimensão é detalhada com sua respectiva descrição e os
resultados gráficos obtidos. Os dados baseiam-se na escala Likert de 5 pontos
\cite{likert1932}, variando dos extremos \enquote{Discordo Totalmente} a
\enquote{Concordo Totalmente}. Além da distribuição de frequências, foram
calculadas a média, a mediana e o desvio padrão de cada afirmação com base no
\autoref{apendice:dados_brutos}, fundamentando a análise quantitativa e
qualitativa subsequente.

\subsection{Experiência do Jogador}

A dimensão \enquote{Experiência do Jogador} investiga como o participante
vivencia o jogo em termos de motivação, envolvimento emocional e sensação de
progresso. Ela é fundamental para jogos sérios, pois experiências agradáveis
favorecem a aprendizagem implícita de conceitos.

A \autoref{tab:experiencia_jogador} apresenta, com base nos dados disponíveis
no \autoref{apendice:dados_brutos}, os valores de média, mediana e desvio
padrão calculados para cada afirmação dessa dimensão.

\begin{table}[H]
 	\caption{Avaliação da Experiência do Jogador - Média, Mediana e Desvio Padrão}
 	\label{tab:experiencia_jogador}
 	\centering
 	\footnotesize
  \begin{tabular}{|c|p{6cm}|ccc|}
 		\hline
 		\rowcolor{headergray}
 		\textbf{Subdimensão} & \textbf{Avaliação} & \textbf{Média} & \textbf{Mediana} & \textbf{Desvio Padrão} \\
 		\hline
 		Engajamento & O jogo conseguiu manter meu interesse. & 4.55 & 5.00 & 0.52 \\
 		\hline
    \rowcolor{accent}
 		Engajamento & Eu estava motivado a continuar jogando. & 4.36 & 4.50 & 0.50 \\
 		\hline
 		Engajamento & O jogo apresentou estímulos que mantiveram minha atenção. & 4.27 & 5.00 & 0.65 \\
 		\hline
    \rowcolor{accent}
 		Diversão & Jogar este jogo foi divertido. & 4.73 & 5.00 & 0.45 \\
 		\hline
 		Satisfação & Senti satisfação ao completar desafios ou derrotar inimigos. & 4.55 & 5.00 & 0.69 \\
 		\hline
    \rowcolor{accent}
 		Satisfação & Eu recomendaria este jogo para meus colegas. & 4.55 & 5.00 & 0.52 \\
 		\hline
 		Desafio & O nível de desafio foi adequado para mim. & 4.18 & 5.00 & 0.87 \\
 		\hline
    \rowcolor{accent}
 		Desafio & Os desafios me incentivaram a tentar melhorar. & 4.27 & 5.00 & 0.86 \\
 		\hline
 		Imersão & Eu estava tão envolvido no jogo que eu perdi a noção do tempo. & 4.09 & 4.00 & 1.04 \\
 		\hline
    \rowcolor{accent}
 		Imersão & Eu esqueci sobre o ambiente ao meu redor enquanto jogava este jogo. & 4.09 & 4.00 & 0.83 \\
 		\hline
 	\end{tabular}
 	\caption*{Fonte: Autor}
 \end{table}


\subsubsection{Engajamento}

O engajamento reflete o quanto o jogador se sente motivado e atento durante a
partida, avaliando o interesse contínuo, o foco e os estímulos providos pelo
jogo. As figuras \ref{fig:01-engajamento}, \ref{fig:02-engajamento} e
\ref{fig:03-engajamento}, a seguir, evidenciam os resultados da avaliação desta
subdimensão.

\begin{figure}[H]
	\centering
	\caption{Gráfico da pergunta de engajamento: \enquote{O jogo conseguiu manter o meu interesse.}}
	\includegraphics[width=0.8\textwidth]{images/graphs/01-engajamento.png}
	\legend{Fonte: Autor}
	\label{fig:01-engajamento}
\end{figure}

\begin{figure}[H]
	\centering
	\caption{Gráfico da pergunta de engajamento: \enquote{Eu estava motivado para continuar jogando.}}
	\includegraphics[width=0.8\textwidth]{images/graphs/02-engajamento.png}
	\legend{Fonte: Autor}
	\label{fig:02-engajamento}
\end{figure}


\begin{figure}[H]
	\centering
	\caption{Gráfico da avaliação de engajamento: \enquote{O jogo apresentou estímulos que mantiveram a minha atenção.}}
	\includegraphics[width=0.8\textwidth]{images/graphs/03-engajamento.png}
	\legend{Fonte: Autor}
	\label{fig:03-engajamento}
\end{figure}


Os índices relacionados ao engajamento, demonstrados na
\autoref{tab:experiencia_jogador}, foram elevados (médias entre 4.27 e 4.55),
com medianas altas e desvios padrão reduzidos. Tais métricas indicam um forte
consenso entre os participantes de que o jogo foi capaz de sustentar a atenção.
A baixa dispersão revela que a maioria dos usuários relatou interesse contínuo
e motivação para prosseguir na experiência.

O conjunto dos resultados sugere que o jogo mantém o interesse dos jogadores de
forma eficiente e consistente ao longo da sessão de jogo.

\subsubsection{Diversão}

A dimensão diversão avalia o prazer proporcionado pela experiência, elemento
fundamental para a motivação intrínseca em jogos sérios. A \autoref{fig:01-diversao}
apresenta a distribuição das respostas para esta subdimensão.

\begin{figure}[H]
	\centering
	\caption{Gráfico da pergunta de diversão: \enquote{Jogar este jogo foi divertido.}}
	\includegraphics[width=0.8\textwidth]{images/graphs/01-diversao.png}
	\legend{Fonte: Autor}
	\label{fig:01-diversao}
\end{figure}



Esta subdimensão apresentou a maior média da categoria (4.73) e a menor
variabilidade ($\sigma=0.45$), indicando um consenso quase unânime de que o
jogo é divertido. Os participantes elogiaram a combinação entre mecânicas de
plataforma e a manipulação de elementos, bem como a fluidez da jogabilidade.

A análise qualitativa corrobora esses dados, apontando que os elementos
visuais, sonoros e mecânicos reforçam a sensação de entretenimento, havendo
poucas ressalvas quanto à curva de dificuldade. Os dados confirmam que o jogo
cumpre com êxito seu papel lúdico.

\subsubsection{Satisfação}

A satisfação reflete a sensação de conquista e o prazer ao superar obstáculos
ou completar ações significativas no jogo. Os resultados detalhados podem ser
observados nas figuras \ref{fig:01-satisfacao} e \ref{fig:02-satisfacao}.

\begin{figure}[H]
	\centering
	\caption{Gráfico da avaliação de satisfação: \enquote{Senti satisfação ao completar desafios ou derrotar inimigos.}}
	\includegraphics[width=0.8\textwidth]{images/graphs/01-satisfacao.png}
	\legend{Fonte: Autor}
	\label{fig:01-satisfacao}
\end{figure}


\begin{figure}[H]
	\centering
	\caption{Gráfico da avaliação de satisfação: \enquote{Eu recomendaria este jogo para meus colegas.}}
	\includegraphics[width=0.8\textwidth]{images/graphs/02-satisfacao.png}
	\legend{Fonte: Autor}
	\label{fig:02-satisfacao}
\end{figure}



As médias obtidas foram altas (4.55), acompanhadas de um desvio padrão baixo, o
que demonstra que os participantes experimentaram uma forte sensação de
realização. A disposição da maioria em recomendar o jogo a colegas reforça o
impacto positivo da experiência.

Os comentários qualitativos destacam o apreço pela criatividade das mecânicas,
pela estética e pelo cuidado no polimento, evidenciando uma boa recepção da
proposta. Os resultados sugerem que o jogo proporciona uma experiência
gratificante, com uma sensação consistente de progresso.

\subsubsection{Desafio}

O desafio analisa o equilíbrio entre a dificuldade imposta e a capacidade do
jogador, buscando identificar se o jogo motiva sem gerar frustração excessiva.
As respostas dos participantes são demonstradas nas figuras
\ref{fig:01-desafio} e \ref{fig:02-desafio}.

\begin{figure}[H]
	\centering
	\caption{Gráfico da avaliação de desafio: \enquote{O nível de desafio foi adequado para mim.}}
	\includegraphics[width=0.8\textwidth]{images/graphs/01-desafio.png}
	\legend{Fonte: Autor}
	\label{fig:01-desafio}
\end{figure}


\begin{figure}[H]
	\centering
	\caption{Gráfico da avaliação de desafio: \enquote{Os desafios me incentivaram a tentar melhorar.}}
	\includegraphics[width=0.8\textwidth]{images/graphs/02-desafio.png}
	\legend{Fonte: Autor}
	\label{fig:02-desafio}
\end{figure}



Nesta subdimensão, as médias permanecem positivas (acima de 4.18), porém
observa-se uma maior variação nas respostas ($\sigma \approx 0.87$). Isso
indica que, enquanto parte do grupo considerou o desafio adequado, outra
parcela sugeriu que a curva de dificuldade inicial poderia ser suavizada.

Nos comentários, alguns participantes relataram que compreender a lógica de
combinação de elementos, ao mesmo tempo que lidavam com o comportamento dos
inimigos, exigiu atenção elevada, embora esse fator também tenha contribuído
para a motivação. Os dados apontam que o jogo oferece um desafio estimulante,
mas que pode se beneficiar de ajustes no balanceamento inicial.

\subsubsection{Imersão}

A imersão avalia o quanto o jogador se sente envolvido a ponto de perder a
noção do tempo ou esquecer o ambiente ao redor, estado essencial para facilitar
a aprendizagem implícita. As figuras \ref{fig:01-imersao} e
\ref{fig:02-imersao}, a seguir, detalham a percepção dos usuários.

\begin{figure}[H]
	\centering
	\caption{Gráfico da pergunta de desafio: \enquote{Eu estava tão envolvido no jogo que perdi a noção do tempo.}}
	\includegraphics[width=0.8\textwidth]{images/graphs/01-imersao.png}
	\legend{Fonte: Autor}
	\label{fig:01-imersao}
\end{figure}


\begin{figure}[H]
	\centering
	\caption{Gráfico da avaliação de desafio: \enquote{Eu esqueci do ambiente ao meu redor enquanto jogava.}}
	\includegraphics[width=0.8\textwidth]{images/graphs/02-imersao.png}
	\legend{Fonte: Autor}
	\label{fig:02-imersao}
\end{figure}


Os resultados mostram médias positivas (4.09), contudo, esta subdimensão
apresentou a maior dispersão da categoria ($\sigma=1.04$ na questão
representada pela \autoref{fig:01-imersao}). Isso indica que o nível de imersão
variou significativamente conforme o perfil do participante. Ainda assim, a
tendência geral demonstra que o jogo conseguiu capturar a atenção e criar uma
sensação de fluxo (\emph{flow}) para a maioria dos usuários.

As respostas qualitativas sugerem que o estilo visual, o ritmo das mecânicas e
a temática alquímica são os principais vetores dessa imersão. Embora haja
espaço para refinamentos, o jogo apresenta um desempenho satisfatório na
criação de uma atmosfera envolvente.

\subsection{Usabilidade}

A \autoref{tab:usabilidade} apresenta a média, a mediana e o desvio padrão
calculados para cada afirmação dessa dimensão, com base nos dados
disponibilizados no \autoref{apendice:dados_brutos}.

\begin{table}[H]
 	\caption{Avaliação da Usabilidade - Média, Mediana e Desvio Padrão}
 	\label{tab:usabilidade}
 	\centering
 	\footnotesize
  \begin{tabular}{|r|p{6cm}|ccc|}
 		\hline
 		\rowcolor{headergray}
 		\textbf{Subdimensão} & \textbf{Avaliação} & \textbf{Média} & \textbf{Mediana} & \textbf{Desvio Padrão} \\
 		\hline
 		Estética & O design do jogo é atraente. & 4.82 & 5.00 & 0.40 \\
 		\hline
    \rowcolor{accent}
 		Estética & Os textos, cores e fontes combinam e são consistentes. & 4.36 & 5.00 & 0.81 \\
 		\hline
 		Aprendibilidade & Aprender a jogar este jogo foi fácil para mim. & 3.55 & 4.00 & 1.12 \\
 		\hline
    \rowcolor{accent}
 		Operabilidade & Eu considero que o jogo é fácil de jogar. & 3.27 & 3.00 & 0.90 \\
 		\hline
 		Operabilidade & As regras do jogo são claras e compreensíveis. & 4.00 & 4.00 & 1.34 \\
 		\hline
    \rowcolor{accent}
 		Acessibilidade & As fontes (tamanho e estilo) utilizadas no jogo são legíveis. & 4.73 & 5.00 & 0.45 \\
 		\hline
 		Acessibilidade & As cores utilizadas no jogo são compreensíveis. & 4.64 & 5.00 & 0.50 \\
 		\hline
    \rowcolor{accent}
 		Proteção & Quando eu cometo um erro é fácil de me recuperar rapidamente. & 3.73 & 4.00 & 1.27 \\
 		\hline
 	\end{tabular}
 	\caption*{Fonte: Autor}
 \end{table}


A dimensão usabilidade avalia o quão intuitiva, clara e confortável é a
interação com o jogo. Ela engloba elementos visuais, a compreensão das
mecânicas, a navegação, a acessibilidade e a tolerância a erros.

\subsubsection{Estética}

A estética examina a qualidade visual do jogo, a coesão gráfica e a clareza dos
elementos apresentados. As figuras \ref{fig:01-estetica} e
\ref{fig:02-estetica} apresentam a distribuição das respostas.

\begin{figure}[H]
  \centering
  \caption{Gráfico da avaliação de estética: \enquote{O design do jogo é atraente.}}
  \includegraphics[width=0.8\textwidth]{images/graphs/01-estetica.png}
  \legend{Fonte: Autor}
  \label{fig:01-estetica}
\end{figure}


\begin{figure}[H]
	\centering
	\caption{Gráfico da avaliação de estética: \enquote{Os textos, cores e fontes combinam e são consistentes.}}
	\includegraphics[width=0.8\textwidth]{images/graphs/02-estetica.png}
	\legend{Fonte: Autor}
	\label{fig:02-estetica}
\end{figure}




As médias apresentaram valores expressivos, com destaque para a atratividade do
design ($\mu=4.82$) e baixíssima variação ($\sigma=0.40$). Isso indica que a
identidade visual foi muito bem avaliada, contribuindo para uma experiência
agradável.

Comentários pontuais sugeriram melhorias no menu e no tutorial, mas reforçaram
a alta qualidade estética geral. A dimensão demonstra que o jogo possui uma
direção de arte consistente e atrativa.

\subsubsection{Aprendibilidade}

A aprendibilidade analisa o quão fácil é aprender a jogar e compreender as
mecânicas principais nos primeiros momentos de interação. A
\autoref{fig:01-aprendibilidade} ilustra os dados coletados para esta métrica.

\begin{figure}[H]
	\centering
	\caption{Gráfico da pergunta de aprendibilidade: \enquote{Aprender a jogar este jogo foi fácil para mim.}}
	\includegraphics[width=0.8\textwidth]{images/graphs/01-aprendibilidade.png}
	\legend{Fonte: Autor}
	\label{fig:01-aprendibilidade}
\end{figure}


Os dados revelam um cenário misto. A média obtida foi de 3.55, com um desvio
padrão elevado ($\sigma=1.12$), indicando divergência de opiniões. Enquanto
parte dos usuários considerou o aprendizado simples, outra parcela
significativa encontrou barreiras iniciais.

Os comentários qualitativos apontam que o tutorial inicial carece de clareza,
especialmente na explicação do sistema de danos e nas combinações de elementos.
A análise sugere que, embora a mecânica seja compreensível a longo prazo, a
curva de aprendizado inicial necessita de intervenções pedagógicas e
instrucionais.

\subsubsection{Operabilidade}

A operabilidade examina a clareza dos comandos, a responsividade e a facilidade
de controle do personagem. Os gráficos apresentados nas figuras
\ref{fig:01-operabilidade} e \ref{fig:02-operabilidade} evidenciam a avaliação
destes aspectos.

\begin{figure}[H]
	\centering
	\caption{Gráfico da pergunta de operabilidade: \enquote{Eu considero que o jogo é fácil de jogar.}}
	\includegraphics[width=0.8\textwidth]{images/graphs/01-operabilidade.png}
	\legend{Fonte: Autor}
	\label{fig:01-operabilidade}
\end{figure}

\begin{figure}[H]
	\centering
	\caption{Gráfico da avaliação de operabilidade: \enquote{As regras do jogo são claras e compreensíveis.}}
	\includegraphics[width=0.8\textwidth]{images/graphs/02-operabilidade.png}
	\legend{Fonte: Autor}
	\label{fig:02-operabilidade}
\end{figure}


Esta subdimensão apresentou os índices mais baixos da avaliação ($\mu=3.27$ na
questão representada na \autoref{fig:01-operabilidade}), sinalizando pontos de
fricção na jogabilidade.

Por outro lado, a clareza das regras obteve média 4.00, mas com um desvio
padrão muito alto ($\sigma=1.34$), reforçando que a compreensão do sistema
variou drasticamente entre os usuários. Os resultados evidenciam que a
operabilidade é o ponto crítico a ser priorizado em futuras atualizações.

\subsubsection{Acessibilidade}

A acessibilidade avalia a legibilidade, a clareza visual, o contraste e a
adequação geral da interface para diferentes perfis de usuários. As figuras
\ref{fig:01-acessibilidade} e \ref{fig:02-acessibilidade} mostram os resultados
obtidos.

\begin{figure}[H]
	\centering
	\caption{Gráfico da pergunta de acessibilidade: \enquote{As fontes (tamanho e estilo) utilizadas no jogo são legíveis.}}
	\includegraphics[width=0.8\textwidth]{images/graphs/01-acessibilidade.png}
	\legend{Fonte: Autor}
	\label{fig:01-acessibilidade}
\end{figure}

\begin{figure}[H]
	\centering
	\caption{Gráfico da avaliação de acessibilidade: \enquote{As cores utilizadas no jogo são compreensíveis.}}
	\includegraphics[width=0.8\textwidth]{images/graphs/02-acessibilidade.png}
	\legend{Fonte: Autor}
	\label{fig:02-acessibilidade}
\end{figure}


Diferentemente da operabilidade, a acessibilidade visual obteve médias altas
($\mu > 4.60$) e variabilidade baixa, indicando que os participantes
consideraram as fontes e as cores do jogo legíveis e compreensíveis. O jogo se
mostra visualmente acessível e adequado para públicos diversos, sem apresentar
barreiras significativas de leitura ou interpretação cromática.

\subsubsection{Proteção Contra Erros}

Esta subdimensão avalia a capacidade do jogo de permitir a recuperação rápida
em caso de erros cometidos pelo jogador. A \autoref{fig:01-protecao} demonstra
a percepção dos jogadores quanto a este aspecto.

\begin{figure}[H]
	\centering
	\caption{Gráfico da pergunta de proteção: \enquote{Quando eu cometo um erro é fácil de me recuperar rapidamente.}}
	\includegraphics[width=0.8\textwidth]{images/graphs/01-protecao.png}
	\legend{Fonte: Autor}
	\label{fig:01-protecao}
\end{figure}


Os resultados mostram médias moderadas (3.73) com variabilidade considerável
($\sigma=1.27$). A aleatoriedade na geração dos elementos foi citada como um
fator que, ocasionalmente, dificulta a recuperação após um erro de combinação.
Apesar disso, a mecânica geral não foi percebida como excessivamente punitiva,
demonstrando uma boa tolerância a falhas, passível de ajustes finos no
balanceamento.

\subsection{Conteúdo}

A dimensão conteúdo verifica se o jogo representa adequadamente os conceitos
educacionais e se os jogadores percebem valor na experiência pedagógica.

A \autoref{tab:conteudo} apresenta a média, a mediana e o desvio padrão
calculados para cada afirmação dessa dimensão, com base nos dados
disponibilizados no \autoref{apendice:dados_brutos}.

\begin{table}[H]
 	\caption{Avaliação do Conteúdo - Média, Mediana e Desvio Padrão}
 	\label{tab:conteudo}
 	\centering
 	\footnotesize
  \begin{tabular}{|r|p{6cm}|ccc|}
 		\hline
 		\rowcolor{headergray}
 		\textbf{Subdimensão} & \textbf{Avaliação} & \textbf{Média} & \textbf{Mediana} & \textbf{Desvio Padrão} \\
 		\hline
 		Relevância & Conceitos de Estruturas de Dados bem representados. & 4.45 & 5.00 & 0.52 \\
 		\hline
    \rowcolor{accent}
 		Relevância & O jogo é útil para treinar conceitos já conhecidos. & 4.10 & 5.00 & 0.94 \\
 		\hline
 		Relevância & A abordagem implícita foi positiva. & 4.27 & 5.00 & 1.35 \\
 		\hline
 	\end{tabular}
 	\caption*{Fonte: Autor}
 \end{table}


\subsubsection{Relevância}

A relevância examina se os conceitos de pilha, fila e lista encadeada foram
compreendidos e vistos como úteis dentro do contexto do jogo. As figuras
\ref{fig:01-relevancia}, \ref{fig:02-relevancia} e \ref{fig:03-relevancia}
detalham a avaliação do conteúdo.

\begin{figure}[H]
	\centering
	\caption{Gráfico da avaliação de relevância: \enquote{Os conceitos de estruturas de dados (pilha, fila e lista encadeada) foram bem representados no jogo, mesmo que de forma implícita.}}
	\includegraphics[width=0.8\textwidth]{images/graphs/01-relevancia.png}
	\legend{Fonte: Autor}
	\label{fig:01-relevancia}
\end{figure}


\begin{figure}[H]
	\centering
	\caption{Gráfico da pergunta de relevância: \enquote{O jogo é útil para treinar conceitos já conhecidos.}}
	\includegraphics[width=0.8\textwidth]{images/graphs/02-relevancia.png}
	\legend{Fonte: Autor}
	\label{fig:02-relevancia}
\end{figure}

\begin{figure}[H]
	\centering
	\caption{Gráfico da pergunta de relevância: \enquote{A abordagem implícita (ensinar sem explicar diretamente) foi positiva.}}
	\includegraphics[width=0.8\textwidth]{images/graphs/03-relevancia.png}
	\legend{Fonte: Autor}
	\label{fig:03-relevancia}
\end{figure}


As médias foram altas ($\mu=4.45$ para representação dos conceitos), indicando
que a transposição das estruturas de dados para as mecânicas de jogo foi
bem-sucedida. Comentários afirmam que os conceitos foram apresentados
\enquote{de maneira simples, mas efetiva} e que o jogo auxilia no treinamento
de conhecimentos já adquiridos.

Vale ressaltar o alto desvio padrão ($\sigma=1.35$) na questão sobre a
abordagem implícita (\autoref{fig:03-relevancia}). Isso sugere que, embora a
média seja positiva (4.27), a eficácia do ensino implícito não foi percebida da
mesma forma por todos; alguns usuários apreciaram a sutileza, enquanto outros
talvez preferissem instruções mais diretas. De modo geral, os dados indicam que
o conteúdo pedagógico é percebido como relevante e correto, validando o
potencial educacional do jogo.

\subsection{Análise Qualitativa}

Além dos dados quantitativos coletados através das escalas Likert
\cite{likert1932}, o questionário disponibilizou campos abertos para que os
participantes pudessem descrever suas experiências, dificuldades e sugestões. A
análise dessas respostas permite uma compreensão mais profunda de como as
mecânicas de jogo dialogaram com os conceitos teóricos propostos.

De modo geral, a recepção da mecânica central, utilizar inventários que se
comportam como estruturas de dados, foi destacada como o ponto alto da
originalidade do projeto. Um dos participantes descreveu a experiência fazendo
uma analogia interessante com jogos clássicos de quebra-cabeça:

\begin{quote}
    \enquote{Gostei da forma em que mistura um jogo 2D com um Tetris, em que
    você tem que manejar sua mão de elementos e ao mesmo tempo ficar de olho na
    sua vida e nos inimigos [...]. A ideia de cada slot de elemento agir de uma
    forma diferente ficou bem legal.}
\end{quote}

Essa percepção sugere que o objetivo de gamificar o gerenciamento de memória
foi atingido, transformando operações abstratas de inserção e remoção em uma
mecânica de sobrevivência e combate. Outro participante ressaltou a eficácia
dos recursos de apoio, especificamente o \emph{Livro de Tutoriais} (item 5 da
\autoref{fig:hud}), na consolidação do aprendizado:

\begin{quote}
    \enquote{Achei o livro ótimo para estudar estrutura de dados, associar o
    inventário com suas respectivas estruturas e os consumíveis com suas
    respectivas operações...}
\end{quote}

No que tange às dificuldades, o principal ponto de fricção identificado foi a
aleatoriedade na geração dos elementos alquímicos. Como o jogador depende de
combinações específicas para atacar, a geração aleatória por vezes entrava em
conflito com as restrições rígidas das estruturas (como o topo da pilha ou o
início da fila), gerando travamentos no fluxo de jogo. Um usuário relatou:

\begin{quote}
  \enquote{Acredito que o principal aspecto de dificuldade do jogo é o RNG da
  geração de elementos, que faz com que muitas vezes seja necessário descartar
  elementos para tornar possível combinar os certos para derrotar o inimigo.}
\end{quote}

Embora o ato de \enquote{descartar} elementos force o jogador a exercitar as
operações de remoção (\emph{pop/dequeue}), o excesso de aleatoriedade foi
percebido como uma barreira à fluidez. Outros comentários apontaram que a curva
de aprendizado inicial é acentuada, sugerindo que \enquote{seria interessante
um tutorial mais guiado no começo}, visto que a necessidade de combinar o
elemento correto com a fraqueza do inimigo não foi imediatamente óbvia para
todos.

Em síntese, a análise qualitativa revela que o jogo foi bem-sucedido em criar
uma mecânica inovadora e engajadora para o ensino de estruturas de dados. As
críticas concentraram-se majoritariamente no balanceamento da aleatoriedade e
na necessidade de um \emph{onboarding} mais suave, aspectos que foram
incorporados como sugestões para trabalhos futuros.

A seguir, na \autoref{sec:trabalhos_futuros}, são discutidos os próximos passos
do projeto. Nessa seção serão retomados os \emph{feedbacks} obtidos durante a
avaliação, os aspectos que não puderam ser implementados nesta versão e algumas
ideias novas para orientar melhorias futuras.
