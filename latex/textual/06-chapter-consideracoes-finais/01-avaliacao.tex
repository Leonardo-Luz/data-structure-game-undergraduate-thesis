\section{Avaliação}

A avaliação do jogo foi conduzida conforme descrito na
\autoref{sec:metodologia_avaliacao}, com o objetivo de verificar se o jogo
atingia seu propósito de ser implícito, acessível e divertido. A análise foi
organizada em três dimensões principais: \textbf{Experiência do Jogador},
\textbf{Usabilidade} e \textbf{Conteúdo}. Cada dimensão permite compreender,
sob diferentes perspectivas, como os participantes perceberam a jogabilidade,
a clareza das mecânicas e a efetividade pedagógica do jogo.

A \textbf{Experiência do Jogador} examina o envolvimento emocional e cognitivo
durante a jogatina, considerando fatores como engajamento, diversão,
satisfação, desafio e imersão. A \textbf{Usabilidade} avalia o quão intuitivo,
claro e consistente é o processo de interação com o jogo. Por fim, o
\textbf{Conteúdo} busca verificar se os conceitos educacionais presentes no
jogo foram compreendidos e percebidos como relevantes.

A seguir, cada subdimensão é apresentada com sua descrição, gráficos
correspondentes, análise quantitativa e qualitativa e uma síntese ao final.

\subsection{Experiência do Jogador}

\fixme{calcular os valores de media, mediana e desvios-padrão}

\fixme{mudar os gráficos para apendice ?}

A dimensão \enquote{Experiência do Jogador} investiga como o participante vivencia
o jogo em termos de motivação, envolvimento emocional e sensação de progresso.
Ela é fundamental para jogos educativos, pois experiências agradáveis favorecem
a aprendizagem implícita de conceitos.

\subsubsection{Engajamento}

O engajamento reflete o quanto o jogador se sente motivado e atento durante a
partida, avaliando interesse contínuo, foco e estímulos do jogo.

\begin{figure}[H]
	\centering
	\caption{Gráfico da pergunta de engajamento: \enquote{O jogo conseguiu manter o meu interesse.}}
	\includegraphics[width=0.8\textwidth]{images/graphs/01-engajamento.png}
	\legend{Fonte: Autor}
	\label{fig:01-engajamento}
\end{figure}

\begin{figure}[H]
	\centering
	\caption{Gráfico da pergunta de engajamento: \enquote{Eu estava motivado para continuar jogando.}}
	\includegraphics[width=0.8\textwidth]{images/graphs/02-engajamento.png}
	\legend{Fonte: Autor}
	\label{fig:02-engajamento}
\end{figure}


\begin{figure}[H]
	\centering
	\caption{Gráfico da avaliação de engajamento: \enquote{O jogo apresentou estímulos que mantiveram a minha atenção.}}
	\includegraphics[width=0.8\textwidth]{images/graphs/03-engajamento.png}
	\legend{Fonte: Autor}
	\label{fig:03-engajamento}
\end{figure}


As médias dos itens relacionados ao engajamento foram altas, com medianas
similares e desvios-padrão reduzidos, indicando avaliações consistentes entre
os participantes. A baixa dispersão revela que a maior parte deles relatou
interesse contínuo e motivação para prosseguir no jogo.

Os comentários qualitativos reforçam essa percepção, destacando que a
dificuldade moderada e a mecânica de combinação de elementos são fatores que
ajudam a manter a atenção. Alguns participantes citaram que a aleatoriedade
dos elementos pode ocasionalmente interferir no ritmo do engajamento, mas sem
comprometer a experiência geral.

O conjunto dos resultados indica que o jogo mantém o interesse dos jogadores de
forma eficiente e consistente ao longo da experiência.

\subsubsection{Diversão}

A dimensão diversão avalia o prazer proporcionado pela experiência, elemento
fundamental para a motivação intrínseca nos jogos educativos.

\begin{figure}[H]
	\centering
	\caption{Gráfico da pergunta de diversão: \enquote{Jogar este jogo foi divertido.}}
	\includegraphics[width=0.8\textwidth]{images/graphs/01-diversao.png}
	\legend{Fonte: Autor}
	\label{fig:01-diversao}
\end{figure}



As médias e medianas apresentaram valores elevados, com baixa variabilidade,
indicando consenso de que o jogo é divertido. Os participantes elogiaram a
combinação entre plataforma e manipulação de elementos, assim como a fluidez da
jogatina.

A análise qualitativa mostra que os elementos visuais, sonoros e mecânicos
reforçam a sensação de diversão, com poucos comentários sugerindo ajustes na
curva de dificuldade.

Os dados indicam que o jogo cumpre com sucesso seu papel lúdico, sendo
percebido como uma experiência divertida e agradável.

\subsubsection{Satisfação}

A satisfação reflete a sensação de conquista e o prazer ao superar obstáculos
ou completar ações significativas no jogo.

\begin{figure}[H]
	\centering
	\caption{Gráfico da avaliação de satisfação: \enquote{Senti satisfação ao completar desafios ou derrotar inimigos.}}
	\includegraphics[width=0.8\textwidth]{images/graphs/01-satisfacao.png}
	\legend{Fonte: Autor}
	\label{fig:01-satisfacao}
\end{figure}


\begin{figure}[H]
	\centering
	\caption{Gráfico da avaliação de satisfação: \enquote{Eu recomendaria este jogo para meus colegas.}}
	\includegraphics[width=0.8\textwidth]{images/graphs/02-satisfacao.png}
	\legend{Fonte: Autor}
	\label{fig:02-satisfacao}
\end{figure}



As médias foram altas e o desvio-padrão baixo, mostrando que os participantes
experimentaram forte sensação de realização. Muitos afirmaram que
recomendariam o jogo a outras pessoas, o que reforça o impacto positivo da
experiência.

Os comentários destacam apreço pela criatividade da mecânica, pela estética e
pelo cuidado no desenvolvimento, mostrando que a proposta foi bem recebida.

Os resultados sugerem que o jogo proporciona uma experiência recompensadora,
com sensação consistente de progresso.

\subsubsection{Desafio}

O desafio analisa o equilíbrio entre dificuldade e capacidade do jogador,
buscando identificar se o jogo motiva sem frustrar.

\begin{figure}[H]
	\centering
	\caption{Gráfico da avaliação de desafio: \enquote{O nível de desafio foi adequado para mim.}}
	\includegraphics[width=0.8\textwidth]{images/graphs/01-desafio.png}
	\legend{Fonte: Autor}
	\label{fig:01-desafio}
\end{figure}


\begin{figure}[H]
	\centering
	\caption{Gráfico da avaliação de desafio: \enquote{Os desafios me incentivaram a tentar melhorar.}}
	\includegraphics[width=0.8\textwidth]{images/graphs/02-desafio.png}
	\legend{Fonte: Autor}
	\label{fig:02-desafio}
\end{figure}



Nesta subdimensão, as médias permanecem em nível positivo, mas com maior
variação entre os participantes. Parte deles considerou o desafio adequado,
enquanto outros sugeriram suavizar o nível de dificuldade inicial.

Nos comentários, alguns relataram que entender a lógica da combinação de
elementos e acompanhar simultaneamente o comportamento dos inimigos exigiu
atenção, mas também contribuiu para a motivação.

O conjunto dos dados aponta que o jogo oferece um desafio estimulante, mas que
pode ser ajustado para melhorar a fluidez inicial.

\subsubsection{Imersão}

A imersão avalia o quanto o jogador se sente envolvido a ponto de esquecer o
ambiente ao redor, elemento essencial para a aprendizagem implícita.

\begin{figure}[H]
	\centering
	\caption{Gráfico da pergunta de desafio: \enquote{Eu estava tão envolvido no jogo que perdi a noção do tempo.}}
	\includegraphics[width=0.8\textwidth]{images/graphs/01-imersao.png}
	\legend{Fonte: Autor}
	\label{fig:01-imersao}
\end{figure}


\begin{figure}[H]
	\centering
	\caption{Gráfico da avaliação de desafio: \enquote{Eu esqueci do ambiente ao meu redor enquanto jogava.}}
	\includegraphics[width=0.8\textwidth]{images/graphs/02-imersao.png}
	\legend{Fonte: Autor}
	\label{fig:02-imersao}
\end{figure}


Os resultados mostram médias positivas, embora com maior dispersão que outras
subdimensões. Isso indica que a imersão variou mais entre os perfis dos
participantes. Ainda assim, a tendência geral demonstra que o jogo conseguiu
capturar a atenção e criar sensação de fluxo.

As respostas qualitativas sugerem que o estilo visual, o ritmo da mecânica e a
temática alquímica contribuem para essa imersão.

Embora haja espaço para refinamentos, o jogo apresenta bom desempenho em termos
de experiência imersiva.

\subsection{Usabilidade}

A dimensão usabilidade avalia o quão intuitiva, clara e confortável é a
interação com o jogo. Ela engloba elementos visuais, compreensão das mecânicas,
navegação, acessibilidade e tolerância a erros.

\subsubsection{Estética}

A estética examina a qualidade visual do jogo, coesão gráfica e clareza dos
elementos apresentados.

\begin{figure}[H]
  \centering
  \caption{Gráfico da avaliação de estética: \enquote{O design do jogo é atraente.}}
  \includegraphics[width=0.8\textwidth]{images/graphs/01-estetica.png}
  \legend{Fonte: Autor}
  \label{fig:01-estetica}
\end{figure}


\begin{figure}[H]
	\centering
	\caption{Gráfico da avaliação de estética: \enquote{Os textos, cores e fontes combinam e são consistentes.}}
	\includegraphics[width=0.8\textwidth]{images/graphs/02-estetica.png}
	\legend{Fonte: Autor}
	\label{fig:02-estetica}
\end{figure}




As médias apresentaram valores altos, com pequena variação entre os
participantes. Isso indica que o design visual foi bem avaliado, contribuindo
para uma experiência agradável e clara.

Comentários sugerem pequenas melhorias no menu e no tutorial, mas reforçam a
boa qualidade estética geral.

A dimensão demonstra que o jogo possui estética consistente e atrativa.

\subsubsection{Aprendibilidade}

A aprendibilidade analisa o quão fácil é aprender a jogar e compreender as
mecânicas principais.

\begin{figure}[H]
	\centering
	\caption{Gráfico da pergunta de aprendibilidade: \enquote{Aprender a jogar este jogo foi fácil para mim.}}
	\includegraphics[width=0.8\textwidth]{images/graphs/01-aprendibilidade.png}
	\legend{Fonte: Autor}
	\label{fig:01-aprendibilidade}
\end{figure}


Os dados mostram médias positivas, mas com maior variabilidade que outras
subdimensões. Parte dos comentários aponta que o tutorial inicial poderia ser
mais claro, especialmente para explicar o sistema de dano e combinação de
elementos.

Apesar disso, a maior parte dos participantes afirmou que aprender a jogar foi
relativamente simples.

A análise sugere que a aprendibilidade é boa, mas pode ser fortalecida com
melhorias no tutorial.

\subsubsection{Operabilidade}

A operabilidade examina a clareza dos comandos, responsividade e facilidade de
navegação.

\begin{figure}[H]
	\centering
	\caption{Gráfico da pergunta de operabilidade: \enquote{Eu considero que o jogo é fácil de jogar.}}
	\includegraphics[width=0.8\textwidth]{images/graphs/01-operabilidade.png}
	\legend{Fonte: Autor}
	\label{fig:01-operabilidade}
\end{figure}

\begin{figure}[H]
	\centering
	\caption{Gráfico da avaliação de operabilidade: \enquote{As regras do jogo são claras e compreensíveis.}}
	\includegraphics[width=0.8\textwidth]{images/graphs/02-operabilidade.png}
	\legend{Fonte: Autor}
	\label{fig:02-operabilidade}
\end{figure}


As respostas foram positivas, embora alguns participantes tenham relatado
dificuldades com comandos específicos, como o pulo. A maior parte avaliou que
as regras são claras e que jogar é intuitivo.

Os resultados mostram que o jogo oferece boa operabilidade, com ajustes
pontuais necessários em alguns controles.

\subsubsection{Acessibilidade}

A acessibilidade avalia legibilidade, clareza visual, contraste e adequação
geral da interface para diferentes perfis.

\begin{figure}[H]
	\centering
	\caption{Gráfico da pergunta de acessibilidade: \enquote{As fontes (tamanho e estilo) utilizadas no jogo são legíveis.}}
	\includegraphics[width=0.8\textwidth]{images/graphs/01-acessibilidade.png}
	\legend{Fonte: Autor}
	\label{fig:01-acessibilidade}
\end{figure}

\begin{figure}[H]
	\centering
	\caption{Gráfico da avaliação de acessibilidade: \enquote{As cores utilizadas no jogo são compreensíveis.}}
	\includegraphics[width=0.8\textwidth]{images/graphs/02-acessibilidade.png}
	\legend{Fonte: Autor}
	\label{fig:02-acessibilidade}
\end{figure}


As médias foram altas e a variabilidade baixa, indicando que os participantes
consideraram o jogo visualmente claro e acessível.

Embora alguns comentários mencionem ajustes de fonte, a percepção geral é
bastante positiva.

O jogo se mostra acessível e adequado para públicos diversos.

\subsubsection{Proteção Contra Erros}

Esta subdimensão avalia a capacidade do jogo de permitir recuperação rápida em
caso de erros cometidos pelo jogador.

\begin{figure}[H]
	\centering
	\caption{Gráfico da pergunta de proteção: \enquote{Quando eu cometo um erro é fácil de me recuperar rapidamente.}}
	\includegraphics[width=0.8\textwidth]{images/graphs/01-protecao.png}
	\legend{Fonte: Autor}
	\label{fig:01-protecao}
\end{figure}


Os resultados mostram médias favoráveis, com moderada variabilidade. A
aleatoriedade na geração dos elementos foi citada como fator que ocasionalmente
dificulta a recuperação, mas a mecânica geral não penaliza excessivamente o
usuário.

O jogo demonstra boa tolerância a erros, com oportunidades de ajustes no
balanceamento.

\subsection{Conteúdo}

A dimensão conteúdo verifica se o jogo representa adequadamente os conceitos
educacionais e se os jogadores percebem valor na experiência pedagógica.

\subsubsection{Relevância}

A relevância examina se os conceitos de pilha, fila e lista encadeada foram
compreendidos e vistos como úteis.

\begin{figure}[H]
	\centering
	\caption{Gráfico da avaliação de relevância: \enquote{Os conceitos de estruturas de dados (pilha, fila e lista encadeada) foram bem representados no jogo, mesmo que de forma implícita.}}
	\includegraphics[width=0.8\textwidth]{images/graphs/01-relevancia.png}
	\legend{Fonte: Autor}
	\label{fig:01-relevancia}
\end{figure}


\begin{figure}[H]
	\centering
	\caption{Gráfico da pergunta de relevância: \enquote{O jogo é útil para treinar conceitos já conhecidos.}}
	\includegraphics[width=0.8\textwidth]{images/graphs/02-relevancia.png}
	\legend{Fonte: Autor}
	\label{fig:02-relevancia}
\end{figure}

\begin{figure}[H]
	\centering
	\caption{Gráfico da pergunta de relevância: \enquote{A abordagem implícita (ensinar sem explicar diretamente) foi positiva.}}
	\includegraphics[width=0.8\textwidth]{images/graphs/03-relevancia.png}
	\legend{Fonte: Autor}
	\label{fig:03-relevancia}
\end{figure}


As médias foram altas e a variabilidade baixa, indicando boa compreensão dos
conceitos representados no jogo. Comentários afirmam que os conceitos foram
apresentados \enquote{de maneira simples, mas efetiva} e que o jogo auxilia no
treinamento de conhecimentos já adquiridos.

Sugestões mencionam ajustes na clareza inicial e no balanceamento dos elementos
necessários para combinações.

Os dados indicam que o conteúdo pedagógico é percebido como relevante,
correto e útil, reforçando o potencial educacional do jogo.

A seguir, na \autoref{sec:trabalhos_futuros}, são discutidos os próximos passos
do projeto. Nessa seção serão retomados os \emph{feedbacks} obtidos durante a
avaliação, os aspectos que não puderam ser implementados nesta versão e algumas
ideias novas para orientar melhorias futuras.
