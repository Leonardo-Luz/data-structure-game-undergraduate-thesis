\section{Trabalhos Futuros} \label{sec:trabalhos_futuros}

O desenvolvimento deste jogo sério cumpriu os objetivos fundamentais
estabelecidos para a fase de concepção e implementação do protótipo. No
entanto, devido a restrições de cronograma, algumas etapas previstas na
metodologia ENgAGED não puderam ser concluídas no escopo atual. A
\autoref{tab:tf_incompletos} detalha estas etapas, que são essenciais para a
validação pedagógica formal da ferramenta.

\begin{table}[H]
    \caption{Trabalhos Futuros: Objetivos Incompletos}
    \label{tab:tf_incompletos}
    \centering
    \footnotesize
    \begin{tabular}{|p{4cm}|p{10cm}|}
        \hline
        \rowcolor{headergray}
        \textbf{Trabalho Futuro} & \textbf{Descrição} \\
        \hline
        Execução da Unidade Instrucional & Executar a fase 4 da metodologia ENgAGED, aplicando o jogo em contexto real de ensino e realizando coleta sistemática de dados. \\
        \hline
        Avaliação da Unidade Instrucional & Executar a fase 5 da metodologia ENgAGED, avaliando a eficácia pedagógica do jogo em sala de aula. \\
        \hline
        Fase 2 e 3 & Concluir a narrativa do jogo, introduzindo o alquimista rival como chefão final e adicionando desafios pedagógicos complementares. \\
        \hline
    \end{tabular}
    \caption*{Fonte: Autor}
\end{table}


Além das etapas metodológicas pendentes, a avaliação realizada com os usuários
(descrita na seção anterior) forneceu \emph{insights} valiosos sobre aspectos
técnicos e de design que necessitam de refinamento. A \autoref{tab:tf_feedback}
sintetiza as melhorias prioritárias identificadas a partir da análise
qualitativa dos testes.

\begin{table}[H]
    \caption{Trabalhos Futuros: Melhorias baseadas no \emph{Feedback}}
    \label{tab:tf_feedback}
    \centering
    \footnotesize
    \begin{tabular}{|p{4cm}|p{10cm}|}
        \hline
        \rowcolor{headergray}
        \textbf{Melhoria} & \textbf{Justificativa} \\
        \hline
        Balanceamento da geração aleatória de elementos & Ajustar o algoritmo
        de geração de elementos para reduzir a aleatoriedade excessiva,
        evitando situações em que o jogador fica \enquote{travado} sem as
        combinações necessárias. \\
        \hline
        Reformulação do Tutorial & Criar uma introdução mais guiada e interativa, explicando com clareza as mecânicas de dano elemental e combinação, suavizando a curva de aprendizado inicial. \\
        \hline
    \end{tabular}
    \caption*{Fonte: Autor}
\end{table}


Por fim, visando expandir o ciclo de vida do jogo e seu potencial de
engajamento, foram mapeadas oportunidades de evolução que extrapolam o escopo
original. Estas ideias, apresentadas na \autoref{tab:tf_ideias}, buscam
modernizar a acessibilidade e ampliar a profundidade pedagógica da ferramenta.

\begin{table}[H]
    \caption{Trabalhos Futuros: Ideias de Expansão}
    \label{tab:tf_ideias}
    \centering
    \footnotesize
    \begin{tabular}{|p{4cm}|p{10cm}|}
        \hline
        \rowcolor{headergray}
        \textbf{Expansão} & \textbf{Motivação} \\
        \hline
        Versão Mobile & Aumentar o alcance e acessibilidade do jogo, atingindo novos públicos como estudantes que utilizam dispositivos móveis como plataforma principal. \\
        \hline
        Modo Arena & Introduzir elementos \emph{arcade} focados em repetição e treino, reforçando os conceitos de estruturas de dados por meio de desafios rápidos e pontuação competitiva. \\
        \hline
        Modo PvP & Promover competitividade e interação social, permitindo que jogadores utilizem estratégias de gerenciamento de inventário para superar oponentes em tempo real. \\
        \hline
        Novas Estruturas (Árvores/Grafos) & Tornar o jogo mais robusto pedagogicamente, abordando tópicos avançados da disciplina e ampliando o repertório de mecânicas disponíveis. \\
        \hline
    \end{tabular}
    \caption*{Fonte: Autor}
\end{table}


A implementação destes trabalhos futuros visa permitir que o jogo evolua de um
protótipo funcional para uma ferramenta educacional robusta, capaz de atender a
diferentes perfis de aprendizes e contextos de ensino. Com essas perspectivas
de continuidade estabelecidas, encerra-se este estudo, apresentando as
conclusões finais na \autoref{sec:conclusao} a seguir.
