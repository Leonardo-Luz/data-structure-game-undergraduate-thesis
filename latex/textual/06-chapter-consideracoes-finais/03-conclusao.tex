\section{Conclusão} \label{sec:conclusao}

O desenvolvimento deste trabalho permitiu explorar o potencial dos jogos sérios
como ferramenta de apoio ao ensino de Computação, abordando especificamente a
dificuldade histórica no aprendizado de Estruturas de Dados. Ao integrar
conceitos de pilhas, filas e listas encadeadas em uma narrativa fantasiosa de
alquimia e mecânicas de plataforma, o projeto buscou mitigar a abstração desses
temas, oferecendo uma alternativa lúdica ao ensino tradicional.

A aplicação da metodologia ENgAGED garantiu que o desenvolvimento não se
distanciasse dos objetivos pedagógicos, resultando em um artefato que equilibra
entretenimento e educação. A avaliação realizada com o público-alvo demonstrou que a
abordagem implícita foi bem-sucedida: os jogadores foram capazes de identificar e
operar as lógicas das estruturas de dados para resolver problemas de combate e
sobrevivência, validando a hipótese de que é possível aprender conceitos complexos
através da experimentação prática em um ambiente de jogo.

Os resultados quantitativos e qualitativos indicaram índices elevados de engajamento,
diversão e estética. A analogia criada entre a manipulação dos inventários e as
operações de memória foi percebida como um diferencial inovador. No entanto, o
estudo também evidenciou desafios técnicos, especificamente no balanceamento da
aleatoriedade (RNG) e na precisão dos controles (operabilidade), fatores que, embora
não tenham invalidado a proposta educacional, impactaram a fluidez da experiência
inicial.

\fixme{
  Conclui-se, portanto, que o jogo atingiu seu objetivo geral de apresentar conceitos
  de estruturas de dados de forma implícita e motivadora. O projeto contribui para a
  área de Informática na Educação ao fornecer um exemplo prático de como mecânicas de
  jogos de ação podem ser utilizadas para o ensino de lógica, fugindo dos tradicionais
  \emph{quizzes} e \emph{puzzles} explícitos. As bases estabelecidas neste trabalho
  oferecem um caminho sólido para futuras iterações que poderão refinar a jogabilidade
  e expandir o conteúdo pedagógico para novos níveis de complexidade.
}
