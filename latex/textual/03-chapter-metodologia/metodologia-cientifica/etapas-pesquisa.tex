\subsection{Etapas da Pesquisa}

Esta pesquisa foi composta por diversas etapas e teve início com uma
Revisão Bibliográfica, por meio da qual foram identificadas
publicações acadêmicas e aplicações relacionadas ao uso de jogos sérios no
ensino de conceitos de programação. Essa etapa proporcionou uma visão das
abordagens já utilizadas, seus resultados e os conceitos mais frequentemente
aplicados.

Em seguida, foi idealizado e desenvolvido um jogo sério com base na metodologia
ENgAGED \cite{battistella2016engaged} adaptada. Esse processo envolveu diversas
fases como a concepção pedagógica, o design e a implementação do jogo proposto.

Posteriormente, realizou-se a coleta de dados por meio do modelo MEEGA+
\cite{meega2020}, recomendado pela metodologia ENgAGED. Nessa etapa, foram
empregadas técnicas qualitativas e quantitativas com o objetivo de avaliar a
experiência dos participantes, a usabilidade do jogo e a assimilação dos
conteúdos implícitos presentes nas mecânicas desenvolvidas.

Por fim, realizou-se a análise e discussão dos resultados. Essa etapa envolveu
o cruzamento dos dados obtidos com os objetivos da pesquisa, a fim de verificar
se o jogo efetivamente representou, de forma consistente, os conceitos de
estruturas de dados e, sobretudo, se proporcionou aos participantes uma
experiência engajadora, divertida e motivadora.
