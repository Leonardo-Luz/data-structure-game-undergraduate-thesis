\subsection{Etapas da Pesquisa}

Esta pesquisa foi composta por diversas etapas e teve início com uma revisão
bibliográfica, por meio da qual foram identificadas publicações acadêmicas e
aplicações relacionadas ao uso de jogos sérios no ensino de conceitos de
programação. Essa etapa proporcionou uma visão das abordagens já utilizadas,
seus resultados e os conceitos mais frequentemente aplicados.

Em seguida, foi idealizado e desenvolvido um jogo sério educacional com base na
metodologia ENgAGED \cite{battistella2016engaged}. Esse processo envolveu
diversas fases como a concepção, design e implementação do jogo proposto,
seguido da coleta de dados por meio de testes com usuários. Nessa etapa, foram
empregadas técnicas qualitativas e quantitativas para avaliar a experiência dos
participantes e a eficácia da ferramenta como recurso educacional.

Por fim, realizou-se a validação dos resultados. Essa etapa envolveu o
cruzamento dos dados obtidos com os objetivos da pesquisa, a fim de verificar
se o jogo contribuiu de forma significativa para o ensino e a aprendizagem de
estruturas de dados e, sobretudo, se proporcionou uma experiência divertida e
motivadora aos participantes.

\fixme{adicionar paragrafo linkando com a metodologia de desenvolvimento ?}
