\subsection{Etapas da Pesquisa}

Esta pesquisa foi composta por diversas etapas e teve início com uma revisão
bibliográfica, por meio da qual foram identificadas publicações acadêmicas e
aplicações relacionadas ao uso de jogos sérios no ensino de conceitos de
programação. Essa etapa proporcionou uma visão das abordagens já utilizadas,
seus resultados e os conceitos mais frequentemente aplicados.

Em seguida, foi idealizado e desenvolvido um jogo sério educacional com base na
metodologia GAMED \cite{aslan2015gamed}. Esse processo envolveu diversas fases como a
concepção, implementação e aplicação do jogo proposto, seguido da coleta de
dados por meio de testes com usuários. Nessa etapa, foram empregadas técnicas
qualitativas e quantitativas para avaliar a experiência dos participantes e a
eficácia da ferramenta como recurso educacional.

Por fim, os resultados foram validados. Esta etapa consistiu no cruzamento dos
dados obtidos com os objetivos da pesquisa, com o propósito de verificar se o
jogo contribuiu de maneira significativa para o processo de ensino e
aprendizagem de estruturas de dados.
