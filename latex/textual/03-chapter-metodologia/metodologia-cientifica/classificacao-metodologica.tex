\subsection{Classificação Metodológica}

Este trabalho caracteriza-se como uma pesquisa aplicada, pois tem como objetivo
solucionar um problema prático relacionado ao ensino de estruturas de dados por
meio da utilização de um jogo sério \cite{mouaheb2012serious}. Diferentemente
da pesquisa puramente teórica, a pesquisa aplicada visa gerar conhecimento com
aplicação direta em contextos específicos. Neste caso, o foco está no ambiente
educacional.

A natureza experimental da pesquisa se deve ao fato de propor uma intervenção
concreta, que envolve o desenvolvimento e a aplicação de um protótipo funcional
de jogo em um ambiente controlado com usuários reais. O objetivo é observar os
efeitos dessa intervenção no processo de aprendizagem.

Adota-se uma abordagem mista, combinando métodos qualitativos e quantitativos
com o intuito de proporcionar uma análise mais abrangente e precisa dos
resultados. Os dados quantitativos são coletados por meio de instrumentos como
questionários estruturados e testes de desempenho, permitindo uma avaliação
objetiva. Já os dados qualitativos são obtidos por meio de observações,
entrevistas e análise do comportamento dos participantes durante a interação
com o jogo. Isso permite compreender de forma mais aprofundada a experiência
dos usuários e a eficácia da ferramenta no processo de ensino e aprendizagem.
