\newcounter{nfrcount}

\newcommand{\nfrvalue}{\ifnum\value{nfrcount}<10 0\fi\arabic{nfrcount}}
\newcommand{\nfrid}{NFR-\nfrvalue}

\newcommand{\nfr}[3]{
  \stepcounter{nfrcount}
  \begin{table}[H]
    \caption{Requisito Não Funcional \nfrvalue}
    \footnotesize
    \begin{tabular}{|p{3cm}|p{8cm}|p{3cm}|}
      \hline
      \rowcolor{headergray}
      \textbf{Identificador} & \textbf{Nome} & \textbf{Categoria} \\

      \hline
      \nfrid & {#1} & {#2} \\

      \hline
      \multicolumn{3}{|p{14cm}|}{\textbf{Descrição:}{#3}} \\
      \hline
    \end{tabular}
    \caption*{Fonte: Autor}
  \end{table}

  \vspace{1em}
}

\subsection{Requisitos Não-Funcionais} \label{sec:req_nao_func}

Os requisitos não-funcionais especificam características de qualidade e
restrições técnicas. Esses requisitos abrangem aspectos como performance,
compatibilidade, segurança, mantibilidade e acessibilidade.

\nfr{Frames por segundo estáveis}{Performance}
{
O sistema deve se manter em um taxa de atualização constante e acima de 30 frames por segundo.
}

\nfr{Múltiplos idiomas}{Usabilidade}
{
O sistema deve possuir tradução para o português e para o inglês.
}
