\newcounter{nfrcount}

\newcommand{\nfrvalue}{\ifnum\value{nfrcount}<10 0\fi\arabic{nfrcount}}
\newcommand{\nfrid}{NFR-\nfrvalue}

\newcommand{\nfr}[3]{
  \stepcounter{nfrcount}
  \begin{table}[H]
    \caption{Requisito Não Funcional \nfrvalue}
    \label{tab:requisito_nao_funcional_\nfrvalue}
    \centering
    \footnotesize
    \begin{tabular}{|p{3cm}|p{8cm}|p{3cm}|}
      \hline
      \rowcolor{headergray}
      \textbf{Identificador} & \textbf{Nome} & \textbf{Categoria} \\

      \hline
      \nfrid & {#1} & {#2} \\

      \hline
      \multicolumn{3}{|p{14cm}|}{\textbf{Descrição:}{#3}} \\
      \hline
    \end{tabular}
    \caption*{Fonte: Autor}
  \end{table}

  \vspace{1em}
}

\subsection{Requisitos Não-Funcionais} \label{sec:req_nao_func}

Os requisitos não-funcionais especificam características de qualidade e
restrições técnicas que o sistema deve atender para garantir boa experiência de
uso, desempenho adequado e facilidade de manutenção. Esses requisitos abrangem
aspectos como performance, compatibilidade, usabilidade, estabilidade,
acessibilidade e organização interna do código. As tabelas compreendidas entre
\ref{tab:requisito_nao_funcional_01} e \ref{tab:requisito_nao_funcional_13}
apresentam esses requisitos de forma estruturada, permitindo uma visão clara
das qualidades esperadas do sistema.

\nfr{Frames por segundo estáveis}{Performance}
{
O sistema deve manter uma taxa de atualização constante e acima de 30 frames
por segundo durante a jogabilidade.
}

\nfr{Tempo de carregamento reduzido}{Performance}
{
O sistema deve apresentar tempos de carregamento curtos ao trocar de fases ou
acessar menus.
}

\nfr{Alta responsividade}{Usabilidade}
{
O sistema deve responder de forma imediata às ações do jogador, garantindo
precisão na execução de comandos.
}

\nfr{Controles intuitivos}{Usabilidade}
{
O sistema deve oferecer controles claros e de fácil aprendizagem, favorecendo a
acessibilidade a novos jogadores.
}

\nfr{Acessibilidade}{Usabilidade}
{
O sistema deve fornecer opções que facilitem o acesso a jogadores com
necessidades específicas, incluindo personalização de controles.
}

\nfr{Tratamento de erros}{Confiabilidade}
{
O sistema deve tratar erros de forma adequada, exibindo mensagens informativas
e evitando encerramentos inesperados.
}

\nfr{Estabilidade}{Confiabilidade}
{
O sistema deve operar sem travamentos ou falhas críticas que comprometam o
andamento da partida.
}

\nfr{Persistência de dados}{Confiabilidade}
{
O sistema deve salvar o progresso do jogador de maneira consistente, evitando
perda de dados.
}

\nfr{Compatibilidade entre plataformas}{Compatibilidade}
{
O sistema deve ser compatível com os sistemas operacionais Linux e Windows.
}

\nfr{Expansão de fases}{Manutenibilidade}
{
O sistema deve permitir a fácil adição de novas fases por meio de estruturas
internas organizadas.
}

\nfr{Qualidade do código}{Manutenibilidade}
{
O código deve ser escrito de forma clara, modular e documentada, facilitando
manutenção e evolução futura.
}

\nfr{Estilo visual}{Estética}
{
O sistema deve manter um estilo visual coerente baseado em \emph{pixel art} 2D.
}

\nfr{Design sonoro}{Estética}
{
O sistema deve incluir efeitos sonoros e trilhas musicais adequadas ao tema e
às ações do jogador.
}
