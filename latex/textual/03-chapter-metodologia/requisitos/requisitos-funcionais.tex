\newcounter{frcount}

\newcommand{\frvalue}{\ifnum\value{frcount}<10 0\fi\arabic{frcount}}
\newcommand{\frid}{FR-\frvalue}

\newcommand{\fr}[3]{
  \stepcounter{frcount}

  \begin{table}[H]
    \caption{Requisito Funcional \frvalue}
    \label{tab:requisito_funcional_\frvalue}
    \centering
    \footnotesize
    \begin{tabular}{|p{3cm}|p{8cm}|p{3cm}|}
      \hline
      \rowcolor{headergray}
      \textbf{Identificador} & \textbf{Nome} & \textbf{Prioridade} \\

      \hline
      \frid & {#1} & {#2} \\

      \hline
      \multicolumn{3}{|p{14cm}|}{\textbf{Descrição:}{#3}} \\
      \hline
    \end{tabular}
    \caption*{Fonte: Autor}
  \end{table}

  \vspace{1em}
}

\subsection{Requisitos Funcionais} \label{sec:req_func}

Os requisitos funcionais descrevem as operações e comportamentos que o sistema
deve oferecer para atender aos objetivos propostos. Eles englobam aspectos como
manipulação de inventários, mecânicas de combate, progressão do jogador,
gerenciamento de estados do jogo e geração de \emph{feedback} educativo. As
tabelas compreendidas entre \ref{tab:requisito_funcional_01} e
\ref{tab:requisito_funcional_16} apresentam esses requisitos de forma
organizada, oferecendo uma visão objetiva e estruturada das funcionalidades
essenciais do sistema.

\fr{Ponto de controle}{Médio}
{
O sistema deve possuir um ponto de controle no meio de cada fase, onde o
usuário poderá retornar caso falhe.
}

\fr{Manipular inventários}{Alta}
{
O sistema deve permitir que os usuários manipulem seus inventários por meio de
ações específicas.
}

\fr{Movimentação do personagem}{Alta}
{
O sistema deve permitir que o personagem do jogador mova-se para a esquerda e
para a direita, pule, realize pulo duplo situacional e responda a parâmetros
físicos como gravidade, aceleração e altura de salto.
}

\fr{Progressão de fases}{Médio}
{
O sistema deve apresentar fases com ponto inicial e final claramente definidos,
permitindo ao jogador avançar progressivamente entre elas.
}

\fr{Obstáculos de fase}{Médio}
{
As fases devem conter obstáculos que dificultem o progresso do jogador, como
lacunas, objetos quebráveis e inimigos posicionados estrategicamente.
}

\fr{Inimigos com comportamento}{Alta}
{
O sistema deve incluir inimigos com comportamentos específicos, como patrulhar,
perseguir e atacar. Cada inimigo deve possuir pontos de vida e pode ser
derrotado pelo jogador.
}

\fr{Consumíveis}{Médio}
{
O sistema deve incluir consumíveis que proporcionem efeitos temporários, como
aumento de velocidade de movimento ou aceleração do processo de fabricação de
elementos.
}

\fr{Concluir fase}{Alta}
{
O sistema deve indicar claramente quando uma fase for concluída, por exemplo ao
alcançar o ponto final da área jogável.
}

\fr{Encerrar partida}{Alta}
{
O sistema deve encerrar a partida quando o jogador perder todos os pontos de
vida ou exceder o limite de tentativas.
}

\fr{Pontuação}{Médio}
{
O sistema deve registrar a pontuação do jogador com base no tempo de conclusão
da fase e outros critérios relevantes.
}

\fr{Menu principal}{Alta}
{
O sistema deve possuir um menu principal com opções para iniciar um novo jogo,
acessar as configurações e sair.
}

\fr{Menu de pausa}{Alta}
{
O sistema deve possuir um menu de pausa acessível durante a jogabilidade,
contendo opções para retomar a partida, voltar ao menu principal e acessar
configurações.
}

\fr{Configurações}{Médio}
{
O sistema deve permitir que o jogador ajuste configurações como volume,
dificuldade e esquema de controles.
}

\fr{Ranking}{Baixa}
{
O sistema deve armazenar e exibir pontuações máximas alcançadas pelos
jogadores.
}

\fr{Interface durante a partida}{Alta}
{
O sistema deve exibir na tela informações relevantes ao jogador, como
pontuação, vidas, saúde, mana e outros elementos essenciais.
}

\fr{\emph{Feedback} visual}{Alta}
{
O sistema deve fornecer \emph{feedback} visual para ações realizadas, como animações
de pulo, dano recebido, coleta de itens e interação com o ambiente.
}
