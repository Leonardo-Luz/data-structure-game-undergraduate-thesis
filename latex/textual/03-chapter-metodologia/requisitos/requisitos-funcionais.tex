\newcounter{frcount}
\newcommand{\frvalue}{\ifnum\value{frcount}<10 0\fi\arabic{frcount}}
\newcommand{\frid}{FR-\frvalue}

\newcommand{\fr}[3]{
  \stepcounter{frcount}

  \begin{table}[H]
    \caption{Requisito Funcional \frvalue}
    \begin{tabular}{|p{3cm}|p{8cm}|p{3cm}|}
      \hline
      \rowcolor{headergray}
      \textbf{Identificador} & \textbf{Nome} & \textbf{Prioridade} \\

      \hline
      \frid & {#1} & {#2} \\

      \hline
      \multicolumn{3}{|p{14cm}|}{\textbf{Descrição:}{#3}} \\
      \hline
    \end{tabular}
    \caption*{Fonte: Autor}
  \end{table}

  \vspace{1em}
}

\subsection{Requisitos Funcionais} \label{sec:req_func}

Os requisitos funcionais definem as funcionalidades que o sistema deve
implementar. Esses requisitos incluem aspectos como manipulação de inventários,
sistema de combate, progresso do jogador, salvamento de partida e geração de
feedback educacional.

\fr{Salvar progresso}{Alta}
{
O sistema deve permitir que os usuários salvem o progresso de suas partidas em multiplas instâncias.
}


\fr{Ponto de controle}{Médio}
{
O sistema deve possuir um ponto de controle no meio de cada fase, onde o usuário poderá retornar caso falhe.
}

\fr{Manipular inventários}{Alta}
{
O sistema deve permitir que os usuários manipulem seus inventários apartir ações.
}
