\subsection{Fase 2 - Projeto da Unidade Instrucional}

A segunda fase dedica-se ao design instrucional, definindo como os conteúdos
serão apresentados, como será realizada a avaliação e quais estratégias
pedagógicas serão empregadas.

\subsubsection*{A2.1 - Definição da Avaliação do Aprendiz}

A avaliação é incorporada diretamente nas mecânicas do jogo, funcionando como mecanismo de feedback contínuo:

\begin{itemize}
  \item \textbf{Avaliação Formativa Implícita}: ao realizar combinações corretas de elementos alquímicos, o jogador recebe feedback positivo instantâneo (sucesso na ação);
  \item \textbf{Avaliação por Penalidade}: combinações incorretas resultam em penalidade (perda de pontos de vida e feedback visual), estimulando o jogador a refletir sobre suas ações e ajustar sua estratégia;
  \item \textbf{Métricas de Desempenho}: ao final de cada fase, são registradas métricas como tempo de conclusão, tentativas, erros e acertos de combinações.
\end{itemize}

Essa abordagem de avaliação está alinhada ao construcionismo
\cite{papert1993children}, pois permite que o aprendiz construa seu
conhecimento através da exploração, tentativa e erro, em um ambiente seguro
onde o fracasso é parte do processo de descoberta.

\subsubsection*{A2.2 - Definição do Conteúdo e da Estratégia Instrucional}

O conteúdo é introduzido de forma gradual na primeira fase, seguindo um nível
de dificuldade crescente. Desde o início, o jogador tem acesso a todas as
funcionalidades do jogo; contudo, essas são apresentadas de maneira progressiva
ao longo dessa etapa inicial, permitindo uma assimilação natural dos conceitos
e mecânicas.

A estratégia instrucional priorizará o \textbf{aprendizado implícito}: os
conceitos não são apresentados verbalmente ou textualmente, mas descobertos
através da interação com as mecânicas do jogo. Dessa forma, a aprendizagem
emerge da necessidade prática de resolver desafios.

\subsubsection*{A2.3 - Decisão sobre Desenvolvimento ou Reutilização}

Optou-se pelo \textbf{desenvolvimento original do jogo}. Essa decisão baseou-se em:

\begin{itemize}
  \item Inexistência de jogos disponíveis que integrem a mecânica de estruturas de dados de forma implicita com temática de alquimia e estilo visual de pixel art inspirado em clássicos como Mario e Mega Man;
  \item Necessidade de controle total sobre a implementação das estruturas de dados para garantir precisão pedagógica;
  \item Oportunidade de customizar completamente a narrativa, visual e mecânicas para alinhar com os objetivos educacionais específicos;
\end{itemize}

\subsubsection*{A2.4 - Revisão do Modelo de Avaliação}

O modelo de avaliação do jogo segue os princípios do \textbf{MEEGA+} (Modelo
para Avaliação de Jogos Educacionais) de forma adaptada, abordando dimensões de:

\begin{itemize}
  \item \textbf{Usabilidade}: facilidade de aprender e usar o jogo, clareza da interface;
  \item \textbf{Engajamento}: capacidade de manter o interesse do jogador, imersão, diversão;
  \item \textbf{Aprendizagem Percebida}: percepção do jogador sobre o aprendizado obtido, aplicabilidade dos conceitos;
  \item \textbf{Experiência do Usuário}: satisfação geral, disposição para recomendação do jogo.
\end{itemize}

Os feedbacks são implementados de forma imediata no jogo, reforçando a aprendizagem e orientando o jogador através de:

\begin{itemize}
  \item Feedback visual (mudanças de cor, animações, efeitos particulares);
  \item Feedback sonoro (sons de sucesso ou erro);
\end{itemize}
