\subsection{Fase 2: Projeto da Unidade Instrucional} \label{subsec:fase_2}

A segunda fase dedica-se ao design instrucional, definindo como os conteúdos
serão apresentados, como será realizada a avaliação e quais estratégias
pedagógicas serão empregadas.

\subsubsection*{A2.1 - Definição da Avaliação do Aprendiz}

A avaliação é incorporada diretamente nas mecânicas do jogo, funcionando como mecanismo de feedback contínuo:

\begin{itemize}
  \item \textbf{Avaliação Formativa Implícita}: ao realizar combinações corretas de elementos alquímicos, o jogador recebe feedback positivo instantâneo (sucesso na ação);
  \item \textbf{Avaliação por Penalidade}: combinações incorretas resultam em penalidade (perda de pontos de vida e feedback visual), estimulando o jogador a refletir sobre suas ações e ajustar sua estratégia;
  \item \textbf{Métricas de Desempenho}: ao final de cada fase, são registradas métricas como tempo de conclusão, tentativas, erros e acertos de combinações.
\end{itemize}

Essa abordagem de avaliação está alinhada ao construcionismo
\cite{papert1993children}, pois permite que o aprendiz construa seu
conhecimento através da exploração, tentativa e erro, em um ambiente seguro
onde o fracasso é parte do processo de descoberta.

\subsubsection*{A2.2 – Definição do Conteúdo e da Estratégia Instrucional}

Nesta atividade foram definidos o conteúdo educacional e o seu sequenciamento
ao longo do jogo, considerando que a estratégia instrucional adotada é a
Aprendizagem Baseada em Jogos \cite{coffey2009digital}. Assim, os conceitos são
organizados de forma progressiva, acompanhando uma curva de complexidade
crescente e integrando-se diretamente às mecânicas propostas.

O conteúdo selecionado aborda operações fundamentais em estruturas de dados
lineares. A progressão instrucional foi estruturada da seguinte forma:

\begin{itemize}
    \item \textbf{\emph{Push}, \emph{Enqueue} e Inserção em Lista}: introdução do conceito de inserção no topo de uma pilha, no final de uma fila e em uma lista encadeada;
    \item \textbf{\emph{Pop}, \emph{Dequeue} e Remoção em Lista}: remoção do topo da pilha, do início da fila e de uma lista encadeada;
    \item \textbf{Iteração em lista}: percorrer e acessar os elementos de forma sequencial de uma lista encadeada;
\end{itemize}

O sequenciamento desses conteúdos foi planejado para que cada conceito seja
apresentado no momento em que sua aplicação se torna necessária para a
resolução de desafios do jogo. Dessa forma, a aprendizagem ocorre de maneira
contextualizada, permitindo ao jogador compreender os princípios das estruturas
de dados a partir da experimentação e da interação com as mecânicas. A
abordagem favorece a aprendizagem implícita, característica dos jogos
educacionais, na qual o domínio conceitual emerge naturalmente da prática
realizada durante o \emph{gameplay}.

\subsubsection*{A2.3 - Decisão sobre Desenvolvimento ou Reutilização}

Optou-se pelo \textbf{desenvolvimento original do jogo}. Essa decisão baseou-se em:

\begin{itemize}
  \item Inexistência de jogos disponíveis que integrem, de forma implícita, mecânicas de estruturas de dados com uma temática fantasiosa e estilo visual em \emph{pixel art}\footnote{técnica e estética gráfica baseada na construção de imagens pixel a pixel.}.
  \item Necessidade de controle total sobre a implementação das estruturas de dados para garantir precisão pedagógica;
  \item Oportunidade de customizar completamente a narrativa, visual e mecânicas para alinhar com os objetivos educacionais específicos;
\end{itemize}

\subsubsection*{A2.4 - Revisão do Modelo de Avaliação}

O modelo de avaliação adotado segue o \textbf{MEEGA+} (Modelo para Avaliação de
Jogos Educacionais) \cite{meega2020}, aplicado de forma adaptada ao contexto de jogos sérios. A
avaliação considera três dimensões fundamentais:

\begin{itemize}
  \item \textbf{Experiência do Usuário}: satisfação geral e disposição do jogador em recomendar o jogo.
  \item \textbf{Usabilidade}: clareza da interface, facilidade de uso e compreensão das ações disponíveis;
  \item \textbf{Conteúdo}: percepção do usuário quanto aos conceitos apresentados;
\end{itemize}

A descrição completa da metodologia de avaliação utilizada, incluindo o
instrumento adaptado do MEEGA+, encontra-se na
\autoref{sec:metodologia_avaliacao}.
