\subsection{Fase 1 - Análise da Unidade Instrucional}

A primeira fase concentra-se na análise do contexto educacional, da definição
do público-alvo e na especificação dos objetivos de aprendizagem que orientarão
todo o desenvolvimento.

\subsubsection*{A1.1 - Especificação da Unidade Instrucional}

A unidade instrucional foi definida como o ensino de conceitos fundamentais de
estruturas de dados (pilha, fila e lista) no contexto de cursos de graduação em
Análise e Desenvolvimento de Sistemas ou disciplinas correlatas de Computação.
O foco está em permitir que os aprendizes compreendam o funcionamento dessas
estruturas e suas operações básicas (inserção, remoção e ordenação) de forma
prática e contextualizada. Os objetivos de aprendizagem específicos incluem:
\begin{itemize}
  \item Compreender o funcionamento de estruturas de dados lineares (pilha, fila e lista);
  \item Identificar as diferenças entre operações LIFO (último a entrar, primeiro a sair) e FIFO (primeiro a entrar, primeiro a sair);
  \item Aplicar conceitos de manipulação de estruturas em cenários de resolução de problemas;
\end{itemize}

\subsubsection*{A1.2 - Caracterização dos Aprendizes}

O público-alvo deste estudo compreende estudantes de graduação em Análise e
Desenvolvimento de Sistemas ou cursos correlatos na área de Computação, com
idade entre 18 e 50 anos, que possuam acesso a computadores pessoais equipados
com sistema operacional capaz de executar navegadores modernos, e que utilizem
teclado e mouse como principais formas de interação.

Embora o jogo tenha sido projetado com esse público em mente, sua
acessibilidade permite que qualquer pessoa interessada em jogos de forma geral
possa participar, sem exigir conhecimento prévio explícito de estruturas de
dados.

\subsubsection*{A1.3 - Definição dos Objetivos de Desempenho}

O objetivo de desempenho do jogo é que, ao final da experiência, o jogador seja capaz de:

\begin{itemize}
  \item Executar operações de inserção e remoção respeitando as regras de cada estrutura;
  \item Entender e aplicar conceitos de ordenação;
  \item Reconhecer padrões de funcionamento das estruturas através da interação implícita com o jogo.
\end{itemize}
