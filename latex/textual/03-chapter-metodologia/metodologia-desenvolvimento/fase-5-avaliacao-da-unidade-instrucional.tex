\subsection{Fase 5 - Avaliação da Unidade Instrucional}

A quinta fase corresponderia à avaliação da unidade instrucional, cujo objetivo
é verificar a efetividade global do jogo enquanto ferramenta educacional,
analisando seu impacto na aprendizagem, na usabilidade e no engajamento dos
estudantes. Essa etapa demandaria a aplicação do jogo em um contexto formal de
ensino, com acompanhamento docente e coleta estruturada de evidências de
aprendizagem.

Assim como na fase anterior, essa avaliação educacional não pôde ser conduzida
dentro do escopo deste trabalho devido a limitações de tempo e à ausência de
aplicação em sala de aula. Sua execução permanece como trabalho futuro,
conforme apresentado na \autoref{sec:trabalhos_futuros}, visando permitir a
validação empírica do potencial pedagógico do jogo e o aprimoramento de suas
mecânicas instrucionais.

Entretanto, realizou-se uma avaliação complementar utilizando o modelo MEEGA+,
voltada à análise da experiência do usuário e da usabilidade do jogo no
contexto de entretenimento. Essa aplicação permitiu observar aspectos
relacionados ao engajamento, clareza da interface e percepção geral do jogador,
embora não abarque de forma completa os componentes pedagógicos envolvidos na
aprendizagem de Estrutura de Dados.
