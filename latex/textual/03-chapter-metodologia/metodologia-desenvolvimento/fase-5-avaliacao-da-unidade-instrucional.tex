\subsection{Fase 5 - Avaliação da Unidade Instrucional}

\fixme{A avaliação final mede a efetividade global do jogo como ferramenta educacional e não será feita no trabalho proposto.}

% \subsubsection*{Metodologia de Avaliação}
%
% Os dados coletados serão analisados de acordo com critérios baseados no MEEGA+:
%
% \begin{itemize}
%   \item \textbf{Usabilidade}: clareza da interface, facilidade de aprender, ausência de bugs que interfiram na experiência;
%
%   \item \textbf{Engajamento}: engajamento emocional, diversão, imersão, motivação intrínseca;
%
%   \item \textbf{Aprendizagem Percebida}: percepção do jogador sobre aprendizado obtido, aplicabilidade dos conceitos, confiança em usar estruturas de dados;
%
%   \item \textbf{Experiência Geral}: satisfação, disposição para recomendação, intenção de uso futuro.
% \end{itemize}
%
% \subsubsection*{Análise de Resultados}
%
% Os resultados serão analisados através de:
%
% \begin{itemize}
%   \item \textbf{Análise Quantitativa}: estatísticas descritivas das métricas de desempenho, correlações entre tempo de jogo e compreensão de conceitos;
%
%   \item \textbf{Análise Qualitativa}: análise temática de respostas abertas, observações de comportamento, entrevistas;
%
%   \item \textbf{Triangulação}: cruzamento de dados quantitativos e qualitativos para validação de resultados.
% \end{itemize}
%
% Os resultados serão cotejados com os objetivos de aprendizagem definidos na
% Fase 1, verificando se o jogo contribuiu significativamente para o processo de
% ensino e aprendizagem de estruturas de dados.
%
% \subsubsection*{Iterações e Melhorias}
%
% Com base nos resultados da avaliação, serão identificadas oportunidades de melhoria:
%
% \begin{itemize}
%   \item Ajustes nas mecânicas e balanceamento de dificuldade;
%   \item Refinamentos na narrativa e imersão;
%   \item Otimizações de performance e acessibilidade;
%   \item Expansões futuras (novos tipos de estruturas, fases adicionais, modos de jogo).
% \end{itemize}
