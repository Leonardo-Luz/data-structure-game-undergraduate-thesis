\subsection{Fase 3 - Desenvolvimento do Jogo Educacional} \label{sec:fase_5_desenvolvimento}

Esta fase abrange as etapas práticas de implementação: levantamento de
requisitos, concepção visual e narrativa, design de mecânicas e implementação
técnica. Esta etapa será mais aprofundada no \autoref{cap:desenvolvimento}

% \subsubsection*{A3.1 - Análise do Jogo e Levantamento de Requisitos}
%
% Os requisitos foram estruturados em duas categorias: funcionais e não-funcionais.
%
% \paragraph*{Requisitos Funcionais}
%
% Os requisitos funcionais definem as funcionalidades que o sistema deve
% implementar. Esses requisitos estão detalhados no \autoref{apd:rf}
% e incluem aspectos como manipulação de inventários, sistema de combate,
% progresso do jogador, salvamento de partida e geração de feedback educacional.
%
% \paragraph*{Requisitos Não-Funcionais}
%
% Os requisitos não-funcionais especificam características de qualidade e
% restrições técnicas. Esses requisitos estão detalhados no \autoref{apd:nrf}
%  e abrangem aspectos como performance, compatibilidade, segurança,
% mantibilidade e acessibilidade.
%
% \subsubsection*{A3.2 - Concepção do Jogo}
%
% \paragraph*{Gênero e Perspectiva}
%
% O jogo é um \textbf{platformer 2D} (jogo de plataforma em duas dimensões) com
% perspectiva \textit{side-scrolling}. Essa escolha baseia-se na familiaridade do
% público geral com clássicos como Super Mario Bros e Mega Man, facilitando a
% adoção do jogo e tornando a experiência intuitiva.
%
% \paragraph*{Temática e Narrativa}
%
% A narrativa é ambientada em um universo de \textbf{alquimia} mística e
% medieval. O protagonista é um \textit{plague doctor} (doutor da peste) que teve
% sua pesquisa sobre a \textbf{pedra filosofal} roubada por um alquimista rival
% enquanto viajava para a capital para apresentá-la. Seu objetivo é recuperar a
% pesquisa, navegando por diferentes regiões e enfrentando adversários durante a
% sua jornada.
%
% Essa narrativa fornece contexto e motivação para as ações do jogador,
% aumentando o engajamento e criando imersão. O enredo também justifica a
% presença dos elementos alquímicos como mecanismo central de combate.
%
% \paragraph*{Mecânicas Centrais}
%
% As mecânicas principais do jogo são:
%
% \begin{itemize}
%   \item \textbf{Movimentação e Exploração}: o jogador controla o plague doctor, movendo-se horizontalmente pelos níveis, pulando e interagindo com o ambiente;
%
%   \item \textbf{Sistema de Inventários Estruturados}: o jogador possui três inventários independentes (pilha, fila e lista), cada um representando uma estrutura de dados específica. Os elementos alquímicos (fogo, água, ar, terra) são armazenados nesses inventários;
%
%   \item \textbf{Combinação de Elementos}: para atacar, o jogador deve remover elementos dos inventários e combiná-los corretamente. Por exemplo:
%   \begin{itemize}
%     \item Dois elementos fogo combinados geram um ataque de fogo;
%     \item Dois elementos água combinados geram um ataque de água;
%     \item A combinação respeitará as regras da estrutura de origem (LIFO para pilha, FIFO para fila, acesso livre para lista).
%   \end{itemize}
%
%   \item \textbf{Vulnerabilidades de Inimigos}: cada inimigo possui vulnerabilidades específicas a um ou mais elementos. Ataques com o elemento correto causam dano;
%
%   \item \textbf{Consumíveis}: itens especiais encontrados durante o jogo auxiliam na progressão:
%   \begin{itemize}
%     \item \textbf{Ordenação}: ordena a estrutura de dados, permitindo reorganizar elementos;
%     \item \textbf{Insert}: insere dois elementos iguais em um inventário específico;
%     \item \textbf{Remove}: remove um elemento de um inventário específico;
%     \item \textbf{Mana}: concede mana infinita por um tempo determinado;
%     \item \textbf{Cura}: cura 1 ponto de vida;
%     \item \textbf{Vida Extra}: aumenta a quantidade de tentativas em 1.
%   \end{itemize}
% \end{itemize}
%
% \subsubsection*{A3.3 - Design do Jogo}
%
% \paragraph*{Elementos Visuais e Artísticos}
%
% O jogo utiliza \textit{pixel art} como estilo visual, em homenagem aos
% clássicos de 8 e 16 bits, facilitando a conexão com o público-alvo e permitindo
% desenvolvimento mais ágil. Os sprites do protagonista, inimigos, objetos e
% cenários foram criados em pixel art original, mantendo coerência visual e
% legibilidade.
%
% A paleta de cores foi escolhida para refletir a temática do personagem
% principal, seguindo majoritariamente tons de azul e roxo, tanto mais escuros
% quanto pastéis.
%
% \paragraph*{Interface e Comunicação Visual}
%
% A interface foi projetada com princípios de \textbf{minimalismo} e
% \textbf{clareza}, evitando sobrecarga visual:
%
% \begin{itemize}
%   \item \textbf{Inventário Visual}: os inventários são exibidos na parte inferior à direita, mostrando os elementos armazenados em cada estrutura através de representações visuais claras;
%
%   \item \textbf{Feedback Visual de Erros}: quando uma combinação incorreta é realizada, o jogo exibe feedback visual (mudança de cor do jogador e efeito de penalidade) sem interromper a fluidez do \textit{gameplay};
%
%   \item \textbf{Indicadores de Vulnerabilidade}: inimigos exibem visualmente seus elementos de vulnerabilidade através de ícones, permitindo que o jogador identifique rapidamente a estratégia apropriada.
% \end{itemize}
%
% \paragraph*{Sistema de Pontuação e Progressão}
%
% A pontuação é baseada em:
%
% \begin{itemize}
%   \item \textbf{Eficiência}: quantos ataques bem-sucedidos foram realizados em relação ao total;
%   \item \textbf{Tempo}: bônus por completar a fase rapidamente;
%   \item \textbf{Sem Dano}: bônus adicional se o jogador completar a fase sem perder vidas.
% \end{itemize}
%
% O progresso é salvo automaticamente em pontos de controle (checkpoints)
% estrategicamente posicionados em cada fase, evitando frustração causada por
% perda de progresso.
%
% \subsubsection*{A3.4 - Implementação Técnica}
%
% \paragraph*{Ferramentas e Tecnologias}
%
% \begin{itemize}
%   \item \textbf{Motor de Jogo}: Unity 2022 LTS, selecionada por sua robustez, suporte multiplataforma e capacidade de criar jogos de performance eficiente;
%
%   \item \textbf{Linguagem de Programação}: C\#, linguagem nativa da Unity, utilizada para implementar toda a lógica do jogo, incluindo estruturas de dados e sistemas de \textit{gameplay};
%
%   \item \textbf{Criação de Assets Visuais}: GIMP (GNU Image Manipulation Program), software livre utilizado para criar e editar sprites em pixel art;
%
%   \item \textbf{Controle de Versão}: Git com repositório remoto, garantindo rastreamento de alterações e possibilidade de recuperação de versões anteriores.
% \end{itemize}
%
% \paragraph*{Arquitetura de Código}
%
% As estruturas de dados (pilha, fila e lista) foram implementadas como classes
% independentes em C\#, permitindo:
%
% \begin{itemize}
%   \item Reaproveitamento de código entre diferentes contextos do jogo;
%   \item Clareza didática: o código das estruturas reflete diretamente os conceitos ensinados;
%   \item Testes unitários simplificados para validar o comportamento das estruturas.
% \end{itemize}
%
% Cada classe implementa operações fundamentais:
%
% \begin{itemize}
%   \item \textbf{Stack (Pilha)}: \texttt{Push()}, \texttt{Pop()}, \texttt{Peek()}. Respeitando a ordem LIFO.
%
%   \item \textbf{Queue (Fila)}: \texttt{Enqueue()}, \texttt{Dequeue()}, \texttt{Peek()}. Respeitando a ordem FIFO.
%
%   \item \textbf{List (Lista)}: \texttt{Add()}, \texttt{Insert()}, \texttt{RemoveAt()}, \texttt{Get()}. Permitindo acesso livre em qualquer posição.
% \end{itemize}
%
% \paragraph*{Sistemas de Jogo Implementados}
%
% \begin{itemize}
%   \item \textbf{Sistema de Movimento}: controle do personagem (esquerda, direita, pulo), colisão com plataformas e inimigos;
%
%   \item \textbf{Sistema de Combate}: detecção de combinações de elementos, validação contra regras de estruturas, cálculo de dano;
%
%   \item \textbf{Sistema de Inventário}: gerenciamento de elementos nos inventários, suporte a operações específicas de cada estrutura;
%
%   \item \textbf{Sistema de Vida}: rastreamento da vida do jogador, aplicação de penalidades, condição de derrota;
%
%   \item \textbf{Sistema de Progresso}: salvamento e carregamento de estado, gestão de fases desbloqueadas, registro de pontuação;
%
%   \item \textbf{Sistema de Feedback}: implementação de mensagens visuais, sonoras e textuais que reforçam o aprendizado.
% \end{itemize}
%
% \subsubsection*{A3.5 - Testes do Jogo}
%
% Os testes foram conduzidos em múltiplas etapas:
%
% \paragraph*{Testes Funcionais}
%
% Testes para validar se todas as funcionalidades implementadas operam conforme
% especificado:
%
% \begin{itemize}
%   \item Operações de estruturas de dados (inserção, remoção, ordenação);
%   \item Lógica de combate e detecção de vulnerabilidades;
%   \item Sistema de salvamento e carregamento;
%   \item Interface e feedback visual.
% \end{itemize}
%
% \paragraph*{Testes Educacionais}
%
% Testes com usuários reais para validar a eficácia pedagógica:
%
% \begin{itemize}
%   \item \textbf{Participantes}: estudantes do curso de Análise e Desenvolvimento de Sistemas (conhecimento prévio de programação), entusiastas de jogos (sem conhecimento específico de estruturas de dados) e pessoas leigas (sem conhecimento técnico);
%
%   \item \textbf{Instrumentos}: observação direta, questionários pós-jogo, análise de métricas de desempenho (tempo, tentativas, acertos);
%
%   \item \textbf{Focos de Avaliação}: compreensão implícita dos conceitos, engajamento, diversão, clareza das mecânicas, dificuldade progressiva.
% \end{itemize}
%
% \paragraph*{Resultados Preliminares}
%
% Os testes iniciais indicaram:
%
% \begin{itemize}
%   \item Boa compreensão dos mecanismos de estruturas de dados mesmo entre participantes sem conhecimento prévio;
%   \item Engajamento sustentado com a temática alquímica e narrativa do \textit{plague doctor};
%   \item Feedback visual e mecânicas intuitivas facilitando a adoção;
%   \item Necessidade de ajustes na dificuldade de certas fases e clareza de alguns consumíveis.
% \end{itemize}
