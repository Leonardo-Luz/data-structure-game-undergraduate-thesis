\subsection{Fase 4 - Execução da Unidade Instrucional}

\fixme{Na quarta fase, o jogo seria integrado em contextos educacionais reais, sendo utilizado em sala de aula como ferramenta complementar de ensino. Entretanto, por falta de tempo, este será proposto como trabalhos futuros.}

% \subsubsection*{Implementação em Contexto Educacional}
%
% O jogo será disponibilizado como atividade complementar em disciplinas de
% Estruturas de Dados de cursos de Computação. Os docentes poderão:
%
% \begin{itemize}
%   \item Integrar o jogo como revisão ou pré-aprendizado de conceitos;
%   \item Observar o desempenho dos alunos através das métricas de progresso registradas pelo jogo;
%   \item Utilizar os resultados como indicadores de compreensão e dificuldades específicas;
%   \item Orientar discussões em sala sobre os conceitos descobertos implicitamente durante o jogo.
% \end{itemize}
%
% \subsubsection*{Coleta de Dados}
%
% Durante a execução, serão coletados:
%
% \begin{itemize}
%   \item Métricas de jogo: tempo de conclusão, tentativas, acertos de combinação;
%   \item Feedback dos participantes: satisfação, percepção de aprendizado, dificuldades encontradas;
%   \item Observações qualitativas: comportamento durante o jogo, discussões com pares, reações emocionais.
% \end{itemize}
