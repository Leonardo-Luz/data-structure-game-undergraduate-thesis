\section{Metodologia de Desenvolvimento}

Para o desenvolvimento do jogo sério \cite{mouaheb2012serious}, será empregada,
de forma adptada, a metodologia GAMED, desenvolvida especificamente para
orientar o processo de criação de jogos digitais educacionais.

\fixme{
  A principal meta de um projeto que visa desenvolver um jogo educacional (DEG) é
  produzir uma aplicação com qualidades desejaveis como: desafiador, engajável,
  prazeroso, interativo, transformativo, e usabilidade. \cite{aslan2015gamed}
  Para chegar a tal resultado esta metodologia...
}

Esta metodologia utiliza \emph{DEG life cycle} como sua fundação,
a qual possui 4 fases principais e diversos processos, sendo estes:

\begin{enumerate}
  \item \textbf{Fase do Design do Jogo} (\emph{Game Design Phase})
  \begin{itemize}
    \item Formulação do Problema \\
      O processo de formulação do problema é composto por diversas etapas, que
      incluem: a identificação do problema educacional, a análise da relevância do
      problema identificado, a investigação do estado da arte relacionado ao tema, e
      a justificativa para o uso da aprendizagem baseada em jogos como abordagem
      pedagógica alternativa. Também faz parte desse processo a definição clara do
      problema educacional e a especificação dos objetivos que se pretende alcançar
      com a utilização de jogos sérios. Ao final dessas etapas, é elaborado o
      documento denominado \emph{Education Problem Specification}.
    \item Geração da ideia do jogo \\
      Neste processo é gerado o documento \emph{Game Idea Specification}.
    \item Design do Jogo
  \end{itemize}
  \item \textbf{Fase do Design do Aplicativo} (\emph{Game Software Design Phase})
  \begin{itemize}
    \item Levantamento de Requisitos
    \item Arquitetação
    \item Design do Aplicativo
  \end{itemize}
  \item \textbf{Fase da Implementação e Publicação do Jogo} (\emph{Game Implementation and Publishing Phase})
  \begin{itemize}
    \item Desenvolvimento
    \item Integração
    \item Publicação
  \end{itemize}
  \item \textbf{Fase da Aprendizagem Baseada em Jogo e \emph{Feedback}} (\emph{Game-based Learning and Feedback Phase})
  \begin{itemize}
    \item Aprendizagem Baseada em Jogo
    \item \emph{Feedback}
  \end{itemize}
\end{enumerate}

% \newcounter{frcount}
\newcommand{\frvalue}{\ifnum\value{frcount}<10 0\fi\arabic{frcount}}
\newcommand{\frid}{FR-\frvalue}

\newcommand{\fr}[3]{
  \stepcounter{frcount}

  \begin{table}[H]
    \caption{Requisito Funcional \frvalue}
    \begin{tabular}{|p{3cm}|p{8cm}|p{3cm}|}
      \hline
      \rowcolor{headergray}
      \textbf{Identificador} & \textbf{Nome} & \textbf{Prioridade} \\

      \hline
      \frid & {#1} & {#2} \\

      \hline
      \multicolumn{3}{|p{14cm}|}{\textbf{Descrição:}{#3}} \\
      \hline
    \end{tabular}
  \end{table}

  \vspace{1em}
}

\section{Requisitos Funcionais}

\fr{Salvar Progresso}{Alta}
{
O sistema deve permitir que os usuários salvem o progresso de suas partidas em multiplas instâncias.
}

% \newcounter{nfrcount}

\newcommand{\nfrvalue}{\ifnum\value{nfrcount}<10 0\fi\arabic{nfrcount}}
\newcommand{\nfrid}{NFR-\nfrvalue}

\newcommand{\nfr}[3]{
  \stepcounter{nfrcount}
  \begin{table}[H]
    \caption{Requisito Não Funcionail \nfrvalue}
    \begin{tabular}{|p{3cm}|p{8cm}|p{3cm}|}
      \hline
      \rowcolor{headergray}
      \textbf{Identificador} & \textbf{Nome} & \textbf{Categoria} \\

      \hline
      \nfrid & {#1} & {#2} \\

      \hline
      \multicolumn{3}{|p{14cm}|}{\textbf{Descrição:}{#3}} \\
      \hline
    \end{tabular}
  \end{table}

  \vspace{1em}
}

\section{Requisitos não funcionais}

\nfr{Frames por segundo estáveis}{Performance}
{
O sistema deve se manter em um taxa de atualização constante.
}


