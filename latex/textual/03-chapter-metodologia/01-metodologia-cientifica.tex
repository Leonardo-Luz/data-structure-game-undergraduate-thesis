\section{Metodologia Científica}

A metodologia científica adotada neste trabalho é de natureza experimental, com abordagem mista, qualitativa e quantitativa, e classifica-se como uma pesquisa aplicada.

\subsection{Classificação Metodológica}

Este trabalho caracteriza-se como uma \textbf{pesquisa aplicada}, pois visa solucionar um problema prático relacionado ao ensino de estruturas de dados por meio da utilização de um jogo sério \cite{mouaheb2012serious}. Diferentemente da pesquisa puramente teórica, a pesquisa aplicada busca produzir conhecimento com aplicação direta em contextos específicos, neste caso, no ambiente educacional.

A \textbf{natureza experimental} da pesquisa se dá pelo fato de que se propõe a testar uma intervenção concreta, o desenvolvimento e a aplicação de um protótipo funcional de jogo, em um ambiente controlado com usuários reais, a fim de observar os efeitos dessa intervenção no processo de aprendizagem.

Adota-se uma \textbf{abordagem mista}, combinando métodos \textbf{qualitativos e quantitativos}, com o intuito de obter uma análise mais ampla e precisa dos resultados. Os dados quantitativos são obtidos por meio de instrumentos como questionários estruturados e testes de desempenho, permitindo uma avaliação objetiva dos resultados. Já os dados qualitativos são obtidos por meio de observações, entrevistas e análise do comportamento dos participantes durante a interação com o jogo, fornecendo uma compreensão mais aprofundada da experiência dos usuários e da eficácia da ferramenta no processo de ensino-aprendizagem.

\subsection{Etapas da Pesquisa}

A condução da pesquisa foi estruturada em três etapas principais:

\begin{enumerate}
  \item \textbf{Revisão bibliográfica} \\
    Consiste no levantamento e análise de publicações acadêmicas, projetos e aplicações existentes relacionados ao ensino de programação, com ênfase em estruturas de dados, utilizando jogos sérios como ferramenta didática;

  \item \textbf{Desenvolvimento e análise} \\
    Engloba o processo de concepção, implementação e aplicação do jogo proposto, seguido da coleta de dados por meio de testes com usuários. Nesta etapa, são aplicadas técnicas qualitativas e quantitativas para avaliar a experiência dos participantes e a eficácia da ferramenta como recurso educacional;

  \item \textbf{Validação dos resultados} \\
    Envolve o cruzamento dos dados obtidos com os objetivos da pesquisa, a fim de verificar se o jogo contribui de maneira significativa para o processo de ensino-aprendizagem de estruturas de dados.
\end{enumerate}
