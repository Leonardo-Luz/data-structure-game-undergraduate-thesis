\section{Metodologia Científica}

A metodologia científica adotada neste trabalho é de natureza exploratória, com
abordagem mista, qualitativa e quantitativa, e classifica-se como uma pesquisa
aplicada. A próxima seção detalha e justifica essas escolhas.

\subsection{Classificação Metodológica}

Este trabalho caracteriza-se como uma pesquisa aplicada. Segundo
\citeonline{creswell2021projeto}, a pesquisa aplicada tem como objetivo gerar
conhecimentos para aplicação prática, dirigidos à solução de problemas
específicos. Neste contexto, o foco está na solução de dificuldades de
aprendizagem na disciplina de Estruturas de Dados \cite{mtaho2024difficulties}
por meio do desenvolvimento e avaliação de um jogo sério \cite{mouaheb2012serious}.

A natureza da pesquisa é classificada como exploratória, uma vez que se propõe
investigar uma abordagem não tradicional e relativamente inexplorada no
contexto educacional: o ensino implícito de estruturas de dados através de um jogo sério.
Pesquisas exploratórias visam proporcionar maior familiaridade com um problema
, permitindo aprofundar a compreensão sobre novas perspectivas e
práticas pedagógicas \cite{creswell2021projeto}.
Neste trabalho, o objetivo é verificar a viabilidade e a aceitação inicial da
ferramenta proposta junto a usuários reais, coletando evidências que
possibilitem a formulação de hipóteses para investigações futuras.
% A pesquisa se caracteriza por uma abordagem aberta ao contexto real de uso, sem
% manipulação rigorosa de variáveis ou a imposição de grupos de controle, o que é
% adequado para a fase inicial de validação de uma proposta não tradicional no
% campo de ensino de Estruturas de Dados.

A abordagem do problema é mista, combinando métodos qualitativos e
quantitativos para proporcionar uma análise abrangente. A abordagem
quantitativa foi realizada por meio da coleta de dados estruturados utilizando
o questionário MEEGA+ \cite{meega2020}, adaptado conforme necessário,
permitindo a mensuração objetiva de fatores como usabilidade, conteúdos
abordados e experiência do jogador através de escalas Likert \cite{likert1932}.
Já os dados qualitativos foram obtidos por meio da análise de perguntas abertas
presentes no formulário de avaliação, permitindo captar percepções subjetivas,
sugestões e críticas detalhadas dos participantes que não seriam identificadas
apenas pelos dados numéricos \cite{creswell2021projeto}.

\subsection{Etapas da Pesquisa}

Esta pesquisa foi composta por diversas etapas e teve início com uma revisão
bibliográfica, por meio da qual foram identificadas publicações acadêmicas e
aplicações relacionadas ao uso de jogos sérios no ensino de conceitos de
programação. Essa etapa proporcionou uma visão das abordagens já utilizadas,
seus resultados e os conceitos mais frequentemente aplicados.

Em seguida, foi idealizado e desenvolvido um jogo sério educacional com base na
metodologia GAMED \cite{aslan2015gamed}. Esse processo envolveu diversas fases como a
concepção, implementação e aplicação do jogo proposto, seguido da coleta de
dados por meio de testes com usuários. Nessa etapa, foram empregadas técnicas
qualitativas e quantitativas para avaliar a experiência dos participantes e a
eficácia da ferramenta como recurso educacional.

Por fim, os resultados foram validados. Esta etapa consistiu no cruzamento dos
dados obtidos com os objetivos da pesquisa, com o propósito de verificar se o
jogo contribuiu de maneira significativa para o processo de ensino e
aprendizagem de estruturas de dados.


% \fixme{
%   A metodologia científica adotada neste trabalho é de natureza experimental, com
%   abordagem mista, qualitativa e quantitativa, e classifica-se como uma pesquisa
%   aplicada. A próxima seção detalha e justifica essas escolhas.
% }
%
% \subsection{Classificação Metodológica}

Este trabalho caracteriza-se como uma pesquisa aplicada, pois tem como objetivo
solucionar um problema prático relacionado ao ensino de estruturas de dados por
meio da utilização de um jogo sério. Diferentemente da pesquisa puramente
teórica, a pesquisa aplicada visa gerar conhecimento com aplicação direta em
contextos específicos. Neste caso, o foco está no ambiente educacional.

\fixme{
  A natureza experimental da pesquisa se deve ao fato de propor uma intervenção
  concreta, que envolve o desenvolvimento e a aplicação de um protótipo funcional
  de jogo em um ambiente controlado com usuários reais. O objetivo é observar os
  efeitos dessa intervenção no processo de aprendizagem.
}

Adota-se uma abordagem mista, combinando métodos qualitativos e quantitativos
com o intuito de proporcionar uma análise mais abrangente e precisa dos
resultados. Os dados quantitativos são coletados por meio de instrumentos como
questionários estruturados, permitindo uma avaliação objetiva. \fixme{ Já os dados
qualitativos são obtidos por meio de observações, entrevistas e análise do
comportamento dos participantes durante a interação com o jogo. } Isso permite
compreender de forma mais aprofundada a experiência dos usuários e a eficácia
da ferramenta no processo de ensino e aprendizagem.

