\section{Metodologia Científica}

A abordagem metodológica científica adotada neste trabalho é de natureza experimental, com enfoque qualitativo e quantitativo, sendo caracterizada como uma pesquisa aplicada.

\subsection{Etapas da Pesquisa}

A condução da pesquisa seguirá as seguintes etapas:

\begin{enumerate}
  \item \textbf{Revisão bibliográfica}: levantamento de publicações e aplicativos sobre ensino de conceitos de programação, com foco em estruturas de dados, utilizando jogos sérios;
  \item \textbf{Desenvolvimento do protótipo}: aplicação da metodologia GAMED para o desenvolvimento de um jogo funcional que incorpore, de forma implícita, conceitos de estruturas de dados;
  \item \textbf{Aplicação do protótipo}: realização de testes com usuários (estudantes de cursos de computação), em ambiente controlado, com acompanhamento e coleta de dados;
  \item \textbf{Instrumentos de coleta de dados}: utilização de questionários para avaliar usabilidade, engajamento e aprendizagem;
  \item \textbf{Análise dos dados}: aplicação de métodos qualitativos e quantitativos para avaliar a eficácia do jogo enquanto ferramenta pedagógica;
  \item \textbf{Validação}: cruzamento dos dados coletados com os objetivos propostos para verificar se a proposta contribui de forma significativa para o processo de ensino-aprendizagem.
\end{enumerate}
