\section{Metodologia de Avaliação} \label{sec:metodologia_avaliacao}

A avaliação do jogo proposto será conduzida com base no modelo MEEGA+,
desenvolvido por \citeonline{meega2020}, utilizado aqui de forma adaptada para
o contexto de jogos sérios. O modelo fornece um instrumento validado para
avaliar a experiência do jogador e a usabilidade, dimensões que compõem os
fatores de qualidade estabelecidos pelos autores após análise fatorial com 843
participantes avaliando 18 jogos diferentes, garantindo consistência interna e
validade de construto.

\subsection{Participantes e Contexto}

A avaliação foi disponibilizada para estudantes da área de Computação,
especialmente aqueles matriculados em disciplinas introdutórias relacionadas a
estruturas de dados. A participação foi voluntária, e os jogadores foram
convidados a experimentar o protótipo desenvolvido, podendo avançar no jogo até
onde fossem capazes.

Além dos estudantes, qualquer pessoa com acesso ao jogo pôde avaliá-lo,
considerando que o título utiliza ensino implícito e não requer conhecimento
prévio sobre estruturas de dados. Assim, a coleta buscou alcançar tanto o
público-alvo acadêmico quanto usuários gerais, ampliando a diversidade das
percepções obtidas.

\subsection{Procedimentos de Coleta de Dados}

Ao término da sessão de jogo, os participantes responderam a um 
questionário baseado no MEEGA+, composto por afirmações avaliadas em escala 
Likert de 5 pontos \cite{likert1932}. O instrumento coleta dados referentes às seguintes 
dimensões:

\begin{itemize}
    \item \textbf{Experiência do Jogador}: imersão, diversão, engajamento e
      desafio percebido;
    \item \textbf{Usabilidade}: clareza das instruções, facilidade de uso,
      coerência interface-jogo e percepção geral de navegabilidade.
\end{itemize}

Também foi incluída uma pergunta aberta para coleta de feedback qualitativo, 
permitindo identificar sugestões, dificuldades e percepções não capturadas 
pelos itens fechados.

\subsection{Instrumento de Avaliação MEEGA+}

O instrumento contém itens estruturados em duas dimensões principais:

\begin{enumerate}
    \item \textbf{Experiência do Jogador}: mede engajamento, imersão,
      motivação, concentração e percepção de desafio;
    \item \textbf{Usabilidade}: avalia clareza, intuitividade, facilidade de
      interação e compreensão da interface.
\end{enumerate}

Embora grande parte dos itens siga a formulação original proposta por 
\citeonline{meega2020}, algumas afirmações foram adaptadas, removidas ou 
acrescentadas para adequar o instrumento ao contexto de um jogo sério voltado 
ao ensino implícito de conceitos de Estrutura de Dados. A 
\autoref{tab:perguntas_avaliacao} apresenta todas as adaptações realizadas, 
bem como as perguntas efetivamente aplicadas aos participantes.

{\footnotesize
\begin{longtable}{|
    >{\centering\arraybackslash}p{2cm}|
    >{\centering\arraybackslash}p{3cm}|
    >{\centering\arraybackslash}p{1cm}|
    >{\raggedright\arraybackslash}p{6cm}|
}
\caption{Itens do questionário do modelo MEEGA+ adaptados} \label{tab:perguntas_avaliacao} \\

\hline
\rowcolor{headergray}
\textbf{Dimensão} & \textbf{Subdimensão} & \textbf{Item} & \textbf{Descrição do Item} \\
\hline
\endfirsthead

\hline
\rowcolor{headergray}
\textbf{Dimensão} & \textbf{Subdimensão} & \textbf{Item} & \textbf{Descrição do Item} \\
\hline
\endhead

\multicolumn{4}{|r|}{\textit{Continua na próxima página}} \\
\hline
\endfoot

\hline
\caption*{Fonte: \cite{meega2020}, editado por Autor}
\endlastfoot

\multirow{10}{2cm}{\centering\textbf{Experiência} \textbf{do Jogador}}
  & Engajamento       & 1 & O jogo conseguiu manter o meu interesse. \\ \cline{2-4}
  & Engajamento       & 2 & Eu estava motivado para continuar jogando. \\ \cline{2-4}
  & Engajamento       & 3 & O jogo apresentou estímulos que mantiveram a minha atenção. \\ \cline{2-4}
  & Diversão          & 4 & Jogar este jogo foi divertido. \\ \cline{2-4}
  & Satisfação        & 5 & Senti satisfação ao completar desafios ou derrotar inimigos. \\ \cline{2-4}
  & Satisfação        & 6 & Eu recomendaria este jogo para meus colegas. \\ \cline{2-4}
  & Desafio           & 7 & O nível de desafio foi adequado para mim. \\ \cline{2-4}
  & Desafio           & 8 & Os desafios me incentivaram a tentar melhorar. \\ \cline{2-4}
  & Imersão    & 9 & Eu estava tão envolvido no jogo que perdi a noção do tempo. \\ \cline{2-4}
  & Imersão    & 10 & Eu esqueci do ambiente ao meu redor enquanto jogava. \\
\hline

\multirow{8}{2cm}{\centering\textbf{Usabilidade}}
  & Estética                          & 11 & O design do jogo é atraente. \\ \cline{2-4}
  & Estética                          & 12 & Os textos, cores e fontes combinam e são consistentes. \\ \cline{2-4}
  & Aprendibilidade                   & 13 & Aprender a jogar este jogo foi fácil para mim. \\ \cline{2-4}
  & Operabilidade                     & 14 & Eu considero que o jogo é fácil de jogar. \\ \cline{2-4}
  & Operabilidade                     & 15 & As regras do jogo são claras e compreensíveis. \\ \cline{2-4}
  & Acessibilidade                    & 16 & As fontes (tamanho e estilo) utilizadas no jogo são legíveis. \\ \cline{2-4}
  & Acessibilidade                    & 17 & As cores utilizadas no jogo são compreensíveis. \\ \cline{2-4}
  & Proteção Contra Erros do Usuário  & 18 & Quando eu cometo um erro é fácil de me recuperar rapidamente. \\
\hline

\multirow{8}{2cm}{\centering\textbf{Conteúdo}}
  & Relevância & 19 & Os conceitos de estruturas de dados (pilha, fila e lista encadeada) foram bem representados no jogo, mesmo que de forma implícita. \\ \cline{2-4}
  & Relevância & 20 & O jogo é útil para treinar conceitos já conhecidos. \\ \cline{2-4}
  & Relevância & 21 & A abordagem implícita (ensinar sem explicar diretamente) foi positiva. \\
\hline

\end{longtable}
}


De forma geral, as perguntas removidas foram devido a não se encaixarem nas
propostas do trabalho atual ou por não serem o foco deste. Além disso, algumas
perguntas relacionadas ao \textbf{conteúdo de ensino} foram colocadas em uma nova
dimensão.

\subsection{Análise dos Dados}

Os dados quantitativos serão analisados por meio de estatísticas descritivas 
(média, mediana e desvio padrão). A consistência interna dos itens poderá ser 
verificada utilizando o coeficiente alfa de Cronbach \cite{Cronbach1951}.

As respostas abertas serão submetidas a análise qualitativa por categorização 
temática, buscando identificar padrões e percepções recorrentes relacionadas à 
experiência de uso e ao potencial educativo do jogo.

\subsection{Limitações}

A avaliação realizada compreendeu apenas aspectos relacionados à experiência 
do jogador e à usabilidade. A etapa de avaliação educacional completa, que 
pressupõe aplicação em ambiente de ensino real, não pôde ser conduzida dentro 
do escopo deste trabalho e é indicada como continuidade em
\autoref{sec:trabalhos_futuros}. Ainda assim, a metodologia apresentada
estabelece um procedimento estruturado para validação empírica do jogo em 
etapas futuras.
