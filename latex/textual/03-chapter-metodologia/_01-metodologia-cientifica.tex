\section{Metodologia Científica}

A metodologia científica adotada neste trabalho é de natureza exploratória com
abordagem mista, qualitativa e quantitativa, e classifica-se como uma pesquisa
aplicada. A próxima seção detalha e justifica essas escolhas.

\subsection{Classificação Metodológica}

Este trabalho caracteriza-se como uma \textbf{pesquisa aplicada}, pois tem como
objetivo solucionar um problema prático relacionado ao ensino de estruturas de
dados por meio da utilização de um jogo sério. Diferentemente da pesquisa
puramente teórica, a pesquisa aplicada visa gerar conhecimento com aplicação
direta em contextos específicos. Neste caso, o foco está no ambiente
educacional.

A natureza \textbf{exploratória} da pesquisa se deve ao fato de buscar o
desenvolvimento de um \textbf{artefato inovador} (o jogo) e sua
\textbf{avaliação de design e usabilidade}, sem propor uma intervenção
experimental para medir o ganho de aprendizagem em um ambiente controlado. O
foco é a concepção e avaliação da qualidade técnica e da experiência do usuário
do protótipo funcional, e não a observação dos efeitos dessa intervenção no
processo de aprendizagem em um estudo de eficácia.

Adota-se uma \textbf{abordagem mista}, combinando métodos qualitativos e
quantitativos com o intuito de proporcionar uma análise mais abrangente e
precisa dos resultados do artefato. Os dados quantitativos são coletados por
meio de instrumentos como questionários estruturados (por exemplo, escalas
Likert ou o modelo MEEGA+), permitindo uma avaliação objetiva da usabilidade e
da experiência de jogo. Já os dados qualitativos são obtidos por meio de
observações diretas da interação dos participantes com o jogo e da análise de
questões abertas nos questionários, o que permite compreender de forma mais
aprofundada a experiência dos usuários e identificar pontos de melhoria no
design da ferramenta.

\subsection{Etapas da Pesquisa}

Esta pesquisa foi composta por diversas etapas, focadas no desenvolvimento e
avaliação do artefato:

\begin{enumerate}
    \item \textbf{Revisão Bibliográfica.} Por meio da qual foram identificadas
      publicações acadêmicas e aplicações relacionadas ao uso de jogos sérios
      no ensino de conceitos de programação e as dificuldades da disciplina de
      Estruturas de Dados. Essa etapa proporcionou uma visão das abordagens já
      utilizadas e os conceitos mais frequentemente aplicados.

    \item \textbf{Desenvolvimento do Artefato.} Em seguida, foi idealizado e
      desenvolvido um jogo sério educacional (protótipo funcional), embasado na
      metodologia ENgAGED ou similar. Esse processo envolveu diversas fases
      como a concepção, design e implementação do jogo proposto em um motor
      como Godot, Unity ou Unreal.

    \item \textbf{Avaliação do Design e Usabilidade.} Nesta etapa, a coleta de
      dados foi realizada por meio de testes com usuários para avaliar o
      \textbf{design, a usabilidade e a aceitação percebida} do artefato. Foram
      empregadas técnicas qualitativas e quantitativas (questionários e
      observação) para avaliar a experiência dos participantes e a qualidade
      técnica da ferramenta como recurso educacional, e \textbf{não a sua
      eficácia pedagógica}.

    \item \textbf{Validação dos Resultados.} Por fim, realizou-se a análise e o
      cruzamento dos dados obtidos na avaliação com os objetivos da pesquisa.
      Esta validação focou em verificar se o artefato desenvolvido é
      \textbf{viável, útil e atraente} para o ensino de estruturas de dados, e
      se proporcionou uma experiência motivadora e coerente em termos de design
      de jogo.
\end{enumerate}
