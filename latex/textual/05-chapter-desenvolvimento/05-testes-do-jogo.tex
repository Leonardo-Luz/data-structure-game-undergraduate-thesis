\section{Testes do Jogo}

Os testes do jogo foram conduzidos com o objetivo de identificar erros, avaliar
o funcionamento das mecânicas e aprimorar a experiência geral de jogabilidade.
Esta etapa teve caráter iterativo, sendo realizada após a implementação de cada
nova funcionalidade, tanto de forma unitária quanto integrada, garantindo a
coerência entre os diferentes componentes do sistema.

Os testes foram executados pelo próprio desenvolvedor, permitindo validar a
adequação pedagógica e técnica do jogo. Essa prática possibilitou identificar
falhas de lógica, inconsistências visuais e problemas de desempenho, além de
ajustar elementos de interface e equilíbrio de dificuldade.

A verificação contínua de interações e funcionalidades contribuiu para
assegurar que o produto final estivesse livre de erros críticos e em
conformidade com os objetivos estabelecidos. Assim, esta fase consolidou-se
como um processo essencial para garantir a estabilidade e a fluidez do jogo
sério desenvolvido.
