\section{Design do Jogo}

Esta seção apresenta o processo de concepção visual e estrutural do jogo,
abrangendo desde a definição estética dos elementos gráficos até a organização
espacial e narrativa das fases. O design do jogo foi orientado pelos requisitos
funcionais estabelecidos na etapa de desenvolvimento, buscando aliar clareza
visual, coerência temática e funcionalidade interativa. São descritos os
principais componentes visuais, como personagens, inimigos, cenários e
interfaces, bem como a modelagem do nível inicial e dos diálogos que compõem a
experiência lúdica e pedagógica proposta.

\subsection{Definir Game Engine}

Dentre as diversas \textit{game engines} disponíveis atualmente, realizou-se uma
análise comparativa considerando critérios como facilidade de uso,
documentação, suporte multiplataforma, performance, comunidade ativa e
adequação aos objetivos educacionais do projeto. A partir desses critérios,
foram selecionadas três alternativas principais para avaliação: \textbf{Godot},
\textbf{Unreal Engine} e \textbf{Unity}.

\begin{table}[H]
  \caption{Comparativo entre Game Engines}
  \label{tab:comparativo_engines}
  \centering
  \footnotesize
  \begin{tabular}{|p{3cm}|p{3cm}p{3cm}p{3cm}|}
\hline
\rowcolor{headergray}
    \textbf{Critério} &
    \textbf{Godot Engine} &
    \textbf{Unity} &
    \textbf{Unreal Engine} \\
\hline
    \textbf{Licença} &
    MIT (código aberto) &
    Proprietária (gratuita para uso educacional) &
    Proprietária (royalties comerciais) \\
\hline
    \textbf{Facilidade de uso} &
    Alta &
    Alta &
    Moderada \\
\hline
    \textbf{Performance} &
    Boa &
    Muito boa &
    Muito boa \\
\hline
    \textbf{Suporte a WebGL} &
    Estável, mas limitado &
    Amplo e otimizado &
    Limitado e com alto custo de compilação \\
\hline
    \textbf{Comunidade e Documentação} &
    Em crescimento &
    Ampla e consolidada &
    Consolidada, porém mais voltada a estúdios \emph{Triple A} \\
\hline
  \end{tabular}
  \caption*{Fonte: Autor, \cite{godotmanual, unitymanual, unrealmanual}}
\end{table}


A \textbf{Godot} apresenta-se como uma ferramenta de código aberto, leve e
altamente personalizável, com uma comunidade em crescimento constante. Seu
principal atrativo está na licença permissiva e na facilidade de prototipagem
rápida. No entanto, limitações no desempenho gráfico e na maturidade de certos
recursos, especialmente para exportações WebGL, podem representar desafios em
projetos que demandam maior desempenho. \cite{godotmanual}

A \textbf{Unreal Engine}, por outro lado, é reconhecida por seu poder gráfico e
ferramentas robustas voltadas a produções de grande escala. Entretanto, sua
curva de aprendizado mais acentuada, o tamanho considerável das compilações e a
demanda por hardware mais potente a tornam menos adequada a projetos de caráter
educacional ou voltados à execução em navegadores. \cite{unrealmanual}

Por fim, a \textbf{Unity} destaca-se como uma solução equilibrada entre
flexibilidade e desempenho. Sua arquitetura modular, ampla documentação,
integração multiplataforma e suporte consolidado ao WebGL permitem o
desenvolvimento de aplicações interativas com boa performance e fácil
implantação em ambientes educacionais. Além disso, a curva de aprendizado mais
suave e o vasto ecossistema de recursos contribuem para a agilidade no processo
de prototipagem e implementação. \cite{unitymanual}

Dessa forma, a \textit{Game Engine} selecionada para o desenvolvimento do
projeto foi a \textbf{Unity}, por sua robustez, suporte multiplataforma e
capacidade de criar jogos com desempenho eficiente, alinhando-se às
necessidades tanto técnicas quanto pedagógicas desta pesquisa.

\subsection{Produzir imagens dos elementos do jogo}

Nesta etapa, são apresentadas as representações visuais dos principais
elementos descritos anteriormente na \autoref{tab:elementos_jogo}. As imagens
ilustram a aparência, o estilo artístico e o papel de cada componente dentro do
ambiente do jogo, complementando as descrições textuais e auxiliando na
compreensão da composição estética e funcional do projeto.

A \autoref{fig:jogador} apresenta o personagem principal, responsável pelas
misturas alquímicas e pela interação com o ambiente. Seu design foi inspirado
nos Médicos da Peste Negra.
\begin{figure}[H]
	\centering
	\caption{Personagem principal do jogo}
	\includegraphics[width=0.4\textwidth]{images/jogador.png}
	\legend{Fonte: Autor}
	\label{fig:jogador}
\end{figure}


O personagem mostrado na \autoref{fig:gato} atua como companheiro e guia do
jogador, aparecendo de forma mística à sua frente. Sua concepção visual
baseia-se no Gato de Cheshire da obra \emph{Alice no País das Maravilhas}
\cite{cheshire_cat}.
\begin{figure}[H]
	\centering
	\caption{Companheiro felino}
	\includegraphics[width=0.4\textwidth]{images/gato.png}
	\legend{Fonte: Autor}
	\label{fig:gato}
\end{figure}


A fogueira da \autoref{fig:fogueira}, que funciona como ponto de controle e
restauração, segue a estética das fogueiras presentes em jogos do gênero
\emph{soulslike}\footnote{O termo \emph{soulslike} refere-se a jogos inspirados
em características da série \emph{Dark Souls}, como alta dificuldade, combate
tático e progressão punitiva.}. O conceito geral foi inspirado por obras como
\emph{Dark Souls} \cite{darksouls}.
\begin{figure}[H]
	\centering
	\caption{Fogueira, ponto de controle do jogador}
	\includegraphics[width=0.4\textwidth]{images/fogueira.png}
	\legend{Fonte: Autor}
	\label{fig:fogueira}
\end{figure}


Os elementos alquímicos Ar, Fogo, Terra e Água, apresentados na
\autoref{fig:elementos_alquimicos}, foram baseados em símbolos tradicionais da
alquimia medieval. \cite{alchemical_symbols}
\begin{figure}[H]
	\centering
	\caption{Elementos alquímicos utilizados nas combinações do jogo}
	\includegraphics[width=0.4\textwidth]{images/elementos_alquimicos.png}
	\legend{Fonte: Autor}
	\label{fig:elementos_alquimicos}
\end{figure}


O objeto que marca o fim da fase, representado na \autoref{fig:bandeira}, toma
como referência elementos clássicos de jogos de plataforma, como a bandeira de
\emph{Super Mario Bros.} \cite{super_mario} e o letreiro de conclusão de fase
de \emph{Sonic the Hedgehog} \cite{sonic}.
\begin{figure}[H]
	\centering
	\caption{Bandeira sinalizando o término da fase}
	\includegraphics[width=0.4\textwidth]{images/bandeira.png}
	\legend{Fonte: Autor}
	\label{fig:bandeira}
\end{figure}


Os principais inimigos enfrentados na fase estão ilustrados na
\autoref{fig:inimigos}.
\begin{figure}[H]
	\centering
	\caption{Inimigos presentes no jogo}
	\includegraphics[width=0.4\textwidth]{images/inimigos.png}
	\legend{Fonte: Autor}
	\label{fig:inimigos}
\end{figure}


Da esquerda para a direita, são apresentados:
{\footnotesize
\begin{enumerate}[label=\arabic*)]
  \item Gosma, inspirada em criaturas similares presentes em franquias como
    \emph{Final Fantasy} \cite{final_fantasy} e \emph{Dragon Quest}
    \cite{dragon_quest};
  \item Cogumelo, cuja estética remete a inimigos clássicos de \emph{Super
    Mario Bros.} \cite{super_mario} e referências visuais de \emph{Hollow
    Knight} \cite{hollow_knight};
  \item Olho Alado, baseado em criaturas recorrentes de séries como \emph{The
    Legend of Zelda} \cite{zelda} e \emph{Final Fantasy} \cite{final_fantasy}.
\end{enumerate}
}

Os blocos modulares que compõem o terreno e as plataformas do cenário estão
representados na \autoref{fig:tiles}.
\begin{figure}[H]
	\centering
	\caption{Blocos do cenário utilizados na construção dos níveis}
	\includegraphics[width=0.4\textwidth]{images/tiles.png}
	\legend{Fonte: Autor}
	\label{fig:tiles}
\end{figure}


As diferentes camadas de paralaxe\footnote{Paralaxe é uma técnica de design
visual em que camadas do cenário se movem em velocidades distintas, criando a
ilusão de profundidade.} utilizadas para compor a profundidade visual do
ambiente estão apresentadas na \autoref{fig:parallaxes}.
\begin{figure}[H]
	\centering
	\caption{Camadas de Parallax que compõem a profundidade visual do cenário}
	\includegraphics[width=0.8\textwidth]{images/parallaxes.png}
	\legend{Fonte: Autor}
	\label{fig:parallaxes}
\end{figure}


Os consumíveis disponíveis ao jogador, responsáveis por fornecer vantagens
temporárias ou restaurar atributos, são mostrados na \autoref{fig:consumiveis}.
\begin{figure}[H]
	\centering
	\caption{Consumíveis disponíveis ao jogador durante a partida}
	\includegraphics[width=0.4\textwidth]{images/consumiveis.png}
	\legend{Fonte: Autor}
	\label{fig:consumiveis}
\end{figure}


Da esquerda para a direita, os consumíveis são:
{\footnotesize
\begin{enumerate}[label=\arabic*)]
  \item Consumível de Inserção;
  \item Consumível de Remoção;
  \item Consumível de Ordenação;
  \item Consumível de Cura;
  \item Consumível de Mana;
  \item Consumível de Vida Extra.
\end{enumerate}
}

Por fim, a \textbf{HUD} apresenta as principais informações de jogabilidade,
como inventário, vida, mana e botões de interação. A \autoref{fig:hud} mostra a
interface numerada.
\begin{figure}[H]
	\centering
	\caption{Interface de jogo com elementos numerados}
	\includegraphics[width=0.8\textwidth]{images/hud.png}
	\legend{Fonte: Autor}
	\label{fig:hud}
\end{figure}


{\footnotesize
\begin{enumerate}[label=\arabic*)]
  \item Indicador de tentativas restantes;
  \item Pontos de vida do jogador;
  \item Barra de mana;
  \item Botão de pausa;
  \item Livro de tutoriais;
  \item Espaço do consumível ativo;
  \item Indicador da saída da pilha;
  \item Indicador da saída da fila;
  \item Indicador da saída da lista;
  \item Indicador da próxima inserção.
\end{enumerate}
}

\subsection{Modelar o jogo}

Nesta etapa, o processo de modelagem do jogo concentrou-se na definição do
nível inicial e na estruturação dos diálogos que compõem a experiência
narrativa e pedagógica. O objetivo foi estabelecer um ambiente introdutório que
permitisse ao jogador compreender as mecânicas básicas, interagir com os
principais elementos do cenário e familiarizar-se com a proposta do jogo.

\begin{figure}[H]
	\centering
	\caption{Design do Nível Inicial}
	\includegraphics[width=0.8\textwidth]{images/level-design-1.png}
	\legend{Fonte: Autor}
	\label{fig:level_inicial}
\end{figure}

