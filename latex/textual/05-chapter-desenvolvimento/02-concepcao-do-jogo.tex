\section{Concepção do Jogo} \label{sec:concepcao_jogo}

Esta seção apresenta o processo de concepção do jogo sério. Aqui são descritos
os principais elementos que compõem o jogo, incluindo seus objetivos,
narrativa, regras, mecânicas, elementos visuais, sistema de pontuação e
\emph{feedback} educacional, estabelecendo as bases conceituais para o desenvolvimento
apresentado nas seções seguintes.

\subsection*{Objetivos do Jogo}

O jogador tem como objetivo narrativo principal recuperar sua pesquisa. Para
alcançar esse propósito, deverá enfrentar diferentes inimigos e superar
diversos obstáculos ao longo das fases, avançando até chegar ao chefe final e
derrotá-lo.

\subsubsection*{Gênero e Perspectiva}

O jogo é um \textbf{platformer 2D}\footnote{Gênero de jogo plataforma em duas
dimensões} de fantasia e aventura com perspectiva
\textit{side-scrolling}\footnote{Estilo de \emph{gameplay} onde o jogador se
move horizontalmente e a tela o segue.}. Essa escolha baseia-se em clássicos
como Super Mario Bros e Mega Man, facilitando a adoção do jogo e tornando a
experiência intuitiva.

\subsection*{Temática e Narrativa}

A narrativa é ambientada em um universo fantasioso de \textbf{alquimia} mística
e medieval. O protagonista, um alquimista representado como um doutor da peste,
tem sua pesquisa sobre a \textbf{pedra filosofal} roubada por um rival enquanto
viajava para apresentá-la ao Círculo dos Alquimistas, um grupo de estudiosos
de grande prestígio.

Seu objetivo passa a ser recuperar essa pesquisa, atravessando diferentes
regiões e enfrentando diversos adversários ao longo da jornada, acompanhado
ocasionalmente por um gato que surge como aliado em momentos específicos.

Essa narrativa fornece contexto e motivação para as ações do jogador,
aumentando o engajamento e criando imersão. O enredo também justifica a
presença dos elementos alquímicos como mecanismo central de combate.

\subsection*{Regras}

As regras do jogo definem as condições que orientam o comportamento do jogador
e do sistema, estabelecendo os limites e critérios que estruturam a experiência
interativa. Elas determinam as ações permitidas, as consequências de erros e
acertos, e os parâmetros que conduzem à progressão ou ao término da fase.

No contexto deste projeto, as regras estão intrinsecamente relacionadas aos
objetivos educacionais e narrativos do jogo. O jogador deve explorar o
ambiente, realizar combinações de elementos alquímicos e superar desafios
seguindo um conjunto de condições previamente definidas. Para progredir, é
necessário alcançar o final de cada fase, vencendo os obstáculos e inimigos
dispostos ao longo do percurso. Entre as principais regras implementadas, destacam-se:

\begin{itemize}
  \item O jogador perde pontos de vida ao realizar combinações incorretas de
    elementos alquímicos ou ao entrar em contato com inimigos e seus
    respectivos projéteis.

  \item Cada inventário possui um limite máximo de três elementos;

  \item A conclusão da fase ocorre mediante a interação do jogador com um
    objeto específico localizado ao final do nível;

  \item O tempo total de conclusão influencia diretamente a pontuação final
    obtida;

  \item O jogador é penalizado por mortes e erros recorrentes, refletindo os
    princípios de avaliação formativa e somativa;

  \item Ao atingir zero pontos de vida, o jogador perde uma tentativa e retorna
    ao ponto de controle anterior, recuperando integralmente sua vida;

  \item Caso o jogador fique sem tentativas restantes e volte a atingir zero
    pontos de vida, a fase é reiniciada, resultando na perda de todo o
    progresso obtido.
\end{itemize}

\subsection*{Mecânicas}

A seguir, são aprofundadas as principais mecânicas desenvolvidas seguindo,
majoritariamente, os requisitos funcionais previamente apresentados na
\autoref{sec:req_func}, sendo divididas entre mecânicas centrais e mecânicas
genéricas.

\subsubsection*{Mecânicas Centrais}

As mecânicas centrais correspondem aos elementos de jogabilidade diretamente
relacionados aos objetivos principais do projeto e à aplicação dos conceitos de
estrutura de dados. Elas representam o núcleo funcional do jogo, onde se
concentram as interações que concretizam a proposta educacional do
desenvolvimento. Na \autoref{tab:mecanicas_centrais}, a seguir, essas mecânicas
são apresentadas e descritas, evidenciando como os princípios de estrutura de
dados foram incorporados à dinâmica do jogo.

\begin{table}[H]
	\caption{Mecânicas Centrais}
	\label{tab:mecanicas_centrais}
	\centering
	\footnotesize
  \begin{tabular}{lp{8cm}}
		\toprule
		\textbf{Mecânica} & \textbf{Descrição} \\
		\midrule
    Manipulação de Inventários & Cada inventário é estruturado com base em uma
    estrutura de dados específica, permitindo, em casos específicos, operações
    como inserção, remoção, ordenação e movimentação de ponteiros. Essas
    operações preservam as propriedades de cada estrutura, proporcionando uma
    experiência prática de seus comportamentos. \\
		\midrule
    Geração de Elementos & Permite ao jogador criar novos elementos mediante o
    consumo de mana. Os elementos gerados são automaticamente alocados no
    primeiro espaço livre de um dos inventários, seguindo a ordem da esquerda para a
    direita. \\
		\midrule
    Consumíveis & Itens de uso estratégico que possibilitam realizar ações
    diretamente sobre os inventários, como inserir, remover ou ordenar
    elementos, servindo como ferramentas para reforçar o aprendizado sobre as
    operações das estruturas de dados. \\
		\midrule
    Mistura de Elementos & Elementos do mesmo tipo podem ser combinados para
    gerar ataques. O dano causado varia conforme a quantidade de elementos
    iguais utilizados na combinação. A realização da mistura exige a remoção
    prévia dos elementos dos inventários correspondentes. \\
		\midrule
    Fraquezas & Cada inimigo apresenta vulnerabilidades específicas a
    determinados elementos. O jogador deve identificar e combinar corretamente
    os elementos para maximizar o dano e avançar com eficiência.
    \\
		\midrule
    Sistema de Pontuação & Calcula o desempenho do jogador com base em
    múltiplos fatores, como acertos e erros nas combinações, número de
    tentativas e tempo total para concluir a fase, promovendo uma avaliação
    quantitativa da performance. \\
		\midrule
    Livro de Tutoriais & aqui são armazenados de forma explícita os tutoriais do
    jogo para que, caso o jogador se perca, ele possa revisar algo que não tenha
    entendido. \\
		\bottomrule
	\end{tabular}
  \caption*{Fonte: Autor}
\end{table}


\subsubsection*{Mecânicas Genéricas}

As mecânicas genéricas compreendem o conjunto de funcionalidades fundamentais
que sustentam a jogabilidade e garantem a coerência da experiência interativa.
Elas formam a base sobre a qual as demais dinâmicas e desafios são construídos,
assegurando que o jogador tenha controle responsivo e \emph{feedbacks} claros de suas
ações. Na \autoref{tab:mecanicas_genericas}, a seguir, essas mecânicas são
apresentadas e descritas de forma a evidenciar seu papel na estrutura
fundamental do jogo.

\begin{table}[H]
	\caption{Mecânicas Genéricas}
	\label{tab:mecanicas_genericas}
	\centering
	\footnotesize
  \begin{tabular}{lp{8cm}}
		\toprule
		\textbf{Mecânica} & \textbf{Descrição} \\
		\midrule
    Movimentação 2D & Permite ao jogador deslocar o personagem em um plano
    bidimensional, controlando direções horizontais, como andar para a esquerda
    e direita. \\
		\midrule
    Pulo Responsivo & Define uma mecânica de salto com resposta imediata aos
    comandos do jogador, garantindo sensação de controle preciso e previsível. \\
		\midrule
    Corrida & Possibilita ao personagem movimentar-se em velocidade aumentada
    mediante a sustentação de um comando específico, ampliando o dinamismo do
    deslocamento. \\
		\midrule
    Pontos de Controle & Permitem salvar o progresso do jogador em determinados
    trechos do nível, reduzindo a frustração causada por reinícios completos e
    otimizando o fluxo de tentativa e erro. \\
		\midrule
    Tentativas Limitadas & Estabelecem um número máximo de vidas ou tentativas antes do reinício total do nível, criando tensão adicional e incentivando o aprendizado a partir dos erros. \\
		\midrule
    Pontos de Vida & Representam a resistência do personagem diante de danos.
    Quando os pontos de vida chegam a zero, o jogador sofre penalidades, como
    perda de progresso ou reinício do nível. \\
		\midrule
    Mana & Recurso utilizado para a geração de elementos, cuja gestão
    estratégica é necessária para equilibrar poder ofensivo e conservação de
    energia durante os desafios. \\
		\midrule
    Pulo Duplo Situacional & Permite ao personagem realizar um segundo salto no
    ar apenas após coletar um orbe específico durante o salto. \\
		\midrule
    Capacidade de Aparar Certos Ataques & O jogador pode bloquear determinados
    projéteis inimigos utilizando um projétil próprio, neutralizando o dano e
    evitando impactos diretos. \\
		\bottomrule
	\end{tabular}
  \caption*{Fonte: Autor}
\end{table}


\subsection*{Elementos do Jogo}

Esta seção apresenta os principais elementos que compõem a estrutura e a
experiência interativa do jogo. Cada componente foi projetado para contribuir
com a jogabilidade, a narrativa e a ambientação, formando um ecossistema
coeso entre mecânica e estética. A \autoref{tab:elementos_jogo} descreve os
objetos implementados e suas respectivas funções no contexto do jogo.

{\footnotesize
  \begin{longtable}{|>{\raggedright\arraybackslash}p{4cm} p{9cm}|}
    \caption{Elementos do jogo e suas descrições} \label{tab:elementos_jogo} \\

    \hline
    \rowcolor{headergray}
    \textbf{Elemento} & \textbf{Descrição} \\
    \hline
    \endfirsthead

    \hline
    \rowcolor{headergray}
    \textbf{Elemento} & \textbf{Descrição} \\
    \hline
    \endhead

    \multicolumn{2}{|r|}{\textit{Continua na próxima página}} \\
    \hline
    \endfoot

    \hline
    \caption*{Fonte: Autor}
    \endlastfoot

    \rowcolor{accent}
    \multicolumn{2}{|l|}{\textbf{Entidades Jogáveis e Interativas}} \\
    \hline
    Jogador & Personagem principal, um alquimista caracterizado como um
    \textit{plague doctor}, responsável por realizar combinações alquímicas e
    interagir com o ambiente. \\
    \hline
    Consumíveis & Itens coletáveis que restauram atributos ou concedem vantagens
    temporárias. \\
    \hline
    Fogueira & Ponto de controle utilizado para salvar o progresso e restabelecer o
    estado do jogador. \\
    \hline
    Gato & Personagem auxiliar que fornece dicas e interage em momentos
    específicos, contribuindo para o enredo e orientação do jogador. \\
    \hline
    Bandeira & Objeto que indica o término da fase, sendo necessário interagir com
    ele para que a pontuação final seja exibida. \\
    \hline

    \rowcolor{accent}
    \multicolumn{2}{|l|}{\textbf{Inimigos}} \\
    \hline
    Gosma & Inimigo básico que se desloca lateralmente em patrulha constante. \\
    \hline
    Cogumelo & Inimigo que persegue o jogador ao entrar em seu alcance e libera
    esporos quando próximo. \\
    \hline
    Olho Alado & Inimigo aéreo que ataca à distância disparando projéteis. \\
    \hline

    \rowcolor{accent}
    \multicolumn{2}{|l|}{\textbf{Ambiente e Cenário}} \\
    \hline
    Objetos de ambientação & Elementos decorativos como grama, arbustos, cercas e
    rochas. \\
    \hline
    Partículas & Efeitos visuais dinâmicos como poeira, fumaça, chuva e vagalumes,
    que enriquecem a atmosfera. \\
    \hline
    Blocos do cenário & Estruturas modulares que compõem o solo e as plataformas do
    ambiente de jogo. \\
    \hline
    \emph{Background} & Representa o céu e o horizonte, compondo o plano mais
    distante do cenário. \\
    \hline
    \emph{Background} Distante de Árvores & Camada intermediária que exibe árvores
    ao longe. \\
    \hline
    \emph{Background} Próximo & Camada mais próxima ao jogador, contendo árvores e
    vegetação detalhada. \\
    \hline
    Camada Frontal de Árvores & Elementos que passam à frente do jogador, criando
    sensação de profundidade. \\
    \hline
    Camada Frontal de Rochas e Arbustos & Sombras e detalhes em primeiro plano que
    reforçam o volume do cenário. \\
    \hline

    \rowcolor{accent}
    \multicolumn{2}{|l|}{\textbf{Interface e Menus}} \\
    \hline
    HUD & Interface principal que exibe informações como pontos de vida,
    tentativas, inventário, livro de tutoriais e botão de pausa. \\
    \hline
    Mensagens Temporários & Mensagens contextuais apresentadas de forma breve,
    utilizadas para instruções rápidas. \\
    \hline
    Menu Principal & Tela inicial que permite o acesso às demais seções do jogo. \\
    \hline
    Menu de Fases & Interface de seleção de níveis disponíveis. \\
    \hline
    Menu de Pause & Tela de pausa com acesso às opções e retorno à partida. \\
    \hline
    Menu de Opções & Acesso às opções de volume e idioma. \\
    \hline
    Menu de Volume & Controle individual dos níveis sonoros do jogo. \\
    \hline
    Menu de Linguagem & Seleção de idioma. \\
    \hline
    Menu de Saída & Opção para encerrar o jogo. \\
  \end{longtable}
}


Cabe destacar que nem todos os elementos desenvolvidos nesta etapa estão
contemplados na \autoref{tab:elementos_jogo} acima. Alguns objetos secundários e recursos
complementares, implementados com o objetivo de aprimorar a imersão e a fluidez
da experiência, foram omitidos por não exercerem papel central na dinâmica do
jogo.

\subsection*{Pontuação}

O sistema de pontuação foi desenvolvido para oferecer uma forma objetiva de
mensurar o desempenho do jogador ao longo das partidas. Ao término de cada
fase, o jogo apresenta um \emph{feedback} somativo por meio de um relatório
de desempenho que sintetiza os resultados obtidos. A pontuação total é
calculada a partir de múltiplos fatores, como o número de combinações corretas
e incorretas, mortes e tempo total de conclusão, permitindo avaliar tanto a
eficiência quanto a precisão das ações realizadas.

Esse cálculo é processado de forma automatizada, garantindo consistência na
avaliação e eliminando subjetividades na análise do desempenho. A pontuação
final é determinada a partir de uma combinação de fatores relacionados ao tempo
de conclusão da fase, ao número de infusões corretas e incorretas, e à
quantidade de mortes do jogador.

A fórmula utilizada para o cálculo da pontuação final é expressa pela
\autoref{eq:calc_pont}, em que cada componente contribui de forma ponderada
para o resultado geral:

\begin{equation}
S = (1000 - 10\,t) + ( (100\,c \cdot \,f ) - 25\,e) - 150\,d
\label{eq:calc_pont}
\end{equation}

em que:

\begin{itemize}
  \item \(S\) representa a pontuação final obtida pelo jogador;
  \item \(t\) é o tempo total de conclusão da fase (em segundos);
  \item \(c\) é o número de infusões corretas realizadas;
  \item \(f\) corresponde à complexidade da combinação.
  \item \(e\) é o número de infusões incorretas;
  \item \(d\) corresponde ao número de mortes ocorridas.
\end{itemize}

\subsection*{\emph{Feedback} Educacional}

O sistema de \textit{feedback} educacional foi concebido para reforçar o
aprendizado de forma contínua e integrada à experiência de jogo. Sua
implementação busca promover uma aprendizagem ativa, na qual o jogador recebe
respostas imediatas às suas ações, sem comprometer a fluidez do
\textit{gameplay}. Essa dinâmica permite que o processo de experimentação e
correção ocorra de maneira natural, favorecendo a compreensão dos efeitos de
cada escolha e das relações entre os conceitos de estrutura de dados
representados nas mecânicas do jogo.

Os diferentes tipos de \textit{feedback} foram projetados para atuar em
múltiplos níveis de interação, variando de respostas visuais imediatas a
avaliações formativas sutis, conforme sua função no processo de aprendizagem.
Quando o jogador realiza uma combinação incorreta, o sistema apresenta efeitos
sonoros e um \emph{feedback} visual de erro, evidenciado por mudanças de cor
no personagem e efeitos de penalidade, como a perda de pontos de vida. Esse
retorno instantâneo sinaliza a falha sem interromper a continuidade da ação,
permitindo que o jogador aprenda com o erro por meio da tentativa e erro.

Por outro lado, ao executar combinações corretas de elementos alquímicos, o
jogador recebe uma \textbf{avaliação formativa implícita}, expressa por efeitos
sonoros e visuais que indicam sucesso na ação. Esse tipo de retorno atua como
reforço positivo, incentivando a experimentação e consolidando o aprendizado
por meio da prática direta.

Complementando esses retornos imediatos, o \emph{feedback} somativo
apresentado ao final de cada fase, descrito na seção anterior,
integra o processo avaliativo de forma global, permitindo ao jogador refletir
sobre seu desempenho e identificar aspectos a melhorar. Assim, o sistema de
\textit{feedback} atua de maneira articulada, combinando respostas instantâneas
e avaliações finais para sustentar um ciclo contínuo de aprendizado.
