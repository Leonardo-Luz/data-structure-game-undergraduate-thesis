\section{Implementação do Jogo} \label{sec:implementacao}

Esta seção apresenta a implementação do jogo desenvolvido na \emph{game engine}
\textbf{Unity}. Inicialmente, contextualizam-se os principais conceitos técnicos
da \emph{engine}, que servem de base para a construção dos sistemas utilizados.
Em seguida, descrevem-se os principais elementos listados na
\autoref{tab:elementos_jogo} e discute-se como as mecânicas centrais e
genéricas, apresentadas nas \autoref{tab:mecanicas_centrais} e
\autoref{tab:mecanicas_genericas}, foram convertidas em funcionalidades
operacionais. Dessa forma, evidencia-se a transição do design conceitual para
sua materialização prática, demonstrando como cada decisão de design contribuiu
para a composição do sistema final.

\subsection*{Conceitos Fundamentais da Unity}

Antes de detalhar a implementação dos componentes do jogo, torna-se necessário
contextualizar alguns conceitos fundamentais da Unity, os quais orientam a
estruturação dos \emph{scripts}, a organização dos objetos e o comportamento do
sistema em tempo de execução. A Unity adota um modelo arquitetural baseado em
componentes, no qual cada objeto do jogo é composto por módulos independentes
responsáveis por sua lógica, aparência, física ou interação.

No centro desse modelo está a classe \texttt{MonoBehaviour}, da qual derivam a
maioria dos scripts comportamentais. Ela define o ciclo de vida padrão utilizado
pela \emph{engine}, permitindo que o desenvolvedor implemente métodos executados
automaticamente em momentos específicos da execução. Entre esses métodos,
destacam-se:

\textbf{\texttt{Start()}}: Executado uma única vez na inicialização do objeto,
sendo empregado para configurar estados iniciais, referências e variáveis de
controle.

\textbf{\texttt{Update()}}: Chamado a cada quadro (frame), sendo adequado para
lógicas dependentes da taxa de atualização, como leitura de entrada, animações e
verificações contínuas.

\textbf{\texttt{FixedUpdate()}}: Executado em intervalos fixos,
independentemente da taxa de quadros, sendo indicado para cálculos de física e
interações que exigem maior estabilidade numérica.

A Unity também integra um sistema de física robusto, por meio do componente
\break\texttt{Rigidbody2D}, que permite que objetos sejam afetados por
forças, gravidade e colisões. A combinação entre \texttt{Rigidbody2D} e
colisores define como cada objeto interage fisicamente com o ambiente,
possibilitando comportamentos essenciais para movimentação, detecção de
obstáculos e mecânicas de combate.

Esses conceitos estruturam toda a implementação do jogo e fundamentam a criação
dos sistemas apresentados nas subseções seguintes.

\subsection*{Implementação do Jogador}

A implementação teve início pelo componente central da experiência: o jogador.
Foram desenvolvidos diversos \emph{scripts} em \textbf{C\#}, responsáveis pelas
mecânicas de movimentação, combate e interação com os inventários.

A \autoref{code:movimentacao} apresenta o trecho referente à movimentação do
jogador, no qual são aplicadas diretamente as funcionalidades do ciclo de vida
da Unity e seus componentes físicos.

\begin{listing}[H]
\caption{Trecho Simplificado do Algoritmo de Movimentação do Jogador}
\begin{minted}[
fontsize=\footnotesize,
linenos,
bgcolor=backcolor,
tabsize=2,
breaklines=true,
numbersep=5pt,
showspaces=false,
showtabs=false,
]{cs}
using UnityEngine;

public class PlayerMove : MonoBehaviour
{
    private Rigidbody2D rb;

    private void FixedUpdate()
    {
        // Atualiza o movimento horizontal do jogador
        Move(Input.GetAxis("Horizontal"));

        // Inicia o salto se o jogador estiver no chão e pressionar o botão.
        if (Input.GetButtonDown("Jump") && isGrounded)
            Jump();
        // Reduz a força do salto caso o botão seja solto durante a ascensão.
        else if (Input.GetButtonUp("Jump") && !isGrounded)
            CancelJump();
    }

    private void Jump()
    {
        // Aplica a velocidade vertical correspondente ao salto.
        rb.linearVelocity = new Vector2(rb.linearVelocity.x, jumpForce);
    }

    private void CancelJump()
    {
        // Corta parte da velocidade vertical para permitir variação na altura
        // do salto conforme o tempo de pressionamento do botão.
        rb.linearVelocity = new Vector2(rb.linearVelocity.x, rb.linearVelocity.y * 0.5f);
    }

    private void Move(float input)
    {
        // Ajusta a velocidade horizontal do jogador de acordo com a entrada recebida.
        rb.linearVelocity = new Vector2(input * speed, rb.linearVelocity.y);
    }
}
\end{minted}
\label{code:movimentacao}
\end{listing}


Com o objetivo de estruturar e padronizar o funcionamento dos inventários,
representados por diferentes estruturas de dados, foi desenvolvida a interface
exibida no \autoref{code:interface_inventario}. Essa interface possibilita a
manipulação genérica das estruturas, permitindo alterar tanto o tipo quanto a
quantidade de inventários utilizados sem comprometer o restante do sistema.

\begin{listing}[H]
\caption{Interface dos Inventários}
\begin{minted}[
fontsize=\footnotesize,
linenos,
bgcolor=backcolor,
tabsize=2,
breaklines=true,
numbersep=5pt,
showspaces=false,
showtabs=false,
]{cs}
public enum InventoryType { Stack, Queue, LinkedList }

public interface IInventory
{
  public InventoryType Type { get; }
  public int Count { get; }
  public int MaxSize { get; }
  public bool IsFull();
  public void Clear();
  public void Sort();
  public Element[] ToArray();
}
\end{minted}
\label{code:interface_inventario}
\end{listing}


Essa interface é utilizada no \autoref{code:combate}, responsável pela lógica de
combate do jogador. Nesse módulo, as operações de ataque integram-se às
operações de remoção e combinação dos inventários, conectando
diretamente a mecânica de combate às representações das estruturas de dados.

\begin{listing}[H]
\caption{Trecho Simplificado do Algoritmo de Combate do Jogador}
\begin{minted}[
fontsize=\footnotesize,
linenos,
bgcolor=backcolor,
tabsize=2,
breaklines=true,
numbersep=5pt,
showspaces=false,
showtabs=false,
]{cs}
using UnityEngine;

public class PlayerCombat : MonoBehaviour
{
  public IInventory[] inventories;

  // Lista temporária contendo os elementos selecionados pelo jogador
  // para compor uma combinação (conjuração).
  public List<Element> castingList = new List<Element>();

  private void Update()
  {
    HandleLinkedListSelection();
    HandleCastingInput();
  }

  private void HandleLinkedListSelection()
  {
    // Responsável por alterar o elemento atualmente selecionado
    // na Lista Encadeada.
  }

  private void HandleCastingInput()
  {
    // Adiciona elementos à conjuração conforme o inventário selecionado.
    if (Input.GetKeyDown(KeyCode.Alpha1))
      AddFromInventory(InventoryType.Stack);
    else if (Input.GetKeyDown(KeyCode.Alpha2))
      AddFromInventory(InventoryType.Queue);
    else if (Input.GetKeyDown(KeyCode.Alpha3))
      AddFromInventory(InventoryType.LinkedList);

    // Finaliza a conjuração quando o jogador confirma a combinação.
    else if (Input.GetKeyDown(KeyCode.Return)) FinishCasting();
  }

  private void AddFromInventory(InventoryType type)
  {
    // Remove um elemento da estrutura de dados correspondente
    // (Pilha, Fila ou Lista Encadeada) e o adiciona à lista de conjuração.
  }

  private void FinishCasting()
  {
    // Executa a conjuração utilizando os elementos selecionados,
  }
}
\end{minted}
\label{code:combate}
\end{listing}


A lógica de geração e inserção de elementos nos inventários, essencial para o
funcionamento das mecânicas centrais, é apresentada no
\autoref{code:geracao_elementos}.

\begin{listing}[H]
\caption{Trecho Simplificado do Algoritmo de Geração de Elementos do Jogador}
\begin{minted}[
fontsize=\footnotesize,
linenos,
bgcolor=backcolor,
tabsize=2,
breaklines=true,
numbersep=5pt,
showspaces=false,
showtabs=false,
]{cs}
using UnityEngine;

public class GenerateElement : MonoBehaviour
{
  // Variáveis e configuração inicial do script

  private void Update()
  {
    bool holdingDown = Input.GetAxis("Vertical") < 0;

    // Quando o jogador mantém o comando para baixo enquanto está no chão,
    // inicia o processo de carregamento da habilidade de geração.
    if (holdingDown && grounded.IsGrounded())
    {
      chargeTimer += Time.deltaTime;
      curMana -= Time.deltaTime;

      // Quando o tempo acumulado atinge o valor de recarga (cooldown),
      // tenta gerar um novo elemento.
      if (chargeTimer >= cooldown)
      {
        TryGenerateElement();
        chargeTimer = 0f;
      }
    }
    // Caso o jogador não esteja carregando,
    // a mana é regenerada até o limite máximo.
    else if (curMana < maxMana)
    {
      curMana += Time.deltaTime * ManaRegen;
    }
  }

  private bool TryGenerateElement()
  {
    // Tenta inserir um elemento aleatório
    // em um dos inventários disponíveis.
    // Se o inventário atual estiver cheio, tenta o próximo.
    // Caso todos estejam indisponíveis,
    // a geração é considerada malsucedida.
  }
}
\end{minted}
\label{code:geracao_elementos}
\end{listing}


\subsection{Implementação dos Inimigos}

A implementação dos inimigos envolveu a criação de sistemas de movimentação,
detecção de colisões, aplicação de dano e controle de comportamentos específicos
para cada tipo de criatura. Além disso, foi incorporado o sistema de fraquezas,
que relaciona cada inimigo a tipos específicos de elementos mais eficazes contra
ele, conforme exemplificado no \autoref{code:fraquezas}. Esse recurso integra
o sistema de combate às estruturas de dados manipuladas pelo jogador.

\begin{listing}[H]
\caption{Trecho Simplificado do Algoritmo de Fraquezas dos Inimigos}
\begin{minted}[
fontsize=\footnotesize,
linenos,
bgcolor=backcolor,
tabsize=2,
breaklines=true,
numbersep=5pt,
showspaces=false,
showtabs=false,
]{cs}
using UnityEngine;

public class EnemyDamage : MonoBehaviour
{
  private void Start()
  {
    weaknesses = GenerateRandomWeaknesses();
  }

  private Element[] GenerateRandomWeaknesses()
  {
    // Gera um conjunto de elementos
    // que representam as fraquezas do inimigo.
    // A seleção ocorre de forma aleatória,
    // permitindo variação entre inimigos.
  }

  private void OnTriggerEnter2D(Collider2D collider)
  {
    // Verifica se o objeto que colidiu corresponde a algum elemento
    // ao qual o inimigo possui fraqueza. Caso corresponda, aplica dano.
    foreach (var weakness in weaknesses)
    {
      if ((weakness == Element.FIRE  && collider.CompareTag("Fire"))  ||
          (weakness == Element.WATER && collider.CompareTag("Water")) ||
          (weakness == Element.AIR   && collider.CompareTag("Air"))   ||
          (weakness == Element.EARTH && collider.CompareTag("Earth")))
      {
        // Obtém o valor de dano do projétil e aplica ao inimigo.
        int damage = collider.GetComponent<ProjectileController>().damage
        health.TakeDamage(damage);
        return;
      }
    }
  }
}
\end{minted}
\label{code:fraquezas}
\end{listing}


\subsection{Implementação da Pontuação}

O cálculo da pontuação final ao concluir uma fase é apresentado no
\autoref{code:calc_pont} e \break fundamenta-se na \autoref{eq:calc_pont}. O
algoritmo considera o tempo total, o número de combinações corretas,
combinações incorretas e mortes, consolidando a avaliação da performance do
jogador de forma objetiva e alinhada às mecânicas do jogo.

\begin{listing}[H]
\caption{Algoritmo utilizado para calcular a pontuação}
\begin{minted}[
fontsize=\footnotesize,
linenos,
bgcolor=backcolor,
tabsize=2,
breaklines=true,
numbersep=5pt,
showspaces=false,
showtabs=false,
]{cs}
public class Score : MonoBehaviour
{
  // ...

  private double CalculateFinalScore(double time)
  {
    // Calcula o bônus de tempo:
    // começa em 1000 e reduz 10 pontos por segundo.
    // Nunca permite que o valor fique negativo.
    double timeBonus = Mathf.Max(0, 1000 - (float)time * 10);

    // Pontuação baseada no número de infusões corretas
    // (cada uma vale 100 pontos).
    double correctScore = (correctInfusions * 100);

    // Penalidade por infusões erradas (cada erro tira 25 pontos).
    double wrongScore = (wrongInfusions * 25);

    // Soma da pontuação líquida relacionada às infusões.
    double infusionScore = correctScore - wrongScore;

    // Penalidade por mortes, cada morte reduz 150 pontos.
    double deathPenalty = deaths * 150;

    // Calcula a pontuação final somando tudo
    // e garantindo que não fique negativa.
    return Mathf.Max(0, (float)(timeBonus + infusionScore - deathPenalty));
  }

  // ...
}
\end{minted}
\label{code:calc_pont}
\end{listing}


Após a implementação dos \emph{scripts} centrais responsáveis pelas mecânicas do jogo
foram desenvolvidos também diversos sistemas de suporte que garantem a
experiência completa do usuário. Esses sistemas não interferem diretamente nas
regras do jogo, mas desempenham um papel essencial na qualidade visual, sonora,
narrativa e de usabilidade da aplicação. A seguir, são apresentados esses
subsistemas, suas finalidades e os principais aspectos de sua implementação.

\subsection*{Implementação da \emph{Pixel Perfect Camera}}

Para garantir uma apresentação visual coerente com a estética de \emph{pixel
art}, foi utilizada a dependência \emph{Pixel Perfect Camera}, disponibilizada
pela própria Unity e integrada ao sistema de câmeras da \emph{Cinemachine}.
Essa combinação permite preservar a nitidez dos pixels em diferentes resoluções
de tela, evitando distorções, desalinhamentos ou artefatos visuais. Além disso,
a configuração aplicada assegura um enquadramento estável e responsivo,
mantendo a consistência estética independentemente da escala do monitor do
usuário.

\subsection*{Implementação dos diálogos e seu controlador}

Foi desenvolvido um sistema dedicado para gerenciamento de diálogos, composto
por um controlador central responsável por exibir textos, avançar as falas e
sincronizar elementos visuais associados. O sistema permite a criação de
diálogos multilineares, apresentação progressiva do texto (\emph{typewriter
effect}) e ativação de eventos específicos no inicio e ao fim das falas. A
arquitetura adotada facilita a expansão futura do número de diálogos e
possibilita sua reutilização em diferentes cenas.

\subsection*{Implementação do controlador de idiomas}

Implementou-se também um controlador de idiomas responsável por gerenciar a
localização do jogo. Esse sistema altera dinamicamente os textos da interface
(HUD), diálogos e demais elementos textuais conforme o idioma selecionado pelo
jogador. Atualmente, estão disponíveis os idiomas português e inglês;
entretanto, o controlador foi estruturado para permitir a adição simples de
novas línguas.

\subsection*{Implementação do controlador de sons}

Foi criado um controlador de sons que organiza e gerencia os diferentes canais
de áudio do jogo. A estrutura divide o áudio em três categorias: volume
principal (\emph{master}), música e \emph{Sound Effects}\footnote{Efeitos
sonoros são sons criados para aumentar a imersão do jogador, o som de impacto
gerado ao jogador cair de um local alto é um exemplo de efeito sonoro.} (SFX). Cada
categoria pode ser ajustada individualmente, permitindo ao jogador personalizar
sua experiência sonora. Além disso, o controlador centraliza a reprodução de
sons contextuais do jogo, garantindo consistência e facilitando a manutenção do
sistema.


\subsection{Outros Componentes}

Além dos sistemas principais, diversos outros \emph{scripts} foram
implementados para garantir um \emph{gameplay} fluido, estável e responsivo.
Entre eles, incluem-se módulos para controle de partículas, interface de
usuário, indicadores visuais, carregamento de cenas e outras funcionalidades de
suporte. Embora não sejam o foco central das mecânicas educacionais, esses
componentes desempenham papel importante na polidez e na imersão da
experiência.

\subsection{Síntese da Implementação}

A \autoref{sec:implementacao} constituiu uma das etapas mais extensas e 
detalhadas do projeto, na qual foram materializados todos os elementos concebidos 
durante a \autoref{sec:concepcao_jogo}, bem como as mecânicas centrais e 
genéricas definidas nessa fase, sempre em conformidade com os requisitos 
funcionais e não funcionais estabelecidos. Nessa etapa, cada componente foi 
desenvolvido, integrado e ajustado de forma progressiva, consolidando a base 
técnica necessária para o funcionamento completo do jogo.

Com a implementação concluída, tornou-se possível avançar para a etapa de
testes, responsável por verificar o comportamento dos sistemas em execução,
avaliar sua estabilidade e assegurar que as funcionalidades produzidas
correspondessem aos objetivos educacionais e de jogabilidade pretendidos. A
seção seguinte descreve como esses testes foram conduzidos ao longo do
desenvolvimento.
