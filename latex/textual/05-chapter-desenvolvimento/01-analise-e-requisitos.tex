\section{Análise do Jogo e Levantamento de Requisitos}

Nesta etapa foi realizado o levantamento de requisitos do jogo, com o objetivo
de identificar suas funções e funcionalidades essenciais. Além disso,
definiu-se a forma de distribuição do conteúdo da Unidade Instrucional (UI),
conforme previamente planejado na \autoref{subsec:fase_2}, garantindo
alinhamento entre os elementos pedagógicos e as mecânicas propostas.

\newcounter{frcount}
\newcommand{\frvalue}{\ifnum\value{frcount}<10 0\fi\arabic{frcount}}
\newcommand{\frid}{FR-\frvalue}

\newcommand{\fr}[3]{
  \stepcounter{frcount}

  \begin{table}[H]
    \caption{Requisito Funcional \frvalue}
    \begin{tabular}{|p{3cm}|p{8cm}|p{3cm}|}
      \hline
      \rowcolor{headergray}
      \textbf{Identificador} & \textbf{Nome} & \textbf{Prioridade} \\

      \hline
      \frid & {#1} & {#2} \\

      \hline
      \multicolumn{3}{|p{14cm}|}{\textbf{Descrição:}{#3}} \\
      \hline
    \end{tabular}
    \caption*{Fonte: Autor}
  \end{table}

  \vspace{1em}
}

\section{Requisitos Funcionais}

\fixme{escrever uma introdução}

\fr{Salvar progresso}{Alta}
{
O sistema deve permitir que os usuários salvem o progresso de suas partidas em multiplas instâncias.
}


\fr{Ponto de controle}{Médio}
{
O sistema deve possuir um ponto de controle no meio de cada fase, onde o usuário poderá retornar caso falhe.
}

\fr{Manipular inventários}{Alta}
{
O sistema deve permitir que os usuários manipulem seus inventários apartir ações.
}


\newcounter{nfrcount}

\newcommand{\nfrvalue}{\ifnum\value{nfrcount}<10 0\fi\arabic{nfrcount}}
\newcommand{\nfrid}{NFR-\nfrvalue}

\newcommand{\nfr}[3]{
  \stepcounter{nfrcount}
  \begin{table}[H]
    \caption{Requisito Não Funcional \nfrvalue}
    \begin{tabular}{|p{3cm}|p{8cm}|p{3cm}|}
      \hline
      \rowcolor{headergray}
      \textbf{Identificador} & \textbf{Nome} & \textbf{Categoria} \\

      \hline
      \nfrid & {#1} & {#2} \\

      \hline
      \multicolumn{3}{|p{14cm}|}{\textbf{Descrição:}{#3}} \\
      \hline
    \end{tabular}
    \caption*{Fonte: Autor}
  \end{table}

  \vspace{1em}
}

\section{Requisitos não funcionais}

\fixme{escrever uma introdução}

\nfr{Frames por segundo estáveis}{Performance}
{
O sistema deve se manter em um taxa de atualização constante e acima de 30 frames por segundo.
}

\nfr{Multiplos idiomas}{Usabilidade}
{
O sistema deve possuir tradução para o português e para o inglês.
}

\nfr{Paleta de cores}{Acessibilidade}
{
O sistema deve possuir uma paleta de cores que não dificulte a visualização de elementos para pessoas daltônicas.
}

\nfr{Tamanho dos textos}{Acessibilidade}
{
O sistema deve possuir opções para alterar o tamanho dos textos.
}

\nfr{Contraste nos textos}{Acessibilidade}
{
O sistema deve possuir uma opção para adicionar contraste aos textos.
}


\subsection*{Distribuição do conteúdo da Unidade Instrucional}

A distribuição do conteúdo da UI foi estruturada para introduzir gradualmente
os conceitos relacionados às estruturas de dados ao longo da primeira fase do
jogo, acompanhando um aumento progressivo de dificuldade. As funcionalidades
são disponibilizadas ao jogador desde o início, porém apresentadas de forma
escalonada, favorecendo a assimilação natural das mecânicas e dos princípios
envolvidos por meio da interação.

As segunda e terceira fases, que incluem aprofundamento conceitual e expansão
narrativa, serão desenvolvidas em trabalhos futuros por exigirem maior tempo e
detalhamento, conforme indicado na \autoref{sec:trabalhos_futuros}.

A estratégia instrucional adotada, Aprendizagem Baseada em Jogos
\cite{coffey2009digital}, incentiva o aprendizado implícito, no qual os
conceitos emergem da experiência prática: em vez de instruções explícitas, o
progresso ocorre a partir da resolução dos desafios e da exploração das
mecânicas, permitindo que o conhecimento seja construído de maneira
contextualizada e integrada ao \emph{gameplay}.
