\begin{table}[H]
	\caption{Mecânicas Centrais}
	\label{tab:mecanicas_centrais}
	\centering
	\footnotesize
  \begin{tabular}{lp{8cm}}
		\toprule
		\textbf{Mecânica} & \textbf{Descrição} \\
		\midrule
    Manipulação de Inventários & Cada inventário é estruturado com base em uma
    estrutura de dados específica, permitindo, em casos específicos, operações
    como inserção, remoção, ordenação e movimentação de ponteiros. Essas
    operações preservam as propriedades de cada estrutura, proporcionando uma
    experiência prática de seus comportamentos. \\
		\midrule
    Geração de Elementos & Permite ao jogador criar novos elementos mediante o
    consumo de mana. Os elementos gerados são automaticamente alocados no
    primeiro espaço livre de um dos inventários, seguindo a ordem da esquerda para a
    direita. \\
		\midrule
    Consumíveis & Itens de uso estratégico que possibilitam realizar ações
    diretamente sobre os inventários, como inserir, remover ou ordenar
    elementos, servindo como ferramentas para reforçar o aprendizado sobre as
    operações das estruturas de dados. \\
		\midrule
    Mistura de Elementos & Elementos do mesmo tipo podem ser combinados para
    gerar ataques. O dano causado varia conforme a quantidade de elementos
    iguais utilizados na combinação. A realização da mistura exige a remoção
    prévia dos elementos dos inventários correspondentes. \\
		\midrule
    Fraquezas & Cada inimigo apresenta vulnerabilidades específicas a
    determinados elementos. O jogador deve identificar e combinar corretamente
    os elementos para maximizar o dano e avançar com eficiência.
    \\
		\midrule
    Sistema de Pontuação & Calcula o desempenho do jogador com base em
    múltiplos fatores, como acertos e erros nas combinações, número de
    tentativas e tempo total para concluir a fase, promovendo uma avaliação
    quantitativa da performance. \\
		\midrule
    Livro de Tutoriais & aqui são armazenados de forma explícita os tutoriais do
    jogo para que, caso o jogador se perca, ele possa revisar algo que não tenha
    entendido. \\
		\bottomrule
	\end{tabular}
  \caption*{Fonte: Autor}
\end{table}
