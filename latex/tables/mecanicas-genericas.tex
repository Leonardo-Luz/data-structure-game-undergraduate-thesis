\begin{table}[H]
	\caption{Mecânicas Genéricas}
	\label{tab:mecanicas_genericas}
	\centering
	\footnotesize
  \begin{tabular}{lp{8cm}}
		\toprule
		\textbf{Mecânica} & \textbf{Descrição} \\
		\midrule
    Movimentação 2D & Permite ao jogador deslocar o personagem em um plano
    bidimensional, controlando direções horizontais, como andar para a esquerda
    e direita. \\
		\midrule
    Pulo Responsivo & Define uma mecânica de salto com resposta imediata aos
    comandos do jogador, garantindo sensação de controle preciso e previsível. \\
		\midrule
    Corrida & Possibilita ao personagem movimentar-se em velocidade aumentada
    mediante a sustentação de um comando específico, ampliando o dinamismo do
    deslocamento. \\
		\midrule
    Pontos de Controle & Permitem salvar o progresso do jogador em determinados
    trechos do nível, reduzindo a frustração causada por reinícios completos e
    otimizando o fluxo de tentativa e erro. \\
		\midrule
    Tentativas Limitadas & Estabelecem um número máximo de vidas ou tentativas antes do reinício total do nível, criando tensão adicional e incentivando o aprendizado a partir dos erros. \\
		\midrule
    Pontos de Vida & Representam a resistência do personagem diante de danos.
    Quando os pontos de vida chegam a zero, o jogador sofre penalidades, como
    perda de progresso ou reinício do nível. \\
		\midrule
    Mana & Recurso utilizado para a geração de elementos, cuja gestão
    estratégica é necessária para equilibrar poder ofensivo e conservação de
    energia durante os desafios. \\
		\midrule
    Pulo Duplo Situacional & Permite ao personagem realizar um segundo salto no
    ar apenas após coletar um orbe específico durante o salto. \\
		\midrule
    Capacidade de Aparar Certos Ataques & O jogador pode bloquear determinados
    projéteis inimigos utilizando um projétil próprio, neutralizando o dano e
    evitando impactos diretos. \\
		\bottomrule
	\end{tabular}
  \caption*{Fonte: Autor}
\end{table}
