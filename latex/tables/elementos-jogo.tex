{\footnotesize
  \begin{longtable}{|>{\raggedright\arraybackslash}p{4cm} p{9cm}|}
    \caption{Elementos do jogo e suas descrições} \label{tab:elementos_jogo} \\

    \hline
    \rowcolor{headergray}
    \textbf{Elemento} & \textbf{Descrição} \\
    \hline
    \endfirsthead

    \hline
    \rowcolor{headergray}
    \textbf{Elemento} & \textbf{Descrição} \\
    \hline
    \endhead

    \multicolumn{2}{|r|}{\textit{Continua na próxima página}} \\
    \hline
    \endfoot

    \hline
    \caption*{Fonte: Autor}
    \endlastfoot

    \rowcolor{accent}
    \multicolumn{2}{|l|}{\textbf{Entidades Jogáveis e Interativas}} \\
    \hline
    Jogador & Personagem principal, um alquimista caracterizado como um
    \textit{plague doctor}, responsável por realizar combinações alquímicas e
    interagir com o ambiente. \\
    \hline
    Consumíveis & Itens coletáveis que restauram atributos ou concedem vantagens
    temporárias. \\
    \hline
    Fogueira & Ponto de controle utilizado para salvar o progresso e restabelecer o
    estado do jogador. \\
    \hline
    Gato & Personagem auxiliar que fornece dicas e interage em momentos
    específicos, contribuindo para o enredo e orientação do jogador. \\
    \hline
    Bandeira & Objeto que indica o término da fase, sendo necessário interagir com
    ele para que a pontuação final seja exibida. \\
    \hline

    \rowcolor{accent}
    \multicolumn{2}{|l|}{\textbf{Inimigos}} \\
    \hline
    Gosma & Inimigo básico que se desloca lateralmente em patrulha constante. \\
    \hline
    Cogumelo & Inimigo que persegue o jogador ao entrar em seu alcance e libera
    esporos quando próximo. \\
    \hline
    Olho Alado & Inimigo aéreo que ataca à distância disparando projéteis. \\
    \hline

    \rowcolor{accent}
    \multicolumn{2}{|l|}{\textbf{Ambiente e Cenário}} \\
    \hline
    Objetos de ambientação & Elementos decorativos como grama, arbustos, cercas e
    rochas. \\
    \hline
    Partículas & Efeitos visuais dinâmicos como poeira, fumaça, chuva e vagalumes,
    que enriquecem a atmosfera. \\
    \hline
    Blocos do cenário & Estruturas modulares que compõem o solo e as plataformas do
    ambiente de jogo. \\
    \hline
    \emph{Background} & Representa o céu e o horizonte, compondo o plano mais
    distante do cenário. \\
    \hline
    \emph{Background} Distante de Árvores & Camada intermediária que exibe árvores
    ao longe. \\
    \hline
    \emph{Background} Próximo & Camada mais próxima ao jogador, contendo árvores e
    vegetação detalhada. \\
    \hline
    Camada Frontal de Árvores & Elementos que passam à frente do jogador, criando
    sensação de profundidade. \\
    \hline
    Camada Frontal de Rochas e Arbustos & Sombras e detalhes em primeiro plano que
    reforçam o volume do cenário. \\
    \hline

    \rowcolor{accent}
    \multicolumn{2}{|l|}{\textbf{Interface e Menus}} \\
    \hline
    HUD & Interface principal que exibe informações como pontos de vida,
    tentativas, inventário, livro de tutoriais e botão de pausa. \\
    \hline
    Mensagens Temporários & Mensagens contextuais apresentadas de forma breve,
    utilizadas para instruções rápidas. \\
    \hline
    Menu Principal & Tela inicial que permite o acesso às demais seções do jogo. \\
    \hline
    Menu de Fases & Interface de seleção de níveis disponíveis. \\
    \hline
    Menu de Pause & Tela de pausa com acesso às opções e retorno à partida. \\
    \hline
    Menu de Opções & Acesso às opções de volume e idioma. \\
    \hline
    Menu de Volume & Controle individual dos níveis sonoros do jogo. \\
    \hline
    Menu de Linguagem & Seleção de idioma. \\
    \hline
    Menu de Saída & Opção para encerrar o jogo. \\
  \end{longtable}
}
