\begin{table}[H]
	\caption{Comparação entre trabalhos relacionados e o trabalho proposto}
	\label{tab:cmp_trabalhos_relatos}
	\centering
	\footnotesize
	\begin{tabular}{r|clllll}
		\toprule
		\textbf{Trabalho}      & \textbf{JD}  & \textbf{CED}    & \textbf{BO}    & \textbf{Ensino}    & \textbf{Estilo}      & \textbf{Gênero}   \\
		\midrule
		Condigjob              & Sim          & Não             & Não            & Explícito          & Simulador            & Puzzle            \\
		CodeBô                 & Sim          & F,L,P           & Busca          & Explícito          & Isométrico           & Puzzle            \\
		CodeBo Unplugged       & Não          & P               & Não            & Explícito          & Tabuleiro            & Puzzle            \\
		AuxED                  & Sim          &                 & Busca          & Explícito          & P\&C                 & Puzzle            \\
		Prog-poly              & Não          & Não             & Não            & Explícito          & Tabuleiro            & Quiz              \\
		\rowcolor{headergray}
		\textbf{Este trabalho} & \textbf{Sim} & \textbf{P,F,LE} & \textbf{Ambos} & \textbf{Implícito} & \textbf{Plataformer} & \textbf{Aventura} \\
		\bottomrule
	\end{tabular}

	\vspace{1.25em}
	\begin{minipage}{0.8\linewidth}
		\footnotesize
		\textbf{JD:} Jogo Digital. \\
		\textbf{CED:} Conceitos de Estrutura de Dados utilizados -
		\textbf{P:} Pilha, \textbf{F:} Fila, \textbf{L:} Lista, \textbf{LE:} Lista Encadeada, \textbf{LDE:} Lista Duplamente Encadeada, \textbf{AB:} Árvore Binária. \\
		\textbf{BO:} Utiliza algoritmos de Busca e Ordenação. \\
		\textbf{Ensino:} Forma de abordagem educacional (Explícito ou Implícito). \\
		\textbf{Estilo:} Estilo de interação do jogo. - \textbf{P\&C:} Point and Click \\
		\textbf{Gênero:} Categoria do jogo.
	\end{minipage}
\end{table}
