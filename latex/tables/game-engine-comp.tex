\begin{table}[H]
  \caption{Comparativo entre \emph{Game Engines}}
  \label{tab:comparativo_engines}
  \centering
  \footnotesize
\begin{tabular}{|
>{\raggedright\arraybackslash}p{3cm}|
>{\raggedright\arraybackslash}p{3cm}
>{\raggedright\arraybackslash}p{3cm}
>{\raggedright\arraybackslash}p{3cm}|
}
\hline
\rowcolor{headergray}
    \textbf{Critério} &
    \textbf{Godot Engine} &
    \textbf{Unity} &
    \textbf{Unreal Engine} \\
\hline
    \textbf{Licença} &
    MIT (código aberto) &
    Proprietária (gratuita para uso educacional) &
    Proprietária (royalties comerciais) \\
\hline
    \textbf{Facilidade de uso} &
    Alta &
    Alta &
    Moderada \\
\hline
    \textbf{Performance} &
    Boa &
    Muito boa &
    Muito boa \\
\hline
    \textbf{Suporte a WebGL} &
    Estável, mas limitado &
    Amplo e otimizado &
    Limitado e com alto custo de compilação \\
\hline
    \textbf{Comunidade e Documentação} &
    Em crescimento &
    Ampla e consolidada &
    Consolidada, porém mais voltada a estúdios \emph{Triple
    A}\tablefootnote{Estúdios que desenvolvem jogos com os maiores orçamentos
    da indústria.} \\
\hline
  \end{tabular}
  \caption*{Fonte: Autor, \cite{godotmanual, unitymanual, unrealmanual}}
\end{table}
