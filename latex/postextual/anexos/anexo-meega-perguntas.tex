\chapter{Instrumento de Avaliação: Modelo MEEGA+} \label{anexo:meega_original}

O instrumento de avaliação utilizado neste trabalho, detalhado no \autoref{apendice:questionario}, é
uma adaptação do modelo MEEGA+ \cite{meega2020}. Para fins de referência e
documentação do modelo original, a \autoref{tab:meega_original} apresenta a
listagem completa dos 35 itens do questionário MEEGA+, conforme proposto por
\citeonline{meega2020}. Todos os itens são avaliados em uma escala Likert de 5
pontos, onde 1 significa \enquote{Discordo Totalmente} e 5 significa \enquote{Concordo
Totalmente}, abrangendo os construtos de Usabilidade, Engajamento e Eficácia
Pedagógica.

{\footnotesize
\begin{longtable}{|
    >{\centering\arraybackslash}p{2cm}|
    >{\centering\arraybackslash}p{3cm}|
    >{\centering\arraybackslash}p{0.5cm}|
    >{\raggedright\arraybackslash}p{6.5cm}|
}
\caption{Itens do questionário do modelo MEEGA+} \label{tab:meega_original} \\

\hline
\rowcolor{headergray}
\textbf{Dimensão} & \textbf{Subdimensão} & \textbf{Item} & \textbf{Descrição do Item} \\
\hline
\endfirsthead

\hline
\rowcolor{headergray}
\textbf{Dimensão} & \textbf{Subdimensão} & \textbf{Item} & \textbf{Descrição do Item} \\
\hline
\endhead

\multicolumn{4}{|r|}{\textit{Continua na próxima página}} \\
\hline
\endfoot

\hline
\caption*{Fonte: \cite{meega2020}}
\endlastfoot

\multirow{12}{2cm}{\centering\textbf{Usabilidade}}
  & Estética & 1 & O design do jogo é atraente (interface, gráficos, tabuleiro, cartas, etc.). \\ \cline{2-4}
  & Estética & 2 & Os textos, cores e fontes combinam e são consistentes. \\ \cline{2-4}
  & Aprendizibilidade & 3 & Eu precisei aprender poucas coisas para poder começar a jogar o jogo. \\ \cline{2-4}
  & Aprendizibilidade & 4 & Aprender a jogar este jogo foi fácil para mim. \\ \cline{2-4}
  & Aprendizibilidade & 5 & Eu acho que a maioria das pessoas aprenderiam a jogar este jogo rapidamente. \\ \cline{2-4}
  & Operabilidade & 6 & Eu considero que o jogo é fácil de jogar. \\ \cline{2-4}
  & Operabilidade & 7 & As regras do jogo são claras e compreensíveis \\ \cline{2-4}
  & Acessibilidade & 8 & As fontes (tamanho e estilo) utilizadas no jogo são legíveis. \\ \cline{2-4}
  & Acessibilidade & 9 & As cores utilizadas no jogo são compreensíveis. \\ \cline{2-4}
  & Acessibilidade & 10 & O jogo permite personalizar a aparência (fonte e/ou cor) conforme a minha necessidade. \\ \cline{2-4}
  & Proteção contra erros do usuário & 11 & O jogo me protege de cometer erros. \\ \cline{2-4}
  & Proteção contra erros do usuário & 12 & Quando eu cometo um erro é fácil de me recuperar rapidamente. \\
\hline

\multicolumn{2}{|c|}{Confiança} & 13 & Quando olhei pela primeira vez o jogo, eu tive a impressão de que seria fácil para mim. \\ \hline
\multicolumn{2}{|c|}{Confiança} & 14 & A organização do conteúdo me ajudou a estar confiante de que eu iria aprender com este jogo. \\ \hline
\multicolumn{2}{|c|}{Desafio} & 15 & Este jogo é adequadamente desafiador para mim. \\ \hline
\multicolumn{2}{|c|}{Desafio} & 16 & O jogo oferece novos desafios (oferece novos obstáculos, situações ou variações) com um ritmo adequado. \\ \hline
\multicolumn{2}{|c|}{Desafio} & 17 & O jogo não se torna monótono nas suas tarefas (repetitivo ou com tarefas chatas). \\ \hline
\multicolumn{2}{|c|}{Satisfação} & 18 & Completar as tarefas do jogo me deu um sentimento de realização. \\ \hline
\multicolumn{2}{|c|}{Satisfação} & 19 & É devido ao meu esforço pessoal que eu consigo avançar no jogo. \\ \hline
\multicolumn{2}{|c|}{Satisfação} & 20 & Me sinto satisfeito com as coisas que aprendi no jogo. \\ \hline
\multicolumn{2}{|c|}{Satisfação} & 21 & Eu recomendaria este jogo para meus colegas. \\ \hline
\multicolumn{2}{|c|}{Interação social} & 22 & Eu pude interagir com outras pessoas durante o jogo. \\ \hline
\multicolumn{2}{|c|}{Interação social} & 23 & O jogo promove momentos de cooperação e/ou competição entre os jogadores. \\ \hline
\multicolumn{2}{|c|}{Interação social} & 24 & Eu me senti bem interagindo com outras pessoas durante o jogo. \\ \hline
\multicolumn{2}{|c|}{Diversão} & 25 & Eu me diverti com o jogo. \\ \hline
\multicolumn{2}{|c|}{Diversão} & 26 & Aconteceu alguma situação durante o jogo (elementos do jogo, competição, etc.) que me fez sorrir. \\ \hline
\multicolumn{2}{|c|}{Atenção focada} & 27 & Houve algo interessante no início do jogo que capturou minha atenção. \\ \hline
\multicolumn{2}{|c|}{Atenção focada} & 28 & Eu estava tão envolvido no jogo que eu perdi a noção do tempo. \\ \hline
\multicolumn{2}{|c|}{Atenção focada} & 29 & Eu esqueci sobre o ambiente ao meu redor enquanto jogava este jogo. \\ \hline
\multicolumn{2}{|c|}{Relevância} & 30 & O conteúdo do jogo é relevante para os meus interesses. \\ \hline
\multicolumn{2}{|c|}{Relevância} & 31 & É claro para mim como o conteúdo do jogo está relacionado com a disciplina. \\ \hline
\multicolumn{2}{|c|}{Relevância} & 32 & O jogo é um método de ensino adequado para esta disciplina. \\ \hline
\multicolumn{2}{|c|}{Relevância} & 33 & Eu prefiro aprender com este jogo do que de outra forma (outro método de ensino). \\ \hline
\multicolumn{2}{|c|}{Aprendizagem percebida} & 34 & O jogo contribuiu para a minha aprendizagem na disciplina. \\ \hline
\multicolumn{2}{|c|}{Aprendizagem percebida} & 35 & O jogo foi eficiente para minha aprendizagem, em comparação com outras atividades da disciplina \\
\end{longtable}}
