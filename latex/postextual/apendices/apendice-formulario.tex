\chapter{INSTRUMENTO DE AVALIAÇÃO UTILIZADO} \label{apendice:questionario}

\fixme{trocar por tabela ?}

Este apêndice apresenta o questionário aplicado aos 11 participantes da
avaliação, adaptado do modelo MEEGA+ \cite{meega2020}. O questionário
utilizou, em grande parte, uma escala Likert de 5 pontos, onde 1 significa \enquote{Discordo
Totalmente} e 5 significa \enquote{Concordo Totalmente}.

\section*{Perfil do Participante}
\begin{enumerate}
    \item Idade*
    \item Gênero (Masculino, Feminino, Outro)
    \item Você possui conhecimento prévio na área de desenvolvimento de software/programação?* (Sim/Não)
    \item Você costuma jogar jogos digitais?* (Sim/Não)
    \item Com qual frequência você joga jogos digitais? (Escala Linear de 1 a 5)
\end{enumerate}

\section*{Conteúdo da Área}
Esta seção avalia a percepção dos usuários sobre a representação e utilidade
pedagógica dos conceitos de Estruturas de Dados no jogo.
\begin{enumerate}
    \item Qual o seu nível de expertise em programação/desenvolvimento? (Escala Linear de 1 a 5)
    \item Os conceitos de estruturas de dados (pilha, fila e lista encadeada)
      foram bem representados no jogo, mesmo que de forma implícita.* (Escala
      Linear de 1 a 5)
    \item O jogo é útil para treinar conceitos já conhecidos.* (Escala Linear de 1 a 5)
    \item Deseja comentar algo sobre como o jogo ensinou os conceitos ou sugerir melhorias? (qualitativa)
\end{enumerate}

\section*{Sobre o Jogo}
\begin{itemize}
    \item A abordagem implícita (ensinar sem explicar diretamente) foi positiva.* (Escala Linear de 1 a 5)
    \item Você ficou preso ou sem entender algo em algum momento? Se sim, descreva. (qualitativa)
    \item Comentários gerais sobre o jogo: (qualitativa)
\end{itemize}

\section*{Avaliação segundo o Modelo MEEGA+ (Adaptado) - Experiência do Jogador}
Esta seção avalia o Engajamento, a Diversão, a Satisfação, a Imersão e o Desafio. \\
1 - Discordo totalmente \\
5 - Concordo totalmente
\begin{enumerate}
    % Engajamento
    \item O jogo conseguiu manter meu interesse.
    \item Eu estava motivado a continuar jogando.
    \item O jogo apresentou estímulos que mantiveram minha atenção.

    % Diversão
    \item Jogar este jogo foi divertido.

    % Satisfação
    \item Senti satisfação ao completar desafios ou derrotar inimigos.
    \item Eu recomendaria este jogo para meus colegas.

    % Desafio
    \item O nível de desafio foi adequado para mim.
    \item Os desafios me incentivaram a tentar melhorar.

    % Imersão
    \item Eu estava tão envolvido no jogo que eu perdi a noção do tempo.
    \item Eu esqueci sobre o ambiente ao meu redor enquanto jogava este jogo.
\end{enumerate}

\section*{Avaliação segundo o Modelo MEEGA+ (Adaptado) - Usabilidade}
Esta seção avalia a Estética, a Aprendibilidade, a Operabilidade, a Acessibilidade e a Proteção contra Erros. \\
1 - Discordo totalmente \\
5 - Concordo totalmente
\begin{enumerate}
    % Estética
    \item O design do jogo é atraente.
    \item Os textos, cores e fontes combinam e são consistentes.

    % Aprendibilidade
    \item Aprender a jogar este jogo foi fácil para mim.

    % Operabilidade
    \item Eu considero que o jogo é fácil de jogar.
    \item As regras do jogo são claras e compreensíveis.

    % Acessibilidade
    \item As fontes (tamanho e estilo) utilizadas no jogo são legíveis.
    \item As cores utilizadas no jogo são compreensíveis.

    % Proteção Contra Erros
    \item Quando eu cometo um erro é fácil de me recuperar rapidamente.
\end{enumerate}

\section*{Obrigado por jogar e por contribuir com este trabalho!}
\begin{enumerate}
    \item Deseja deixar algum comentário final? (qualitativa)
\end{enumerate}
