\chapter{DADOS BRUTOS DA AVALIAÇÃO} \label{apendice:dados_brutos}

Este apêndice contém a transcrição integral dos dados obtidos através do
questionário aplicado aos 11 participantes (P1 a P11). As respostas são
segregadas em seis tabelas conforme a estrutura do instrumento de coleta de
dados.

\section*{Tabela 1: Perfil do Participante}

\begin{table}[H]
    \caption{Dados de Perfil dos Participantes ($\text{N}=11$)}
    \label{tab:dados_perfil_apendice}
    \centering
    \footnotesize
    \begin{tabular}{|c|c|c|c|c|c|}
        \hline
        \rowcolor{headergray}
        \textbf{ID} & \textbf{Idade (Q1)} & \textbf{Gênero (Q2)} & \textbf{Conhece Prog.? (Q3)} & \textbf{Joga Jogos? (Q4)} & \textbf{Frequência (1-5) (Q5)} \\
        \hline
        P1  & 31-40 & M & Sim & Sim & 5 \\
        P2  & 18-30 & M & Sim & Sim & 5 \\
        P3  & 18-30 & M & Não & Sim & 4 \\
        P4  & <17 & M & Não & Sim & 4 \\
        P5  & 18-30 & M & Sim & Sim & 5 \\
        P6  & 18-30 & M & Sim & Sim & 5 \\
        P7  & 18-30 & M & Sim & Sim & 5 \\
        P8  & 18-30 & M & Sim & Sim & 5 \\
        P9  & 18-30 & M & Sim & Sim & - \\
        P10 & 18-30 & M & Sim & Sim & 5 \\
        P11 & 18-30 & M & Sim & Sim & 3 \\
        \hline
    \end{tabular}
    \caption*{Fonte: Autor. *Frequência (Q5): 1 = Raramente, 5 = Diariamente. -: Dado não preenchido ou não aplicável.}
\end{table}

\section*{Tabela 2: Conteúdo da Área}

\begin{table}[H]
    \caption{Respostas Quantitativas: Conteúdo da Área ($\text{N}=11$)}
    \label{tab:dados_conteudo}
    \centering
    \footnotesize
    \begin{tabular}{|c|c|c|c|}
        \hline
        \rowcolor{headergray}
        \textbf{ID} & \textbf{Expertise Prog. (1-5) (Q6)} & \textbf{Conceitos Repres.? (1-5) (Q7)} & \textbf{Útil p/ Treinar? (1-5) (Q8)} \\
        \hline
        P1 & 3 & 4 & 5 \\
        P2 & 1 & 5 & 5 \\
        P3 & - & - & - \\
        P4 & - & - & - \\
        P5 & 3 & 4 & 3 \\
        P6 & 5 & 4 & 5 \\
        P7 & 3 & 5 & 5 \\
        P8 & 5 & 4 & 3 \\
        P9 & 4 & 5 & 4 \\
        P10 & 4 & 5 & 3 \\
        P11 & 5 & 5 & 5 \\
        \hline
    \end{tabular}
    \caption*{Fonte: Autor. *1 = Mais Baixo/Discordo Totalmente, 5 = Mais Alto/Concordo Totalmente. -: Dado não preenchido ou não aplicável.}
\end{table}

\section*{Tabela 3: Avaliação MEEGA+ (Experiência do Jogador)}

\begin{table}[H]
    \caption{Respostas Brutas da Escala Likert - Experiência do Jogador ($\text{N}=11$)}
    \label{tab:dados_meega_exp}
    \centering
    \scriptsize
    \begin{tabular}{|c|c|c|c|c|c|c|c|c|c|c|}
        \hline
        \rowcolor{headergray}
        \textbf{ID} & \textbf{Q12} & \textbf{Q13} & \textbf{Q14} & \textbf{Q15} & \textbf{Q16} & \textbf{Q17} & \textbf{Q18} & \textbf{Q19} & \textbf{Q20} & \textbf{Q21} \\
        \hline
        P1 & 4 & 4 & 5 & 5 & 5 & 4 & 3 & 3 & 4 & 4 \\
        P2 & 5 & 5 & 5 & 5 & 5 & 5 & 5 & 5 & 4 & 4 \\
        P3 & 5 & 5 & 5 & 5 & 5 & 5 & 5 & 5 & 5 & 5 \\
        P4 & 4 & 4 & 3 & 4 & 4 & 4 & 5 & 2 & 2 & 5 \\
        P5 & 5 & 5 & 4 & 5 & 5 & 5 & 3 & 4 & 5 & 4 \\
        P6 & 5 & 5 & 5 & 5 & 5 & 5 & 5 & 5 & 4 & 3 \\
        P7 & 5 & 5 & 5 & 5 & 5 & 3 & 5 & 4 & 4 & 5 \\
        P8 & 5 & 4 & 4 & 5 & 4 & 5 & 5 & 5 & 3 & 3 \\
        P9 & 5 & 4 & 4 & 5 & 5 & 4 & 5 & 5 & 4 & 5 \\
        P10 & 4 & 4 & 4 & 5 & 3 & 4 & 5 & 4 & 5 & 4 \\
        P11 & 5 & 5 & 5 & 5 & 5 & 5 & 5 & 5 & 5 & 5 \\
        \hline
    \end{tabular}
    \caption*{Fonte: Autor. *1 = Discordo Totalmente, 5 = Concordo Totalmente. \textbf{Q12-Q14}: Engajamento; \textbf{Q15}: Diversão; \textbf{Q16-Q17}: Satisfação; \textbf{Q18-Q19}: Desafio; \textbf{Q20-Q21}: Imersão.}
\end{table}

\section*{Tabela 4: Avaliação MEEGA+ (Usabilidade) e Abordagem Implícita}

\begin{table}[H]
    \caption{Respostas Brutas da Escala Likert - Usabilidade e Abordagem Implícita ($\text{N}=11$)}
    \label{tab:dados_meega_usabilidade}
    \centering
    \footnotesize
    \begin{tabular}{|c|c|c|c|c|c|c|c|c|c|}
        \hline
        \rowcolor{headergray}
        \textbf{ID} & \textbf{Q9} & \textbf{Q22} & \textbf{Q23} & \textbf{Q24} & \textbf{Q25} & \textbf{Q26} & \textbf{Q27} & \textbf{Q28} & \textbf{Q29} \\
        \hline
        P1 & 5 & 5 & 5 & 2 & 2 & 4 & 5 & 5 & 3 \\
        P2 & 5 & 5 & 5 & 5 & 4 & 5 & 5 & 5 & 5 \\
        P3 & 5 & 5 & 5 & 4 & 3 & 5 & 5 & 5 & 5 \\
        P4 & 1 & 4 & 4 & 5 & 5 & 3 & 5 & 5 & 4 \\
        P5 & 4 & 4 & 5 & 4 & 3 & 2 & 4 & 5 & 3 \\
        P6 & 4 & 5 & 2 & 3 & 4 & 3 & 3 & 5 & 5 \\
        P7 & 5 & 5 & 5 & 3 & 3 & 4 & 5 & 5 & 4 \\
        P8 & 4 & 4 & 4 & 2 & 3 & 1 & 5 & 5 & 4 \\
        P9 & 4 & 5 & 5 & 5 & 4 & 4 & 5 & 5 & 3 \\
        P10 & 4 & 5 & 5 & 5 & 4 & 5 & 4 & 5 & 2 \\
        P11 & 5 & 5 & 5 & 4 & 4 & 5 & 5 & 5 & 5 \\
        \hline
    \end{tabular}
    \caption*{Fonte: Autor. *1 = Discordo Totalmente, 5 = Concordo Totalmente. \textbf{Q22-Q23}: Estética; \textbf{Q24}: Aprendibilidade; \textbf{Q25-Q26}: Operabilidade; \textbf{Q27-Q28}: Acessibilidade; \textbf{Q29}: Proteção Contra Erros. A questão \textbf{Q9} (Abordagem Implícita) está destacada como a primeira coluna.}
\end{table}

\section*{Tabela 5: Respostas Qualitativas (Conteúdo e Barreiras)}

Esta tabela agrupa as respostas abertas relacionadas à percepção pedagógica e às dificuldades encontradas no fluxo de jogo.

\begin{table}[H]
    \caption{Respostas Qualitativas: Conteúdo da Área e Barreiras no Jogo ($\text{N}=11$)}
    \label{tab:dados_qualitativos_1}
    \centering
    \scriptsize
    \begin{tabular}{|c|p{7cm}|p{7cm}|}
        \hline
        \rowcolor{headergray}
        \textbf{ID} & \textbf{Comentários sobre Ensino/Melhorias (Q8)} & \textbf{Ficou Preso ou sem Entender Algo? (Q10)} \\
        \hline
        P1 & Acho que o jogo apresenta o conceito de maneira simples mas efetiva. & Fiquei preso por causa do RNG de geração dos elementos. \\
        \rowcolor{accent}
        P2 & - & - \\
        P3 & - & demorei para entender que pra da dano tem que usar o elemento que esta na cabeça do inimigo. \\
        \rowcolor{accent}
        P4 & - & Não, porque o Silvio e o amigo dele explicou o jogo antes. \\
        P5 & talvez so diminuir um pouco a dificultade por causa que isso dificulta um pouco no começo para entender como funciona a pilha,fila e listra & - \\
        \rowcolor{accent}
        P6 & Gostei da forma em que mistura um jogo 2d , com um tetris, em que você tem que manejar sua mão de elementos e ao mesmo tempo ficar de olho na sua vida e nós inimigos, a dificuldade do jogo está um pouco alta mas ao mesmo tempo prende bastante a pessoa . Acho que poderia melhorar na fonte e no menu do jogo, organizar um pouco mais o menu e os tutoriais. & No começo, é um pouco estranho de entender, por que o tutorial está um pouco bagunçado, não sei se é por conta da fonte e etc, mas assim que você começa a jogar você consegue aprender. \\
        P7 & - & - \\
        \rowcolor{accent}
        P8 & - & Sim, a mecanica de magia é um pouco complexa para entender. O livro explica, mas acredito que seria interessante um tutorial mais guiado no começo. \\
        P9 & - & - \\
        \rowcolor{accent}
        P10 & Aumento do tempo de invulnerabilidade após hit e possibilidade de ao menos mudar de direção enquanto conjura elementos & - \\
        P11 & Achei o livro ótimo para estudar estrutura de dados, associar o inventário com suas respectivas estruturas e os consumíveis com suas respectivas operações de CRUD. & - \\
        \hline
    \end{tabular}
    \caption*{Fonte: Autor. -: Dado não preenchido.}
\end{table}

\section*{Tabela 6: Respostas Qualitativas (Comentários Gerais e Finais)}

Esta tabela conclui a transcrição dos dados qualitativos, agrupando os comentários abertos sobre a experiência geral e considerações finais.

\begin{table}[H]
    \caption{Respostas Qualitativas: Comentários Gerais e Finais ($\text{N}=11$)}
    \label{tab:dados_qualitativos_2}
    \centering
    \scriptsize
    \begin{tabular}{|c|p{7cm}|p{7cm}|}
        \hline
        \rowcolor{headergray}
        \textbf{ID} & \textbf{Comentários Gerais sobre o Jogo (Q11)} & \textbf{Comentário Final (Q30)} \\
        \hline
        P1 & Acredito que o principal aspecto de dificuldade do jogo é o RNG da geração de elementos, que faz com que muitas vezes seja necessário descartar elementos para tornar possível combinar os certos para derrotar o inimigo. Se fosse possível amenizar esta questão, acho que a jogabilidade ficaria mais fluída e divertida. & Eu tive bastante dificuldade para fazer o pulo antes da arena, onde é preciso segurar o SPACE, porque muitas vezes não registrava o comando de pulo e eu caia direto. Não sei se isso é bug ou o meu teclado com problemas. \\
        \rowcolor{accent}
        P2 & - & - \\
        P3 & A ideia de cada slot de elemento agir de uma forma diferente ficou bem legal. & O jogo fico muito bom com uma ideia de jogabilidade bem diferente. \\
        \rowcolor{accent}
        P4 & Bem bacana, espero a segunda fase, se tiver record mundial vou fazer kkkkk. & Ficou daora o jogo. \\
        P5 & muito bom so que com dificultade um pouco grande por causa dos comados serem meios exoticos no meio & jogo muito massa, e muito bonitinho \\
        \rowcolor{accent}
        P6 & Jogo é muito interessante, a mecânica principal, gostei dos gráficos, acho que a única coisa que eu mudaria é a organização e o design do tutorial dos menus. Espero que lancem mais fases. & Parabéns pelo jogo , da pra ver que o criador colocou bastante tempo e carinho para cria-lo. \\
        P7 & - & - \\
        \rowcolor{accent}
        P8 & Show. Gostaria de poder realizar ações durante o pulo para ficar mais dinâmico. & - \\
        P9 & - & - \\
        \rowcolor{accent}
        P10 & Só achei que às vezes demora vir o elemento necessário já que tem que ter pelo menos dois e numa posição que seja possível mesclar. Fora isso, legal e bonito. & Parabéns leo! \\
        P11 & Muito bom, me diverti jogando e ainda reforcei conceitos de estrutura de dados. & Parabéns, o desenvolvedor desse jogo é extremamente competente e merece muito sucesso. \\
        \hline
    \end{tabular}
    \caption*{Fonte: Autor. -: Dado não preenchido.}
\end{table}
