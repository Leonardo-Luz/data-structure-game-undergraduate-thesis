\setlength{\absparsep}{18pt}

\begin{resumo}
  Estruturas de Dados é uma disciplina frequentemente associada a
  alta carga cognitiva e, consequentemente, a elevados índices de reprovação e
  evasão, principalmente devido à natureza abstrata de seus conceitos e à baixa
  motivação relatada pelos estudantes \cite{mtaho2024difficulties}. Nesse
  contexto, este trabalho propõe uma alternativa ao ensino tradicional por meio
  do desenvolvimento de um jogo sério \cite{mouaheb2012serious}, projetado para
  ensinar e reforçar conteúdos educacionais de maneira lúdica. O jogo foi
  criado utilizando, de forma adaptada, a metodologia de desenvolvimento
  ENgAGED \cite{battistella2016engaged}, em conjunto com a estratégia
  instrucional de Aprendizagem Baseada em Jogos (GBL) \cite{coffey2009digital}.

  Como resultado, foi desenvolvido um protótipo capaz de representar
  implicitamente conceitos fundamentais de estruturas de dados, como pilhas,
  filas e listas encadeadas, integrando-os às mecânicas do jogo. O protótipo
  foi avaliado por estudantes de Análise e Desenvolvimento de Sistemas e áreas
  correlatas, bem como por alguns participantes sem conhecimento prévio na
  área, utilizando uma versão adaptada do modelo de avaliação MEEGA+
  \cite{meega2020}. Os resultados indicam que o jogo apresenta boa usabilidade,
  proporciona uma experiência positiva ao jogador e comunica de forma eficaz os
  conteúdos propostos, embora tenham sido identificados alguns problemas no
  sistema de aleatoriedade da mecânica central e a necessidade de um tutorial
  mais guiado.

	\vspace{\onelineskip}
	\noindent
	\textbf{Palavras-chave}: Jogo Sério. Estruturas de Dados. Aprendizagem Baseada em Jogos.
\end{resumo}
