\documentclass{beamer}

% ---------------------------------------------------
% Configurações Gerais
% ---------------------------------------------------
\usepackage[utf8]{inputenc}
\usepackage[brazil]{babel}
\usepackage{graphicx}
\usepackage{booktabs}
\usepackage{hyperref}
\usepackage{caption}

\usetheme{default}
\usecolortheme{default}
\setbeamertemplate{navigation symbols}{}

% ---------------------------------------------------
% Cores e estilo minimalista claro
% ---------------------------------------------------
\definecolor{ifrsGreen}{RGB}{0,128,0}
\setbeamercolor{title}{fg=ifrsGreen}
\setbeamercolor{frametitle}{fg=ifrsGreen}
\setbeamercolor{structure}{fg=ifrsGreen}
\setbeamercolor{section in toc}{fg=ifrsGreen}

\setbeamertemplate{footline}[frame number]

% ---------------------------------------------------
% Informações do Trabalho
% ---------------------------------------------------
\title[Desenvolvimento de Jogo Sério]{Desenvolvimento de Jogo Sério para Auxílio no Ensino de Conceitos de Estruturas de Dados}
\author[Leonardo Luz Fachel]{\textbf{Leonardo Luz Fachel}}
\institute[IFRS]{Instituto Federal de Educação, Ciência e Tecnologia do Rio Grande do Sul \\ Campus Osório \\ Curso Superior de Tecnologia em Análise e Desenvolvimento de Sistemas}
\date[2025]{Orientador: Prof. Bruno Chagas Alves Fernandes \\ Osório, 2025}

% ---------------------------------------------------
\begin{document}

% ---------------------------------------------------
% CAPA INSTITUCIONAL
% ---------------------------------------------------
\begin{frame}[plain]
    \centering
    \vspace{0.5cm}
    \includegraphics[width=2cm]{images/ifrs-logo.png}\\[0.7cm]
    {\textbf{Instituto Federal de Educação, Ciência e Tecnologia do Rio Grande do Sul}}\\
    {Campus Osório}\\[2cm]

    {\textbf{\inserttitle}}\\[1cm]
    {\textbf{Aluno:} \insertauthor}\\[0.3cm]
    {\textbf{Orientador:} Prof. Bruno Chagas Alves Fernandes}\\[2cm]
    {Osório, 2025}
\end{frame}

% ---------------------------------------------------
\begin{frame}{Sumário}
  \tableofcontents
\end{frame}

% ---------------------------------------------------
\section{Introdução}
\begin{frame}{Introdução}
  \begin{itemize}
    \item O ensino de \textbf{Estruturas de Dados} é um desafio recorrente, com altos índices de reprovação e evasão.
    \item Conceitos abstratos e ensino tradicional reduzem a motivação dos alunos.
    \item \textbf{Jogos Sérios} surgem como alternativa promissora, promovendo o aprendizado ativo e engajado.
    \item Fundamentação no \textbf{Construcionismo de Papert (1993)}: o conhecimento é construído pela ação.
  \end{itemize}
\end{frame}

% ---------------------------------------------------
\section{Objetivos}
\begin{frame}{Objetivo Geral}
  \begin{block}{}
    Desenvolver um jogo sério que aborde conceitos fundamentais de estruturas de dados de forma implícita, por meio de mecânicas lúdicas e interativas.
  \end{block}
\end{frame}

\begin{frame}{Objetivos Específicos}
  \begin{itemize}
    \item Investigar modelos de jogos sérios aplicados ao ensino de computação.
    \item Projetar e implementar um jogo educacional fundamentado na metodologia \textbf{GAMED}.
    \item Incorporar representações implícitas de estruturas como \textbf{pilhas, filas e listas}.
    \item Avaliar a eficácia e a usabilidade do jogo.
    \item Analisar o feedback dos usuários e propor melhorias.
  \end{itemize}
\end{frame}

% ---------------------------------------------------
\section{Justificativa}
\begin{frame}{Justificativa}
  \begin{itemize}
    \item Dificuldade no ensino de conteúdos abstratos e baixa retenção de aprendizado.
    \item Necessidade de tornar o ensino mais motivador e interativo.
    \item Jogos sérios permitem contextualizar conceitos em narrativas envolventes.
    \item A metodologia \textbf{GAMED} garante equilíbrio entre aprendizagem e diversão.
  \end{itemize}
\end{frame}

% ---------------------------------------------------
\section{Referencial Teórico}
\begin{frame}{Referencial Teórico}
  \begin{itemize}
    \item \textbf{Estruturas de Dados:} essenciais na Computação, mas de difícil compreensão (Cormen, 2022).
    \item \textbf{Jogos Sérios:} aplicam design de jogos com objetivos educacionais (Mouaheb, 2012).
    \item \textbf{Game-Based Learning (GBL):} aprendizagem emergente da interação com o jogo (Malone \& Lepper, 2021).
    \item \textbf{Construcionismo:} aprendizado pela criação e exploração de artefatos (Papert, 1993).
  \end{itemize}
\end{frame}

\begin{frame}{Ferramenta de Desenvolvimento}
  \begin{itemize}
    \item A \textbf{Unity} foi escolhida como engine principal.
    \item Permite mecânicas avançadas e integração eficiente de estruturas de dados.
    \item Suporte multiplataforma e comunidade ampla de desenvolvedores.
  \end{itemize}
\end{frame}

% ---------------------------------------------------
\section{Metodologia}
\begin{frame}{Metodologia Científica}
  \begin{itemize}
    \item Pesquisa \textbf{aplicada}, de caráter \textbf{experimental} e abordagem \textbf{mista}.
    \item Combina métodos qualitativos e quantitativos.
    \item Instrumentos:
    \begin{itemize}
      \item Questionários estruturados e testes de desempenho.
      \item Observações e entrevistas durante o uso do jogo.
    \end{itemize}
  \end{itemize}
\end{frame}

\begin{frame}{Metodologia de Desenvolvimento}
  \begin{itemize}
    \item Baseada no processo \textbf{ENgAGED} (Battistella e Wangenheim, 2016).
    \item Integra design instrucional e design de jogos.
    \item Estruturada em 5 fases:
      \begin{enumerate}
        \item Análise da Unidade Instrucional
        \item Projeto do Jogo
        \item Desenvolvimento do Protótipo
        \item Testes com Usuários
        \item Avaliação da Aprendizagem
      \end{enumerate}
  \end{itemize}
\end{frame}

\begin{frame}{Fases do Processo ENgAGED}
  \begin{columns}
  \column{0.5\textwidth}
    \textbf{Análise e Projeto}
    \begin{itemize}
      \item Definição de objetivos e público-alvo.
      \item Criação de narrativa, regras e mecânicas.
    \end{itemize}

  \column{0.5\textwidth}
    \textbf{Desenvolvimento e Avaliação}
    \begin{itemize}
      \item Protótipo funcional e testes com estudantes.
      \item Avaliação de engajamento e aprendizagem.
    \end{itemize}
  \end{columns}
\end{frame}

% ---------------------------------------------------
\section{Trabalhos Relacionados}
\begin{frame}{Trabalhos Relacionados — Artigos}
  \begin{itemize}
    \item \textbf{CondigJob} (Costa, 2023): ensino de C, abordagem explícita.
    \item \textbf{CodeBô} (Araujo \& Silva, 2025): ensino implícito de pilhas, filas e listas.
    \item \textbf{Prog-Poly} (Nascimento, 2022): tabuleiro baseado em Monopoly.
    \item \textbf{Glatz et al. (2023):} jogo mobile sobre busca e ordenação.
  \end{itemize}
\end{frame}

\begin{frame}{Trabalhos Relacionados — Aplicativos}
  \begin{itemize}
    \item \textbf{Human Resource Machine} (2015): lógica de programação.
    \item \textbf{AlgoBot} (2018): puzzles e automação.
    \item \textbf{MOP’N SPARK} (2025): puzzles e ambientação fantasiosa.
    \item \textbf{Iron Ears} (2020): estruturas de dados em mecânicas produtivas.
  \end{itemize}
\end{frame}

% ---------------------------------------------------
\section{Desenvolvimento}
\begin{frame}{Desenvolvimento do Jogo}
  \begin{itemize}
    \item Gênero: \textbf{plataforma 2D}.
    \item Estilo visual: \textbf{pixel art}.
    \item Linguagem: \textbf{C\# com Unity}.
    \item Estruturas abordadas: \textbf{pilha, fila e lista}.
    \item Mecânicas educativas integradas de forma implícita às ações do jogador.
  \end{itemize}
\end{frame}

\begin{frame}{Design e Fases}
  \begin{itemize}
    \item Fases planejadas:
      \begin{itemize}
        \item Tutorial introdutório.
        \item Fase 1 — Boss Slime.
        \item Fase 2 — Mid Boss.
        \item Fase 3 — Boss Final (Alquimista rival).
      \end{itemize}
    \item Sons e feedbacks:
      \begin{itemize}
        \item Passos, pulos, conjuração, dano, HUD, progressão.
      \end{itemize}
  \end{itemize}
\end{frame}

% ---------------------------------------------------
\section{Resultados e Conclusões}
\begin{frame}{Resultados Parciais}
  \begin{itemize}
    \item Protótipo funcional com todas as mecânicas centrais.
    \item Feedback positivo dos primeiros usuários.
    \item Indícios de aumento no engajamento e compreensão.
  \end{itemize}
\end{frame}

\begin{frame}{Trabalhos Futuros}
  \begin{itemize}
    \item Implementar novas estruturas: \textbf{árvores} e \textbf{grafos}.
    \item Sistema de \textbf{conquistas} e \textbf{modo arcade}.
    \item Versão \textbf{multijogador e PVP}.
  \end{itemize}
\end{frame}

\begin{frame}{Conclusão}
  \begin{itemize}
    \item O jogo proposto é uma ferramenta promissora para o ensino de estruturas de dados.
    \item A integração entre entretenimento e aprendizado é essencial.
    \item Jogos sérios podem transformar a aprendizagem em uma experiência envolvente e significativa.
  \end{itemize}
\end{frame}

% ---------------------------------------------------
\section*{Referências}
\begin{frame}[allowframebreaks]{Referências}
  \tiny
  \bibliographystyle{apalike}
  \bibliography{refs}
\end{frame}

% ---------------------------------------------------
\begin{frame}
  \centering
  \Huge{\textbf{Obrigado!}}\\[0.5cm]
  \Large{Dúvidas e Perguntas?}
\end{frame}

\end{document}
